\chapter{Conclusions}
\label{ch:conclusion}

This thesis documents two closely related searches for the decays of charged \B mesons using a data set collected by the \lhcb detector during 2011-2016, corresponding to an integrated luminosity of 4.8\invfb.
The first, \decay{\Bp}{\Dsp\Kp\Km}, is observed for the first time, and the branching fraction is determined to be  
\begin{equation*}
\mathcal{B}(B^{+} \to D_s^{+}K^{+}K^{-} ) = (7.1 \pm 0.5 \pm 0.6 \pm 0.7) \times 10^{-6}, 
\end{equation*}
\noindent where the first uncertainty is statistical, the second systematic and the third due to the uncertainty on 
the branching fraction of the normalisation mode \decay{\Bp}{\Dsp\Dzb}.
The phase-space distributions of the selected \decay{\Bp}{\Dsp\Kp\Km} decays are investigated. 
% The sample of this three-body decay is analysed to examine the possible resonant structures that may contribute. 
% The decays are only observed at low values of the invariant mass of the two kaons, $m(\Kp\Km) < 1900 \mevcc$. 
% No obvious peaking structures are observed, however a study of the helicity angle as a function of $m(\Kp\Km)$ implies a spin-1 resonance contributes in the region $1500<m(\Kp\Km)<1750\mevcc$. 

The second part of the thesis searches for the annihilation topology decay \decay{\Bp}{\Dsp\phiz}. This is isolated by searching for decays in which the invariant mass of the two kaons, $m(\Kp\Km)$, is within a small window around the \phiz-meson mass. Input is used from the first part of this thesis to help determine how \decay{\Bp}{\Dsp\Kp\Km} decays that didn't proceed via a \phiz meson may contribute. No significant \decay{\Bp}{\Dsp\phiz} signal is observed and a limit of
\begin{equation*}
\mathcal{B}(B^{+} \to D_s^{+}\phi) < 4.9 \times 10^{-7}
\end{equation*}
is set on the branching fraction at 95\% confidence level.

% This supersedes a previous \lhcb measurement that found evidence for this decay using a data set corresponding to an integrated luminosity of 1\invfb recorded in 2011~\cite{LHCb-PAPER-2012-025}. The analysis presented in this thesis benefits from a larger data set and accounts for the sizeable contributions from $\decay{\Bp}{\Dsp\Kp\Km}$ decays within the $\phiz$-meson mass window that were previously neglected.


% In future, this work could be extended to a full amplitude analysis of \decay{\Bp}{\Dsp\Kp\Km} decays using the complete \lhcb Run I and Run II data sets. This would provide more information about the different resonances that contribute and account for any interference between the competing amplitudes.   


%%%%%%%%%%%%
The limit on the \decay{\Bp}{\Dsp\phiz} branching fraction presented here and in Ref~\cite{LHCb-PAPER-2017-032} supersede the previous evidence reported by the LHCb collaboration in Ref.~\cite{LHCb-PAPER-2012-025}. The updated analysis takes advantage of a much larger dataset now available at \lhcb. 
This update determines that there are sizeable contributions from $\decay{\Bp}{\Dsp\Kp\Km}$ decays within the $\phiz$-meson mass window that were previously neglected; these should have been considered. As such, this measurement paints a more detailed and comprehensive picture of the processes contributing to \decay{\Bp}{\Dsp\Kp\Km} decays.
Additionally, the result presented here is consistent with the prediction that rescattering contributions to $\decay{\Bp}{\Dsp\phiz}$ decays are small.

In future, a full amplitude analysis of \decay{\Bp}{\Dsp\Kp\Km} decays using the complete \lhcb Run I and Run II data sets could incorporate a further search for \decay{\Bp}{\Dsp\phiz} decays.


