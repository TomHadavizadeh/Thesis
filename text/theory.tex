\chapter{Theory} 
\label{ch:theory}

\minitoc

%What's the standard model all about? 

%\section{In the beginning...}
%\section{Some other stuff}



%\section{The Standard model of Particle of Physics}
\section{The weak force in \bquark-hadron decays}
Say how strong force complicated study of CKM physics.

\subsection{The weak force}
\subsection{Weak interactions of quarks}
\subsection{The CKM matrix}
\subsection{\bquark-hadron physics}
\subsection{QCD and hadronisation}



\section{Annihilation topology decays}

{\color{Red}
\begin{itemize}
\item Find origin of hadronic uncertainties 
\end{itemize}}


\subsection{Pure annihilation topology decays}
{\color{Red}
\begin{itemize}
\item Pure annihilation - charmless Bc
\end{itemize}}
\subsection{Other annihilation decays}

{\color{Red}
\begin{itemize}
\item non pure - donals Bc 
\end{itemize}}

\section{Rescattering}

{\color{Red}
\begin{itemize}
\item What is rescattering?
\item Examples of it?
\item Why it's expected to be small
\end{itemize}}

\section{Theoretical predictions for \decay{\Bp}{\Dsp\phiz} decay}
{\color{Blue}
In the SM, the decay $\decay{\Bp}{\Dsp\phi}$ proceeds dominantly via the annihilation diagram shown in Fig.~\ref{fig:DsPhiDiagram}. 
This suppressed topology requires the wave functions of the incoming quarks to overlap sufficiently to annihilate into a virtual \Wp boson. 
The decay is further suppressed by the small magnitude of the CKM matrix element \Vub associated with the annihilation vertex. 


}

\subsection{Standard model predictions}
{\color{Blue}
Several SM predictions have been made for the branching fraction of the $\decay{\Bp}{\Dsp\phi}$ decay~\cite{Zou:2009zza, Mohanta:2002wf, Mohanta:2007uu, Lu:2001yz}, using input from lattice calculations~\cite{fB:2013HPQCD,fB:2016ETM, fB:2016Fermi}. These predictions are in the range $(1-7)\times10^{-7}$, where the limit on the precision is dominated by hadronic uncertainties. 

In addition, unlike many rare hadronic decays including $\decay{\Bp}{\Dsp\Kp\Km}$, possible contributions from rescattering effects are expected to be small, for example contributions from intermediate states such as $\decay{\Bp}{\Dsp\omega}$~\cite{Gronau:2012gs}.

}

{\color{Red}
\begin{itemize}
\item Find origin of hadronic uncertainties 
\end{itemize}}

\subsection{BSM models and predictions}
{\color{Blue}
However, additional diagrams contributing to this decay can arise in some extensions of the SM, such as supersymmetric models with R-parity 
violation. They could enhance the branching fraction and/or produce large \CP asymmetries~\cite{Mohanta:2002wf, Mohanta:2007uu}, which makes the $\decay{\Bp}{\Dsp\phi}$ decay a promising place to search for new physics beyond the SM.\footnote{Charge conjugation is implied throughout this paper. Furthermore, $\phi$ denotes the $\phi(1020)$ resonance.}
}

{\color{Red}
\begin{itemize}
\item Higgs doublet
\item SUSY 
\end{itemize}}

\subsection{Previous measurements}

{\color{Blue}
The \lhcb experiment reported evidence for the decay $\decay{\Bp}{\Dsp\phi}$ using $pp$ collision data corresponding to an integrated luminosity of 1\invfb taken during 2011, at a centre-of-mass energy of 7\tev~\cite{Aaij:2012zh}. A total of $6.7^{+4.5}_{-2.6}$ candidates was observed. The branching fraction was determined to be 

\begin{equation}
\mathcal{B}(\decay{\Bp}{\Dsp\phi}) = (1.87^{+1.25}_{-0.73} \pm 0.19 \pm 0.32) \times 10^{-6},
\end{equation}
where the first uncertainty is statistical, the second is systematic and the third is due to the uncertainty on the branching fraction of the decay $\decay{\Bp}{\Dsp\Dzb}$, which was used as normalisation. 
Given the large uncertainties on both the theoretical and experimental values, the previously measured value is consistent with the range of SM values given above.
}

{\color{Red}
\begin{itemize}
\item Include plot and measurement
\item say something about similarites 
\end{itemize}}


\section{Theoretical predictions for \decay{\Bp}{\Dsp\Kp\Km} decay}



{\color{Blue}
The decay $\decay{\Bp}{\Dsp\Kp\Km}$ is mediated by a $\decay{\bquarkbar}{\uquarkbar}$ transition shown in Fig.~\ref{fig:DsPhiDiagram} and is therefore suppressed in the Standard Model (SM) due to the small size of the Cabibbo-Kobayashi-Maskawa (CKM) matrix element \Vub. 
}


{\color{Red}
\begin{itemize}
\item explain what a dalitz plot is
\end{itemize}
}

\subsection{Standard model predictions}
{\color{Blue}
The branching fraction for this decay is currently not measured, however a similar decay, \decay{\Bp}{\Dsp \piz}, has been observed with a branching fraction of $\mathcal{B}(\decay{\Bp}{\Dsp \piz}) = (1.5 \pm 0.5) \times 10^{-5}$~\cite{Aubert:2006xy}.
}


{\color{Red}
\begin{itemize}
\item talk about \decay{\Bp}{\Dsp\piz}
\item Could talk about \decay{\Bp}{\Dp\Kp\pim} and estimate   
\end{itemize}}

\subsection{Previous measurements}



