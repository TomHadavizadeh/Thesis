\chapter{Introduction} 
\label{ch:introduction}

\minitoc 



Overview 

The research presented in this thesis has been published in Ref.~\cite{LHCb-PAPER-2017-032}. 

{\color{Red}
\begin{itemize}
\item Historical context 
\item 
\end{itemize}}



\section{The Standard model of Particle of Physics}

The standard model (SM) provides explanation for a large number of natural phenomena to a high precision. There are currently a number of unresolved issues that are not explained, implying that this theory is not the complete description of the natural world.



\subsection{Building blocks}

The SM is composed of a small number of fundamental particles; some are the components of matter and others are force-carriers that mediate interactions.




{\color{Red}
\begin{itemize}
\item Fermions and bosons
\item QFT?
\item people?
\item Feynman diagrams
\end{itemize}}

\subsubsection{Fundamental fermions}

The fundamental spin-$1/2$ fermions that constitute matter can be divided into two categories: leptons and quarks. 
These particles can be divided up into \emph{families} or \emph{generations} containing two leptons and two quarks. Each of these contains an up-type and down-type quark, a charged lepton and a neutrino. There are three of these \emph{generations} that are identical clones of one another, differing only by their rest masses; each subsequent generation is heavier than the previous. Their interactions are identical in the SM. 
These particles are listed in Table~\ref{tab:intro_particles} along with their electromagnetic charge.




\begin{table}[h]
   \begin{center}
      \begin{tabular}{lcc | lcc}
         \hline
         \multicolumn{3}{c|}{Quarks} & \multicolumn{3}{c}{Leptons}\\
         \hline
         Name       & Symbol            & charge  & Name                & Symbol            & charge    \\ 
         \hline
         Up         & \uquark           &  $+2/3$ & Electron neutrino   & \neue             &  $0$   \\ 
         Down       & \dquark           &  $-1/3$ & Electron            & \en               &  $-1$  \\ 
         \hline
         Charm      & \cquark           &  $+2/3$ & Muon neutrino       & \neum             &  $0$   \\ 
         Strange    & \squark           &  $-1/3$ & Muon                & \mun              &  $-1$  \\ 
         \hline
         Top        & \tquark           &  $+2/3$ & Tau neutrino        & \neut             &  $0$   \\ 
         Bottom     & \bquark           &  $-1/3$ & Tau                 & \taum             &  $-1$  \\ 
         \hline
      \end{tabular}
   \end{center}
   \caption{The fundamental fermions in the standard model. The electromagnetic charges are given in units of the absolute value of the electron charge: $1.6\times 10^{-19}$\,C.}
   \label{tab:intro_particles}
\end{table}


\subsubsection{Forces}

The SM provides a mechanism for three fundamental forces: the electromagnetic, weak and strong forces. Each of these are mediated by force-carriers or gauge bosons

\begin{table}[h]
   \begin{center}
      \begin{tabular}{cccc}
         \hline

         Force              & Mediator      & Symbol    & charge               \\
         \hline 
         Electromagnetic    & Photon        & \Pgamma   & \\
         Weak               & $W$ boson     & \Wpm      & \\
         Strong             & gluons        & $g$       & \\
         Gravity            &               &           & \\

         \hline
      \end{tabular}
   \end{center}
   \caption{The fundamental interactions in the standard model.}
   \label{tab:lumi}
\end{table}

\subsection{The origin of mass}
The final component of the SM is the Higgs boson, a key component to allow the fundamental particles to gain mass.

\subsection{Parameters}


\subsection{Triumphs}
\subsection{Inadequacies}
\begin{itemize}
\item Baryon asymmetry 
\item Neutrino masses
\item Gravity
\item Dark matter
\item hierarchy problem
\end{itemize}
%\subsection{QCD, QED, Weak}

