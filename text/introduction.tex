\chapter{Introduction} 
\label{ch:introduction}

\minitoc 



%\section{Overview} 


Throughout history, society has endeavoured to glimpse into the heart of matter, breaking it down into it's elementary building blocks.
% The term \emph{atom} was originally used by the ancient Greek philosophers to speculate on the nature of matter and used to denote something that was indivisible. 
% However, modern particle physics took until the 19th century to get going; the first of today's elementary particles, the electron, was discovered by J.J. Thomson in 1897~\cite{electron}.
Although the term \emph{atom} was originally used by the Ancient Greek philosophers to describe indivisible particles, modern particle physics took until the 19th century to take off; the first of today's elementary particles, the electron, was discovered by J.J. Thomson in 1897~\cite{electron}.
Rutherford's discovery of the proton~\cite{Rutherford:1911zz} and Chadwick's discovery of the neutron~\cite{Chadwick:1932ma} all but completed the picture, as between them these three particles constitute essentially all stable matter in the universe.  

In 1934 Yukawa proposed that the neutrons and protons were bound together using an interaction appropriately called the strong force~\cite{193548}. This required a new particle to be hypothesised: the \emph{meson} (meaning intermediate). The mass of this particle would have to sit inbetween the light electron and heavier  protons and neutrons. The electron correspondingly became part of a class of particles call \emph{leptons} (meaning light) and the protons and neutrons part of \emph{baryons} (meaning heavy). 
The search for the strong force mediator resulted in the joint discoveries of the pion and muon~\cite{Lattes:1947mw,Lattes:1947mx}, both initially thought to be mesons. 


Throughout the 20th century further pieces of the puzzle were uncovered, including anti-electrons, neutrinos and a wealth of mesons and baryons. 
%Over the next {\color{Red}something} years further pieces of the puzzle were uncovered, including anti-electrons, neutrinos and a wealth of mesons and baryons. 
The seemingly bizarre complexity of particles was eventually explained by Gell-Mann~\cite{GellMann:1964nj} and Zweig~\cite{Zweig:1964jf} through the quark model. This proposed that mesons and baryons were in fact composite particles, made up of elementary constituents called \emph{quarks}. These were considered point-like and could be combined rather simply: a quark and anti-quark combined to make a meson ($q\bar{q}$), and three quarks or three anti-quarks combined to form baryons ($qqq$,$\bar{q}\bar{q}\bar{q}$)\footnote{More exotic combinations ($q\bar{q}q\bar{q}$, $qqqq\bar{q}$...) were also proposed and recently observed~\cite{PhysRevLett.91.262001,PhysRevLett.115.072001}.}.     

The theoretical framework for the Standard Model of Particle Physics began with the unification of the electromagnetic and weak forces in the 1960s. The development of the strong interaction in the 1970s completed the theory in the form that it is understood today. During next 40 years the remaining particles predicted to be a part of this picture were rapidly discovered: the charm quark in 1974~\cite{PhysRevLett.33.1406,PhysRevLett.33.1404}; the $\tau$ lepton in 1975~\cite{PhysRevLett.35.1489}; the \bquark-quark in 1977~\cite{PhysRevLett.39.252}; the \W and \Z bosons in 1983~\cite{Arnison:1983rp,Banner:1983jy}; the \tquark-quark in 1995~\cite{PhysRevLett.74.2626,PhysRevLett.74.2422} and finally the Higgs Boson in 2012~\cite{Aad:2012tfa,Chatrchyan:2012xdj}.      
A brief introduction to the Standard Model is detailed in this chapter: the successes and limitations are highlighted, along with potential theories that may supersede this model.
A more detailed description of the relevant theoretical aspects that underpin the rare hadronic \B meson decays are described in Chapter~\ref{ch:theory}. 

The experimental apparatus used to collect the data analysed in this thesis, the \lhcb experiment, is described in Chapter~\ref{ch:detector}, followed by the methodology used to process this data set in Chapter~\ref{ch:selection}. Finally, the statistical methods used to extract measurements can be found in Chapters~\ref{ch:B2DsKK} and~\ref{ch:B2DsPhi}.

The research presented in this thesis has been published in Ref.~\cite{LHCb-PAPER-2017-032}. 



\section{The Standard Model of Particle Physics}

The Standard Model (SM) provides explanation for a large number of natural phenomena to a high precision. There are currently a number of unresolved issues that are not explained, implying that this theory is not the complete description of the natural world. The theory was formulated in the 1970s, with the final particles being predicted and discovered by physicists around the world in the decades following.

\subsection{Building blocks}

The SM is composed of a small number of fundamental particles; some are the components of matter and others are force-carriers that mediate interactions. The fundamental particles can be combined into composite combinations, bound by these fundamental interactions to form a rich tapestry of observable particles.

Mathematically, the theory is described by the unitary product group $\text{SU}(3)_{C}\times\text{SU}(2)_{L}\times\text{U}(1)_{Y}$, as the components are represented as gauge field theories. Here, $C$, $L$, and $Y$ refer to the fundamental charges: colour, weak isospin and hypercharge; the generators of each symmetry group. Throughout this thesis decays will be represented by Feynman Diagrams, pictorial representations of mathematical expressions~\cite{PhysRev.76.749}.  
The standard model is encapsulated by a Lagrangian density
\begin{equation}
\mathcal{L} = \mathcal{L}_{\text{QCD}}+\mathcal{L}_{\text{QED}}+\mathcal{L}_{\text{Weak}}+\mathcal{L}_{\text{Higgs}},
\end{equation}
where the total quantity receives contributions from the fundamental forces as well as the spontaneous symmetry breaking mechanism allowing the particles to obtain masses. 
The particles of the SM can be separated into fundamental fermions (particles with half-integer spin) and fundamental bosons (particles with integer spin). These are described in the following sections.  


\subsubsection{Fundamental fermions}

The fundamental spin-$1/2$ fermions that constitute matter can be divided into two categories: leptons and quarks. 
The fermions can be split into \emph{families} or \emph{generations} containing two leptons and two quarks. Each of these contains an up-type and down-type quark, a charged lepton and a neutrino. There are three of these \emph{generations} that are identical clones of one another, differing only by their rest masses; each subsequent generation is heavier than the previous. The interactions of these three generations via the fundamental forces are identical in the SM. 
These particles are listed in Table~\ref{tab:intro_particles} along with their electromagnetic charge.
\begin{table}[h]
   \begin{center}
      \begin{tabular}{lcr | lcr}
         \hline
         \multicolumn{3}{c|}{Quarks} & \multicolumn{3}{c}{Leptons}\\
         \hline
         Name       & Symbol            & Q  & Name                & Symbol            & $Q$    \\ 
         \hline
         Up         & \uquark           &  $+2/3$ & Electron neutrino   & \neue             &  $0$   \\ 
         Down       & \dquark           &  $-1/3$ & Electron            & \en               &  $-1$  \\ 
         \hline
         Charm      & \cquark           &  $+2/3$ & Muon neutrino       & \neum             &  $0$   \\ 
         Strange    & \squark           &  $-1/3$ & Muon                & \mun              &  $-1$  \\ 
         \hline
         Top        & \tquark           &  $+2/3$ & Tau neutrino        & \neut             &  $0$   \\ 
         Bottom     & \bquark           &  $-1/3$ & Tau                 & \taum             &  $-1$  \\ 
         \hline
      \end{tabular}
   \end{center}
   \caption{The fundamental fermions in the standard model. The electromagnetic charge, $Q$, is given in units of the absolute value of the electron charge: $1.6\times 10^{-19}$\,C.}
   \label{tab:intro_particles}
\end{table}
The first generation, containing the \uquark, \dquark, \en and \neue particles, is the least massive generation and most stable. The second and third generations exist only fleetingly, before decaying back down to the first generation. Therefore, all stable matter in the universe is comprised of first-generation particles. 
As detailed in Table~\ref{tab:intro_particles}, all fermions except neutrinos have a electromagnetic charge. 
Additionally, they have other quantum numbers that define their interaction with the other fundamental interactions. 

% {\color{Red}
% \begin{itemize}
% \item Quantum numbers?
% \item Degrees of freedom: colour, flavour
% \end{itemize}}


\subsubsection{Antimatter}

For each of the fundamental fermions in the standard model there exists a antimatter partner, as predicted by P. Dirac in 1928~\cite{Dirac610}. The positive electron partner, called positron, was first discovered by C. Anderson in 1932~\cite{PhysRev.43.491}.
Antimatter particles only differ from their matter counterparts by their quantum numbers, such as charge or baryon and lepton number; the mass and lifetimes of a particles and its antiparticle are identical. Matter and antimatter can recombine or \emph{annihilate}, releasing the energy stored in their rest masses. 
Therefore, each of the fermions in Table~\ref{ch:introduction} has a corresponding antimatter equivalent with the opposite electromagnetic charge.  

\subsubsection{Fundamental interactions}

The SM provides a mechanism for three fundamental forces: the electromagnetic, weak and strong forces. Each of these are mediated by force-carriers called gauge bosons.
Details of these forces are listed in Table~\ref{tab:intro_forces} along with gravity, the forth fundamental force. 

\begin{table}[h]
   \begin{center}
      \begin{tabular}{ccccc}
         \hline

         Force                  & Boson             & Symbol    & Mass (\gevcc)     & Range (\m)                    \\
         \hline 
         Strong                 & gluon             & $g$       & 0                 & $10^{-15}$                    \\
         Electromagnetic        & photon            & \Pgamma   & 0                 & $\infty$                      \\
         \multirow{ 2}{*}{Weak} & $W$ boson         & \Wpm      & $80.385\pm0.015$  & \multirow{ 2}{*}{$10^{-18}$}  \\
                                & $Z$ boson         & \Z        & $90.188\pm0.002$  &                               \\
         \hline
         Gravity                & \emph{graviton?}  &           & 0                 & $\infty$                      \\

         \hline
      \end{tabular}
   \end{center}
   \caption{The fundamental interactions listed in order of strength.}
   \label{tab:intro_forces}
\end{table}

The electromagnetic interaction is mediated by the massless photon. This affects all particles with a electromagnetic charge. The range of this force is infinite. The interaction is described fully by Quantum Electrodynamics (QED), an Abelian gauge field theory in which the photon doesn't undergo self-interactions as it is uncharged\footnote{Light-by-light scattering has been observed but this is mediated by quark loops~\cite{Aaboud:2017bwk}}.

The strong force is mediated by the massless gluon. This couples to colour charge, which comes in three variants; red, green and blue.  
The strong interaction is described by Quantum Chromodynamics (QCD), a non-Abelian gauge theory. This allows gluon self-coupling as the gluons carry colour charge themselves. This leads to complicated dynamics that limits the range of the strong force. Of the fundamental fermions, only the quarks have colour charge. Leptons don't interact strongly. 



The weak force is mediated by the massive \Wpm and \Z bosons. The range of this force is limited to the order of $1/m_{\W,\,\Z}$ and appears much weaker than the electromagnetic force for interactions with energies much below $\sim m_{\W,\,\Z}$. The weak interaction is also described a non-Abelian gauge theory in which the gauge bosons can self-interact. 

% {\color{Red}
% \begin{itemize}
% \item EM - Abeilan: no self coupling (except light by light scattering)
% \item Weak, strong: non-abeilan: self coupling
% \item Range: EM- massles photon -> infinite 
% \item Range: Weak - massive bosons -> short range
% \item Range: Strong - massless but self interaction/confinement -> short range (force is infinite )
% \end{itemize}}


\subsubsection{The origin of mass}
The final particle of the SM is the Higgs boson, a key component to allow the fundamental particles to gain mass. Discovered in 2012 at the Large Hadron Collider~\cite{Aad:2012tfa,Chatrchyan:2012xdj} it constitutes the final particle predicted to be part of the SM. 

In the standard model all fermions would naively be expected to be massless, as the Lagrangian density terms that would allow them to gain a mass are excluded due to symmetry constraints. The exact symmetry can be broken by a mechanism proposed by R. Brout, F. Englert and P. Higgs~\cite{PhysRevLett.13.508,PhysRevLett.13.321} in which a new scalar field is introduced called the Higgs field. 

The masses of particles arise as a result of the self coupling of the Higgs field. The field potential contains quadratic and quartic terms in the field strength, allowing a situation where stable minima exist, displaced from the origin (which is an unstable maximum). The perturbations of this field occur around any one of the degenerate minima, requiring the symmetry of the system to be spontaneously broken. This leads the vacuum expectation value, $v$, of the field to be non-zero.   

% Gauge boson masses are a result of the kinetic energy of the Higgs field. The dynamics of the field result in Lagrangian density mass terms for the \Wpm and \Z bosons,
% \begin{equation}
% m_{\Wpm} = \frac{g_{\Wpm}v}{2}, ~~m_{\Z} = \frac{g_{\Z}v}{2}, 
% \end{equation}
% where $g_{\Wpm}$ and $g_{\Z}$ are the weak coupling strengths of \Wpm and \Z boson respectively.
% %, related via $g_{\Wpm} = g_{\Z}\cos{\theta_{W}}$.


% The fermion masses arise from the Higgs-fermion field interactions directly. Again the non-zero vacuum expectation value results in a Lagrangian density mass term for the fermions 
% \begin{equation}
% m_{f} = \frac{\lambda_{f}v}{\sqrt{2}}
% \end{equation}
% where $\lambda_{f}$ is a coupling that determines the strength of the interaction between the higgs field and fermion. This dictates the mass of the fermion.

% The Higgs boson gains mass itself from the Higgs field kinematics and potential. When displaced by the non-zero vacuum expectation value the Lagrangian density obtains a higgs mass term
% \begin{equation}
% m_{H} = \sqrt{2\lambda}v,
% \end{equation}
% where $\lambda$ is one the terms that determines the shape of the Higgs potential. 

Gauge boson masses are a result of the kinetic energy of the Higgs field. The fermion masses arise from the Higgs-fermion field interactions directly. The Higgs boson itself gains mass from the Higgs field kinematics and potential.


The masses of the SM fermions as determined using experiment input are shown in Table~\ref{tab:intro_particles_masses}. 
\begin{table}[h]
   \begin{center}
      \begin{tabular}{cc}
         \hline
         Particle          & Mass (\mevcc)                    \\ 
         \hline
         \uquark           & $2.2^{+0.6}_{-0.4}$   \\ 
         \dquark           & $4.7^{+0.5}_{-0.4}$   \\ 
         \en            & $0.5109989461\pm0.0000000031$ \\ 
         \hline
         \cquark           & $1280\pm30$         \\ 
         \squark           & $96^{+8}_{-4}$        \\
         \mun           & $105.6583745\pm0.0000024$      \\
         \hline
         \tquark           & $173000\pm600$           \\ 
         \bquark           & $4180^{+40}_{-30}$\\
         \taum             & $1776.86\pm0.12 $         \\       
         \hline                                
         % \neue          &       \\                                
         % \neum          &      \\                                 
         % \neut          &      \\                                 
         % \hline
      \end{tabular}
   \end{center}
   \caption{The masses of the charged fundamental fermions, from Ref.~\cite{PDG2016}.}
   \label{tab:intro_particles_masses}
\end{table}
In the standard model neutrinos are massless particles, however they have been observed to oscillate between flavours, requiring that at least two are massive. 
It is clear that the charged lepton masses can be determined much more precisely than the quarks.  
The quark masses can't be measured directly, instead they are determined indirectly using input from hadronic measurements.

% {\color{Red}
% \begin{itemize}
% \item maybe say how they can be measured?
% \end{itemize}}
% Measuring quark masses is different for light and heavy quarks (defined by the mass scale of the strong interaction).
% Light: chiral perturbation theory
% Heavy: heavy quark effective theory

Whilst very significant, the fermion masses themselves are not responsible for the majority of mass in the universe. The proton is a composite particle of two \uquark-quarks and one \dquark-quark with a mass of $m_{\proton}=938\mevcc$. This is about 100 times larger than the sum of the constituent quark masses. The remainder of the mass originates from the  binding dynamics of the strong interaction. The proton can be thought of as containing a sea of quarks and gluons that mediate the attraction between the valence quarks. The invariant mass of the whole system gives rise to the proton mass. 
 

\subsubsection{\CP violation}

The combined symmetries of charge conjugation (swapping particles for antiparticles) and parity inversion (mirroring space) are referred to as \CP-symmetry. The SM contains mechanisms that allow this \CP-symmetry to be violated in decays: processes can happen at different rates for matter and antimatter.  
The weak interaction introduces all of the \CP violation observed in the SM. Both the quark and lepton sectors have the freedom to mix between different generations, introducing a complex weak phase that can cause asymmetries in interference between competing processes. In principle processes mediated by the strong interaction could exhibit \CP-violation, however this has not been observed to occur. 

\subsection{Parameters}

The SM contains 26 free parameters. The theory doesn't predict the value of these numbers, instead they must be worked out experimentally.  
\begin{description}
\item \textbf{Masses:} the masses of the 12 fundamental fermions (alternatively parametrised as the Higgs-fermion couplings).
\item \textbf{Couplings:} the three coupling strengths of the electromagnetic, strong and weak forces.
\item \textbf{Higgs potential:} two parameters determine the Higgs potential: the vacuum expectation value and the Higgs Boson mass.
\item \textbf{Mixing parameters:} these eight parameters allow the weak interaction eigenstates and mass eigenstates to differ in both the quark and lepton sector.
\item \textbf{Strong \CP  phase:} this could allow \CP violation in the strong interaction, but is experimentally measured to be extremely small.
\end{description}

The relatively large number of free parameters in the SM may hint to this model only being a subset of a more complete theory. Trends are observed between the quantities, for example the masses of the fermions in each generation are each fairly similar in size. Some larger theory may be responsible for these trends.  

\subsection{Triumphs}
The SM predicts a number of processes to an extraordinary accuracy. 
\subsubsection{Magnetic moments}
The intrinsic magnetic moment for electrons was predicted by Dirac to be $g=2$. In QED this quantity receives corrections as a result of higher order effects that make the value slightly larger than two. The electrons anomalous coupling $a_{e} = (g-2)/2$ has been measured~\cite{PhysRevLett.100.120801} and predicted theoretically~\cite{PhysRevD.96.019901} to be 
\begin{equation}
\begin{split}
a_e(\text{theory}) & = (1159652182.031\pm0.015\pm0.015\pm0.720)\times10^{-12}\\
a_e(\text{exp.})   & = (1159652180.73\pm0.28)\times10^{-12}.
\end{split}
\end{equation}
The remarkable level of agreement demonstrates the predictive accuracy of the SM. 



\subsubsection{Precision tests of electroweak theory}

The SM makes predictions for the properties of the electroweak gauge bosons. Tests were performed on the \Z boson measuring the total decay width, a quantity sensitive to the number of light neutrino species. Experiments conducted at the Large Electron Positron collider (\lep) and SLAC Linear Collider determined the number of species to be $2.9840\pm0.0082$~\cite{ALEPH:2005ab}, consistent with the SM picture.
Additionally, these experiments used vast collections of \Z bosons to measure branching fractions, polarisations and asymmetries to high precision.  

\subsection{Shortfalls}
In spite of the success of SM predictions a number of areas are not explained by the theory.  

\subsubsection{Baryon asymmetry} 

The universe is observed to be matter dominated. Macroscopic quantities of antimatter elsewhere in the universe would result in matter-antimatter boundaries that radiate energy from annihilation. This would lead to intense sources of high energy radiation visible to gamma ray telescopes; a phenomenon that has not been observed~\cite{vonBallmoos2014}. 
%The matter dominance of the local solar system can be demonstrated rather simply: the probes sent from Earth to planets as far away as Jupiter have not annihilated on impact, so therefore must be made of matter. 
Three conditions that can lead to asymmetries between matter and anitmatter were postulated by A.D. Sakharov in 1966~\cite{Sakharov:1967dj}. These require baryon number violation, $C$- and \CP-symmetry violation and interactions out of thermal equilibrium. The source of \CP-violation in the SM is far too small to account for the observed cosmic asymmetry. For processes at the electroweak energy scale, the quark masses (excluded the \tquark-quark) are comparatively small, leading to a tiny level of \CP violation. Sources beyond the scope of the SM are therefore expected to contribute to this asymmetry. 

% \subsubsection{Lepton flavour anomalies}
% Recent measurements at the \lhc and \B-factories hint towards anomalous decay rates in decays to different generations of leptons~\cite{LHCb-PAPER-2017-013,LHCb-PAPER-2014-024}. In the standard model the couplings to the lepton flavours are equal, the only differences in decays rates arise as a result of the different masses.

% {\color{Red}
% \begin{itemize}
% \item add reference 
% \end{itemize}
% }

\subsubsection{Neutrino masses} 
In the SM, only left handed neutrinos exist.  The SM fermion masses can be thought of as couplings between the left- and right-handed fields. Without a right-handed counterpart, neutrinos cannot have mass. This has been demonstrated to be incorrect by the observation of neutrino flavour oscillations that require non-zero masses~\cite{PhysRevLett.81.1158,PhysRevLett.87.071301}. 
Models beyond the SM offer some possible explanations. As neutrinos are not electromagnetically charged they could be their own antiparticles. These types of fermions are known as Majorana fermions~\cite{Majorana:1937vz} and could lead to such phenomena as neutrino-less double beta decay. 
Another possibility is that neutrinos are Dirac fermions, with a right-handed singlet counterparts. However, these would not interact with any of the SM forces.


\subsubsection{Gravity}
The SM doesn't incorporate a gauge field theory of gravity. The interaction can be hypothesised to be quantised by a massless spin-2 graviton. However, a complete integration of general relativity and the standard model would require a theory of quantum gravity. Gravity is of comparative strength to the SM forces around the Planck energy scale, $\mathcal{O}(10^{19})\gev$.    

\subsubsection{Dark matter and dark energy}
Astrophysical evidence suggests that baryonic matter only makes up small fraction of the total matter content of the universe. 
The distribution of galactic rotation curves suggests a large halo of matter is present in the galaxy that only interacts gravitationally~\cite{DMrotation}. Additionally phenomena such as the bullet cluster and gravitational lensing point towards a higher density of matter than can be observed with electromagnetic radiation~\cite{0004-637X-604-2-596,0004-637X-606-2-819}. 

The acceleration of universe's expansion is attributed to dark energy; a currently poorly understood type of energy distinct from dark matter and baryonic matter.

\subsubsection{Strong \CP violation (or lack thereof)}  
The strong interaction \CP violating phase is not constrained by the theory but experimentally it is found to be to be extremely small from measurements of the neutrons electric dipole moment. The current limits constrain the strong \CP violating phase to be below $\mathcal{O}(10^{-10})$~\cite{Kuckei2007} 

\subsubsection{Hierarchy problem}
The typical energies of different interactions in the SM are on vastly difference scales, for example the electroweak interaction scale is $\mathcal{O}(10^{3})\gev$, whilst the GUT scale (the energy at which the electroweak and strong force could be unified) is of order $\mathcal{O}(10^{16})\gev$. Additionally, gravity is expected to be comparable at the Planck scale $\mathcal{O}(10^{19})\gev$. These different mass scales make it difficult to understand why the Higgs mass is so much less than these energies; quantum corrections would be expected to raise the mass to these scales. Additional theories are required to cancel these contributions. 



\section{Beyond the Standard Model}
\label{sec:intro_BSM}
As the SM is deficient in it's description of the natural world, a large number of theories proposing extensions have been hypothesised.
A small selection of Beyond the Standard Model (BSM) theories are summarised here. 

\subsection{Supersymmetry}

The Supersummetry (SUSY) theory unifies fermions and bosons by proposing a symmetry that can interchange the two. This requires creating an additional set of particles for each SM fermion and boson; the supersymmeteric partners. This symmetry is broken to allow the SM and SUSY partners to have different masses. This theory predicts dark matter candidates, fixes the hierarchy problem and unifies the gauge interactions. 
However, the broken symmetry results in a large number of free parameters, leading to a huge phase-space in which to search. So far no evidence of SUSY has been observed.

\subsubsection{$R$-parity violating SUSY}

Within SUSY, the original SM particles and the newly introduced SUSY particles can be distinguished by a quantum number called $R$-parity, for which SM (SUSY) particles have eigenvalues $+1$ ($-1$). In SUSY models that conserved $R$-parity, the lightest SUSY particle must be stable, as it would not be able to decay to just SM particles. This results in a convenient dark matter candidate, and prevents rapid proton decay that could be observed if baryon and lepton number conservation can be violated. 

Some models exist in which $R$-parity violation (RPV) occurs. If only one of baryon or lepton number is also violated then the rapid decay of the proton can be avoided. 

\subsubsection{Two Higgs doublets}
A number of extensions to the SM exist which include additional Higgs doublets. The SM contains a single Higgs doublet, corresponding to the scalar Higgs boson and the three Goldstone bosons~\cite{PhysRev.117.648,Goldstone1961} that allow the \Wpm and \Z gauge bosons to gain a longitudinal degree of freedom and hence mass. Introducing a second Higgs doublet results in four more physical states; this gives a total of three neutral Higgs bosons and two charged. 
SUSY requires two Higgs doublets with up-type and down-type fermions coupling to different doublets.   
% http://pdglive.lbl.gov/Particle.action?node=S055&init=0
% http://pdglive.lbl.gov/Particle.action?node=S064&init=0

\subsection{Leptoquarks}

A number of theories result in the creation of \emph{leptoquarks}: additional particles that facilitate quarks and leptons to be interchanged. In these theories the SM strong $\text{SU}(3)$ and electroweak $\text{SU}(2)\times\text{U}(1)$ symmetries are embedded in a larger symmetry, for example $\text{SU}(5)$. The gauge bosons of this symmetry can couple to leptons and quarks. Leptoquarks can allow lepton flavour universality to be violated~\cite{PhysRevD.97.015019}. 

\subsection{Extra dimensions}
Attempts by T. Kaluza and O. Klein to unify the electromagnetic interaction and gravity by extending space-time into a fifth dimension~\cite{Kaluza:1921tu,Klein1926}. Many different models exist that propose the geometry of this additional dimension. These provide solutions to the hierarchy problem and propose additional particles that could be observed. 
%http://pdg.lbl.gov/2017/reviews/rpp2017-rev-extra-dimensions.pdf

\subsection{Axions}
The problems surrounding the lack of strong \CP violation can be circumvented by introducing a new field to the SM as proposed by R. Peccei and H. Quinn~\cite{PhysRevLett.38.1440}. This results in a new light or massless gauge boson called the Axion, however this has not been observed.     
%http://pdg.lbl.gov/2017/reviews/rpp2017-rev-axions.pdf
