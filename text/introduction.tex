\chapter{Introduction} 
\label{ch:introduction}

\minitoc 



%\section{Overview} 


Throughout history, society has endeavoured to glimpse into the heart of matter, trying to break it down into it's elementary building blocks.
The term \emph{atom} was originally used by the ancient Greek philosophers to speculate on the nature of matter and used to denote something that was indivisible. 
However, modern particle physics took until the 19th century to get going; the first of today's elementary particles, the electron, was discovered by J.J. Thomson in 1897~\cite{electron}.

Rutherford's discovery of the proton~\cite{Rutherford:1911zz} and Chadwick's discovery of the neutron~\cite{Chadwick:1932ma} all but completed the picture, as between them these three particles constitute essentially all stable matter in the universe.  
In 1934 Yukawa proposed that the neutrons and protons were bound together using an interaction appropriately called the strong force~\cite{193548}. This required a new particle to be hypothesised: the \emph{meson} (meaning intermediate). The mass of this particle would have to sit inbetween the light electron and heavier  protons and neutrons. The electron correspondingly became part of a class of particles call \emph{leptons} (meaning light) and the protons and neutrons part of \emph{baryons} (meaning heavy). 
The search for the strong force mediator resulted in the joint discoveries of the pion and muon~\cite{Lattes:1947mw,Lattes:1947mx}, both initially thought to be mesons. 


Over the next {\color{Red}something} years further pieces of the puzzle were uncovered, including anti-electrons, neutrinos and a wealth of mesons and baryons. 
The seemingly bizarre complexity of particles was eventually explained by Gell-Mann~\cite{GellMann:1964nj} and Zweig~\cite{Zweig:1964jf} through the quark model. This proposed that mesons and baryons were in fact composite particles, made up of elementary constituents called \emph{quarks}. These were considered point-like could be combined rather simply: a quark and anti-quark combined to make a meson ($q\bar{q}$), and three quarks or three anti-quarks combined to form baryons ($qqq$,$\bar{q}\bar{q}\bar{q}$)\footnote{More exotic combinations ($q\bar{q}q\bar{q}$, $qqqq\bar{q}$...) were also proposed and recently observed~\cite{PhysRevLett.91.262001,PhysRevLett.115.072001}.}.     


{\color{Red}
\begin{itemize}
%\item atom used to mean a particle that can't be made into smaller pieces 
%\item Historical context 
\item formulation of the SM
\end{itemize}}





A brief introduction to the Standard Model of Particle Physics is detailed later in this chapter. A more detailed description of the relevant theoretical aspects that underpin the rare hadronic \B meson decays is described in Chapter~\ref{ch:theory}. A description of experimental apparatus used to collect data is included in Chapter~\ref{ch:detector}, followed by the methodology used to process this data in Chapter~\ref{ch:selection}. The statistical methods used to extract measurements of the decays can be found in Chapters~\ref{ch:B2DsKK} and~\ref{ch:B2DsPhi}.


The research presented in this thesis has been published in Ref.~\cite{LHCb-PAPER-2017-032}. 



\section{The Standard Model of Particle Physics}

The Standard Model (SM) provides explanation for a large number of natural phenomena to a high precision. There are currently a number of unresolved issues that are not explained, implying that this theory is not the complete description of the natural world. The theory was formulated in the 1970s, with particles being predicted and discovered by physicists around the world in the decades following.

\subsection{Building blocks}

The SM is composed of a small number of fundamental particles; some are the components of matter and others are force-carriers that mediate interactions. The fundamental particles can be combined into composite combinations, bound by these fundamental interactions to form a rich tapestry of observable particles.

Mathematically, the theory is described by the unitary product group $\text{SU}(3)_{C}\times\text{SU}(2)_{L}\times\text{U}(1)_{Y}$, as the components are represented as gauge field theories. Throughout this thesis decays will be represented by Feynman Diagrams, pictorial representations of the mathematical amplitudes for various processes~\cite{PhysRev.76.749}.  
The standard model is encapsulated by a Lagrangian density
\begin{equation}
\mathcal{L} = \mathcal{L}_{\text{QCD}}+\mathcal{L}_{\text{QED}}+\mathcal{L}_{\text{Weak}}+\mathcal{L}_{\text{Higgs}},
\end{equation}
where the total quantity receives contributions from the fundamental forces as well as the spontaneous symmetry breaking mechanism allowing the particles to obtain masses. 
The particles of the SM can be separated into fundamental fermions (particles with half-integer spin) and fundamental bosons (particles with integer spin). These are described in the following sections.  


\subsubsection{Fundamental fermions}

The fundamental spin-$1/2$ fermions that constitute matter can be divided into two categories: leptons and quarks. 
These particles can be divided up into \emph{families} or \emph{generations} containing two leptons and two quarks. Each of these contains an up-type and down-type quark, a charged lepton and a neutrino. There are three of these \emph{generations} that are identical clones of one another, differing only by their rest masses; each subsequent generation is heavier than the previous. The interactions of these three generations via the fundamental forces are identical in the SM. 
These particles are listed in Table~\ref{tab:intro_particles} along with their electromagnetic charge.
\begin{table}[h]
   \begin{center}
      \begin{tabular}{lcr | lcr}
         \hline
         \multicolumn{3}{c|}{Quarks} & \multicolumn{3}{c}{Leptons}\\
         \hline
         Name       & Symbol            & Q  & Name                & Symbol            & $Q$    \\ 
         \hline
         Up         & \uquark           &  $+2/3$ & Electron neutrino   & \neue             &  $0$   \\ 
         Down       & \dquark           &  $-1/3$ & Electron            & \en               &  $-1$  \\ 
         \hline
         Charm      & \cquark           &  $+2/3$ & Muon neutrino       & \neum             &  $0$   \\ 
         Strange    & \squark           &  $-1/3$ & Muon                & \mun              &  $-1$  \\ 
         \hline
         Top        & \tquark           &  $+2/3$ & Tau neutrino        & \neut             &  $0$   \\ 
         Bottom     & \bquark           &  $-1/3$ & Tau                 & \taum             &  $-1$  \\ 
         \hline
      \end{tabular}
   \end{center}
   \caption{The fundamental fermions in the standard model. The electromagnetic charge, $Q$, is given in units of the absolute value of the electron charge: $1.6\times 10^{-19}$\,C.}
   \label{tab:intro_particles}
\end{table}
The first generation, containing the \uquark, \dquark, \en and \neue particles, is the least massive generation and most stable. The second and third generations exist only fleetingly, before decaying back down to the first generation. Therefore, all stable matter in the universe is comprised of first-generation particles. 

As detailed in Table~\ref{tab:intro_particles}, all fermions except neutrinos have a electromagnetic charge. 

%Additionally the 

{\color{Red}
\begin{itemize}
\item Quantum numbers?
\item Degrees of freedom: colour, flavour
\end{itemize}}


\subsubsection{Antimatter}

For each of the fundamental fermions in the standard model there exists a anti-matter partner, as predicted by P. Dirac in 1928~\cite{Dirac610}. The positive electron partner, called positron, was first discovered by C. Anderson in 1932~\cite{PhysRev.43.491}.
Antimatter particles only differ from their matter counterparts by their quantum numbers, such as charge or baryon and lepton number. The mass of a particles antimatter partner is identical. Matter and antimatter can recombine or \emph{annihilate}, releasing the energy stored in their rest masses. 
Therefore, each of the fermions in Table~\ref{ch:introduction} has a corresponding antimatter equivalent with the opposite electromagnetic charge.  

\subsubsection{Fundamental interactions}

The SM provides a mechanism for three fundamental forces: the electromagnetic, weak and strong forces. Each of these are mediated by force-carriers called gauge bosons.
Details of these forces are listed in Table~\ref{tab:intro_forces} along with gravity, the forth fundamental force. 

\begin{table}[h]
   \begin{center}
      \begin{tabular}{ccccc}
         \hline

         Force                  & Boson             & Symbol    & Mass (\gevcc)     & Range (\m)                    \\
         \hline 
         Strong                 & gluon             & $g$       & 0                 & $10^{-15}$                    \\
         Electromagnetic        & photon            & \Pgamma   & 0                 & $\infty$                      \\
         \multirow{ 2}{*}{Weak} & $W$ boson         & \Wpm      & $80.385\pm0.015$  & \multirow{ 2}{*}{$10^{-18}$}  \\
                                & $Z$ boson         & \Z        & $90.188\pm0.002$  &                               \\
         \hline
         Gravity                & \emph{graviton?}  &           &                   & $\infty$                      \\

         \hline
      \end{tabular}
   \end{center}
   \caption{The fundamental interactions in the standard model listed in order of strength.}
   \label{tab:intro_forces}
\end{table}

The electromagnetic interaction is mediated by the massless photon. This affects all particles with a electromagnetic charge, including all fundamental fermions except neutrinos. As the mediator is massless the range of this force is infinite. The interaction is described fully by Quantum Electrodynamics (QED), an Abelian gauge field theory in which the photon doesn't interaction with itself at the lowest order\footnote{Light-by-light scattering has been observed when mediated by quark loops~\cite{Aaboud:2017bwk}}.

The strong force is mediated by the massless gluon. This couples to colour charge, which comes in three variants; red, green and blue.  
The strong interaction is described by Quantum Chromodynamics (QCD), a non-Abelian gauge theory. This allows gluon self-coupling as the gluons carry colour charge themselves. This leads to complicated dynamics that limits the range of the strong force. Of the fundamental fermions, only the quarks have colour charge. Leptons don't interact strongly. 



The weak force is mediated by the massive \Wpm and \Z bosons. The range of this force is limited to the order of $1/m_{\Wpm,\,\Z}$ and appears much weaker than the electromagnetic force for interactions with energies much below $\sim m_{\Wpm,\,\Z}$. The weak interaction is also described a non-Abelian gauge theory in which the gauge bosons can self-interact. 

% {\color{Red}
% \begin{itemize}
% \item EM - Abeilan: no self coupling (except light by light scattering)
% \item Weak, strong: non-abeilan: self coupling
% \item Range: EM- massles photon -> infinite 
% \item Range: Weak - massive bosons -> short range
% \item Range: Strong - massless but self interaction/confinement -> short range (force is infinite )
% \end{itemize}}


\subsubsection{The origin of mass}
The final particle of the SM is the Higgs boson, a key component to allow the fundamental particles to gain mass. Discovered in 2012 at the Large Hadron Collider~\cite{20121,201230} it constitutes the final particle predicted to be part of the SM. 

In the standard model all fermions would naively be expected to be massless, as the Lagrangian density terms that would allow them to gain a mass are excluded due to symmetry constraints. The exact symmetry can be broken by a mechanism proposed by R. Brout, F. Englert and P. Higgs~\cite{PhysRevLett.13.508,PhysRevLett.13.321} in which a new scalar field is introduced called the Higgs field. 

The masses of particles arise as a result of the self coupling of the Higgs field. The field potential contains quadratic and quartic terms in the field strength, allowing a situation where stable minima exist, displaced from the origin (which is an unstable maximum). The perturbations of this field occur around any one of the degenerate minima, requiring the symmetry of the system to be spontaneously broken. This leads the vacuum expectation value, $v$, of the field to be non-zero.   

Gauge boson masses are a result of the kinetic energy of the Higgs field. The dynamics of the field result in Lagrangian density mass terms for the \Wpm and \Z bosons,
\begin{equation}
m_{\Wpm} = \frac{g_{\Wpm}v}{2}, ~~m_{\Z} = \frac{g_{\Z}v}{2},
\end{equation}
where $g_{\Wpm}$ and $g_{\Z}$ are the coupling strengths of \Wpm and \Z boson respectively.


The fermion masses arise from the Higgs-fermion field interactions directly. Again the non-zero vacuum expectation value results in a Lagrangian density mass term for the fermions 
\begin{equation}
m_{f} = \frac{\lambda_{f}v}{\sqrt{2}}
\end{equation}
where $\lambda_{f}$ is a coupling that determines the strength of the interaction between the higgs field and fermion. This dictates the mass of the fermion.

The Higgs boson gains mass itself from the Higgs field kinematics potential. When displaced by the non-zero vacuum expectation value the Lagrangian density obtains a higgs mass term
\begin{equation}
m_{H} = \sqrt{2\lambda}v,
\end{equation}
where $\lambda$ is one the terms that determines the shape of the Higgs potential. 
The masses of the fermions in the SM are shown in Table~\ref{tab:intro_particles_masses}. 
\begin{table}[h]
   \begin{center}
      \begin{tabular}{cc}
         \hline
         Particle          & Mass                        \\ 
         \hline
         \uquark           & $2.2^{+0.6}_{-0.4}\mevcc$   \\ 
         \dquark           & $4.7^{+0.5}_{-0.4}\mevcc$   \\ 
         \en            & $0.5109989461\pm0.0000000031\mevcc$ \\ 
         \hline
         \cquark           & $1280\pm30\mevcc$         \\ 
         \squark           & $96^{+8}_{-4}\mevcc$        \\
         \mun           & $105.6583745\pm0.0000024\mevcc$      \\
         \hline
         \tquark           & $173\pm0.6\gevcc$           \\ 
         \bquark           & $4.18^{+0.04}_{-0.03}\gevcc$\\
         \taum             & $1.77686\pm0.00012 \gevcc$         \\       
         \hline                                
         \neue          &       \\                                
         \neum          &      \\                                 
         \neut          &      \\                                 
         \hline
      \end{tabular}
   \end{center}
   \caption{The masses of the fundamental fermions. }
   \label{tab:intro_particles_masses}
\end{table}

{\color{Red}
\begin{itemize}
\item maybe say how they can be measured?
\item The majority of mass is actually from binding energy 
\item how gauge bosons get mass
\end{itemize}}


\subsection{Parameters}

The SM contains 26 free parameters. The theory doesn't predict the value of these numbers, instead they must be worked out experimentally.  
\begin{description}
\item \textbf{Masses:} the masses of the 12 fundamental fermions.
\item \textbf{Couplings:} the three coupling strengths of the electromagnetic, strong and weak forces.
\item \textbf{Higgs potential:} two parameters determine the Higgs potential: the vacuum expectation value and the Higgs Boson mass.
\item \textbf{Mixing parameters:} these eight parameters allow the weak interaction eigenstates and mass eigenstates to differ in both the quark and lepton sector.
\item \textbf{Strong \CP  phase:} this could allow \CP violation in the strong interaction, but is experimentally measured to be extremely small.
\end{description}

The relatively large number of free parameters in the SM may hint to this model only being a subset of a more complete theory. Trends are observed between the quantities, for example the masses of the fermions in each generation are each fairly similar in size. Some larger theory may be responsible for these trends.  

\subsection{Triumphs}
The SM predicts a number of processes to an extraordinary accuracy. 

\begin{itemize}
\item g-2
\item SM measurements
\end{itemize}

\subsection{Shortfalls}


\begin{itemize}
\item Baryon asymmetry 
\item Neutrino masses
\item Gravity
\item Dark matter
\item Strong \CP 
\item hierarchy problem
\end{itemize}
%\subsection{QCD, QED, Weak}

\section{Beyond the Standard Model}
As the SM is deficient in it's description of the natural world, a large number of theories proposing extensions have been hypothesised.


{\color{Red}
\begin{itemize}
\item SUSY
\item Extra Dimensions
\item Strong CP axions 
\end{itemize}
}
