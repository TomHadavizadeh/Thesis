\chapter{Introduction} 
\label{ch:introduction}

\minitoc 



\section{Overview} 


{\color{Red}
\begin{itemize}
\item Historical context 
\end{itemize}}


A brief introduction to the Standard Model of Particle Physics is detailed later in this chapter. A more detailed description of the relevant theoretical aspects that underpin the rare hadronic \B meson decays is described in Chapter~\ref{ch:theory}. A description of experimental apparatus used to collect data is included in Chapter~\ref{ch:detector}, followed by the methodology used to process this data in Chapter~\ref{ch:selection}. The statistical methods used to extract measurements of the decays can be found in Chapters~\ref{ch:B2DsKK} and~\ref{ch:B2DsPhi}.


The research presented in this thesis has been published in Ref.~\cite{LHCb-PAPER-2017-032}. 



\section{The Standard Model of Particle Physics}

The Standard Model (SM) provides explanation for a large number of natural phenomena to a high precision. There are currently a number of unresolved issues that are not explained, implying that this theory is not the complete description of the natural world. The theory was formulated in the 1970s, with particles being predicted and discovered by physicists around the world in the decades following.

\subsection{Building blocks}

The SM is composed of a small number of fundamental particles; some are the components of matter and others are force-carriers that mediate interactions. The fundamental particles can be combined into composites combination, bound by these fundamental interactions to form a rich tapestry of particles.

Mathematically, the theory is described by the unitary product group $\text{SU}(3)\times\text{SU}(2)\times\text{U}(1)$ 

The particles of the SM are either fermions or bosons.



{\color{Red}
\begin{itemize}
\item Fermions and bosons
\item QFT?
\item people?
\item Feynman diagrams
\end{itemize}}

\subsubsection{Fundamental fermions}

The fundamental spin-$1/2$ fermions that constitute matter can be divided into two categories: leptons and quarks. 
These particles can be divided up into \emph{families} or \emph{generations} containing two leptons and two quarks. Each of these contains an up-type and down-type quark, a charged lepton and a neutrino. There are three of these \emph{generations} that are identical clones of one another, differing only by their rest masses; each subsequent generation is heavier than the previous. Their interactions are identical in the SM. 
These particles are listed in Table~\ref{tab:intro_particles} along with their electromagnetic charge.
\begin{table}[h]
   \begin{center}
      \begin{tabular}{lcc | lcc}
         \hline
         \multicolumn{3}{c|}{Quarks} & \multicolumn{3}{c}{Leptons}\\
         \hline
         Name       & Symbol            & charge  & Name                & Symbol            & charge    \\ 
         \hline
         Up         & \uquark           &  $+2/3$ & Electron neutrino   & \neue             &  $0$   \\ 
         Down       & \dquark           &  $-1/3$ & Electron            & \en               &  $-1$  \\ 
         \hline
         Charm      & \cquark           &  $+2/3$ & Muon neutrino       & \neum             &  $0$   \\ 
         Strange    & \squark           &  $-1/3$ & Muon                & \mun              &  $-1$  \\ 
         \hline
         Top        & \tquark           &  $+2/3$ & Tau neutrino        & \neut             &  $0$   \\ 
         Bottom     & \bquark           &  $-1/3$ & Tau                 & \taum             &  $-1$  \\ 
         \hline
      \end{tabular}
   \end{center}
   \caption{The fundamental fermions in the standard model. The electromagnetic charges are given in units of the absolute value of the electron charge: $1.6\times 10^{-19}$\,C.}
   \label{tab:intro_particles}
\end{table}
The first generation, containing the \uquark, \dquark, \en and \neue particles, are the least massive generation and most stable. The second and third generations exist only fleetingly, before decaying back down to the first generation. Therefore, all stable matter in the universe is comprised of first-generation particles. 

\subsubsection{Antimatter}

For each of the fundamental fermions in the standard model there exists a anti-matter partner. First discovered by 

\subsubsection{Fundamental interactions}

The SM provides a mechanism for three fundamental forces: the electromagnetic, weak and strong forces. Each of these are mediated by force-carriers called gauge bosons.

\begin{table}[h]
   \begin{center}
      \begin{tabular}{ccccc}
         \hline

         Force              & Boson             & Symbol    & Mass      & Range                 \\
         \hline 
         Strong             & gluon             & $g$       & 0         &                       \\
         Electromagnetic    & photon            & \Pgamma   & 0         &                       \\
         Weak               & $W$ boson         & \Wpm      &           &                       \\
                            & $Z$ boson         & \Z        &           &                       \\
         \hline
         Gravity            & \emph{graviton?}  &           &           &                       \\

         \hline
      \end{tabular}
   \end{center}
   \caption{The fundamental interactions in the standard model.}
   \label{tab:lumi}
\end{table}

\subsection{The origin of mass}
The final component of the SM is the Higgs boson, a key component to allow the fundamental particles to gain mass. Discovered in 2012 at the Large Hadron Collider (cite) it constitutes the final particle predicted to be part of the SM. 

\subsection{Parameters}


\subsection{Triumphs}

\begin{itemize}
\item g-2
\item SM measurements
\end{itemize}

\subsection{Shortfalls}
\begin{itemize}
\item Baryon asymmetry 
\item Neutrino masses
\item Gravity
\item Dark matter
\item hierarchy problem
\end{itemize}
%\subsection{QCD, QED, Weak}
\section{Beyond the Standard Model}

