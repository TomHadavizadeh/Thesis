\chapter{Introduction} 
\label{ch:introduction}

\minitoc 



%\section{Overview} 


Throughout history, society has endeavoured to glimpse into the heart of matter, trying to break it down into elementary chunks.
The term \emph{atom} was used by the ancient Greek philosophers to speculate on the nature of matter. 
However, modern particle physics took until the 19th century to get going; the first of today's elementary particles, the electron, was discovered by J.J. Thomson in 1897~\cite{electron}.

Rutherford's discovery of the proton~\cite{Rutherford:1911zz} and Chadwick's discovery of the neutron~\cite{Chadwick:1932ma} all but completed the picture, as between them these three particles constitute essentially all stable matter in the universe.  
In 1934 Yukawa proposed that the neutrons and protons were bound together using an interaction appropriately called the strong force~\cite{193548}. This required a new particle to be hypothesised: the \emph{meson}. The mass of this particle would have to sit inbetween the light electron and heavier  protons and neutrons, hence the name meson meaning intermediate. The electron correspondingly became part of a class of particles call \emph{leptons} (light) and the protons and neutrons part of \emph{baryons} (heavy). 
The search for the strong force mediator resulted in the joint discoveries of the pion and muon~\cite{Lattes:1947mw,Lattes:1947mx}, both initially thought to be mesons. 


Over the next {\color{Red}something} years further pieces of the puzzle were uncovered, including anti-electrons, neutrinos and a wealth of mesons and baryons. 
The seemingly bizarre complexity of particles was eventually explained by Gell-Mann~\cite{GellMann:1964nj} and Zweig~\cite{Zweig:1964jf} through the quark model. This proposed that mesons and baryons were composed of elementary particles called \emph{quarks} that combined rather simply lead to the   


{\color{Red}
\begin{itemize}
\item atom used to mean a particle that can't be made into smaller pieces 
\item Historical context 
\item formulation of the SM
\end{itemize}}





A brief introduction to the Standard Model of Particle Physics is detailed later in this chapter. A more detailed description of the relevant theoretical aspects that underpin the rare hadronic \B meson decays is described in Chapter~\ref{ch:theory}. A description of experimental apparatus used to collect data is included in Chapter~\ref{ch:detector}, followed by the methodology used to process this data in Chapter~\ref{ch:selection}. The statistical methods used to extract measurements of the decays can be found in Chapters~\ref{ch:B2DsKK} and~\ref{ch:B2DsPhi}.


The research presented in this thesis has been published in Ref.~\cite{LHCb-PAPER-2017-032}. 



\section{The Standard Model of Particle Physics}

The Standard Model (SM) provides explanation for a large number of natural phenomena to a high precision. There are currently a number of unresolved issues that are not explained, implying that this theory is not the complete description of the natural world. The theory was formulated in the 1970s, with particles being predicted and discovered by physicists around the world in the decades following.

\subsection{Building blocks}

The SM is composed of a small number of fundamental particles; some are the components of matter and others are force-carriers that mediate interactions. The fundamental particles can be combined into composite combinations, bound by these fundamental interactions to form a rich tapestry of observable particles.

Mathematically, the theory is described by the unitary product group $\text{SU}(3)\times\text{SU}(2)\times\text{U}(1)$ 



{\color{Red}
\begin{itemize}
\item QFT?
\item Feynman diagrams
\end{itemize}}

The particles of the SM can be separated in to fundamental fermions (particles with half-integer spin) and fundamental bosons (particles with integer spin). These are described in the following sections.  


\subsubsection{Fundamental fermions}

The fundamental spin-$1/2$ fermions that constitute matter can be divided into two categories: leptons and quarks. 
These particles can be divided up into \emph{families} or \emph{generations} containing two leptons and two quarks. Each of these contains an up-type and down-type quark, a charged lepton and a neutrino. There are three of these \emph{generations} that are identical clones of one another, differing only by their rest masses; each subsequent generation is heavier than the previous. The interactions of these three generations via the fundamental forces are identical in the SM. 
These particles are listed in Table~\ref{tab:intro_particles} along with their electromagnetic charge.
\begin{table}[h]
   \begin{center}
      \begin{tabular}{lcr | lcr}
         \hline
         \multicolumn{3}{c|}{Quarks} & \multicolumn{3}{c}{Leptons}\\
         \hline
         Name       & Symbol            & Q  & Name                & Symbol            & $Q$    \\ 
         \hline
         Up         & \uquark           &  $+2/3$ & Electron neutrino   & \neue             &  $0$   \\ 
         Down       & \dquark           &  $-1/3$ & Electron            & \en               &  $-1$  \\ 
         \hline
         Charm      & \cquark           &  $+2/3$ & Muon neutrino       & \neum             &  $0$   \\ 
         Strange    & \squark           &  $-1/3$ & Muon                & \mun              &  $-1$  \\ 
         \hline
         Top        & \tquark           &  $+2/3$ & Tau neutrino        & \neut             &  $0$   \\ 
         Bottom     & \bquark           &  $-1/3$ & Tau                 & \taum             &  $-1$  \\ 
         \hline
      \end{tabular}
   \end{center}
   \caption{The fundamental fermions in the standard model. The electromagnetic charge, $Q$, is given in units of the absolute value of the electron charge: $1.6\times 10^{-19}$\,C.}
   \label{tab:intro_particles}
\end{table}
The first generation, containing the \uquark, \dquark, \en and \neue particles, is the least massive generation and most stable. The second and third generations exist only fleetingly, before decaying back down to the first generation. Therefore, all stable matter in the universe is comprised of first-generation particles. 

As detailed in Table~\ref{tab:intro_particles}, all fermions except neutrinos have a electromagnetic charge. 

 Additionally the 

{\color{Red}
\begin{itemize}
\item Degrees of freedom: colour, flavour
\end{itemize}}


\subsubsection{Antimatter partners}

For each of the fundamental fermions in the standard model there exists a anti-matter partner. The positive electron partner, called positron, was first discovered by C. Anderson in 1932~\cite{PhysRev.43.491}.

Antimatter particles only differ from their matter counterparts by their quantum numbers, such as charge or baryon and lepton number. The mass of a particles antimatter partner is identical. Matter and antimatter can recombine or \emph{annihilate}, releasing the energy stored in their rest masses. 


\subsubsection{Fundamental interactions}

The SM provides a mechanism for three fundamental forces: the electromagnetic, weak and strong forces. Each of these are mediated by force-carriers called gauge bosons.

\begin{table}[h]
   \begin{center}
      \begin{tabular}{ccccc}
         \hline

         Force              & Boson             & Symbol    & Mass (\gevcc)      & Range (\m)                 \\
         \hline 
         Strong             & gluon             & $g$       & 0         & $10^{-15}$                     \\
         Electromagnetic    & photon            & \Pgamma   & 0         & $\infty$                      \\
         \multirow{ 2}{*}{Weak}& $W$ boson      & \Wpm      & $80.385\pm0.015$ & \multirow{ 2}{*}{$10^{-18}$}                       \\
                            & $Z$ boson         & \Z        & $90.188\pm0.002$ &  \\
         \hline
         Gravity            & \emph{graviton?}  &           &           & $\infty$                      \\

         \hline
      \end{tabular}
   \end{center}
   \caption{The fundamental interactions in the standard model.}
   \label{tab:lumi}
\end{table}


{\color{Red}
\begin{itemize}
\item EM - Abeilan: no self coupling (except light by light scattering)
\item Weak, strong: non-abeilan: self coupling
\item Range: EM- massles photon -> infinite 
\item Range: Weak - massive bosons -> short range
\item Range: Strong - massless but self interaction/confinement -> short range (force is infinite )
\end{itemize}}


\subsection{The origin of mass}
The final component of the SM is the Higgs boson, a key component to allow the fundamental particles to gain mass. Discovered in 2012 at the Large Hadron Collider (cite) it constitutes the final particle predicted to be part of the SM. 

\subsection{Parameters}


\subsection{Triumphs}

\begin{itemize}
\item g-2
\item SM measurements
\end{itemize}

\subsection{Shortfalls}
\begin{itemize}
\item Baryon asymmetry 
\item Neutrino masses
\item Gravity
\item Dark matter
\item hierarchy problem
\end{itemize}
%\subsection{QCD, QED, Weak}
\section{Beyond the Standard Model}

