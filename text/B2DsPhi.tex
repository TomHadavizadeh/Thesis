\chapter{Mass fit to \decay{\Bp}{\Dsp\phiz} candidates} 
\label{ch:B2DsPhi}

\minitoc

In this chapter the methodology used to search for \decay{\Bp}{\Dsp\phiz} decays is described.
The branching fraction $\BF(\decay{\Bp}{\Dsp\phiz})$ is constructed by measuring the yield of \decay{\Bp}{\Dsp\phiz} decays relative to the normalisation channel \decay{\Bp}{\Dsp\Dzb}. This ratio is corrected by the ratio of selection efficiencies for the two modes. 
The branching fraction $\BF(\decay{\Bp}{\Dsp\phiz})$ is then determined by multiplying this corrected ratio by externally measured values for the branching fractions $\BF(\decay{\Bp}{\Dsp\Dzb})$ and $\BF(\decay{\Dzb}{\Kp\Km})$, and dividing by $\BF(\decay{\phiz}{\Kp\Km})$. 


{\color{Red}
\begin{itemize}
\item Details of blinding procedure for historic context?
\end{itemize}
}


\section{Fit strategy}
\label{sec:B2DsPhi_fitstrategy}
The strategy used to search for \decay{\Bp}{\Dsp\phiz} is more complicated than the method used to search for \decay{\Bp}{\Dsp\Kp\Km} decays as outlined in Chapter~\ref{ch:B2DsKK}. This is necessary to allow various different signal and background components to be distinguished. In particular, the majority of \decay{\Bp}{\Dsp\Kp\Km} decays now act as a background; those not proceeding via the \phiz resonance must be distinguished from \decay{\Bp}{\Dsp\phiz} decays. 
The ratio of \decay{\Bp}{\Dsp\phiz} and \decay{\Bp}{\Dsp\Dzb} yields is determined using a simultaneous unbinned maximum likelihood fit in a number of different categories. Three sets of categories are used, separating the candidates according to \Dsp meson decay mode, invariant mass of \Kp\Km pair, $m(\Kp\Km)$, and the cosine of an angle $\cos\theta_{K}$. The details and definitions of these categories are listed in Sec~\ref{sec:B2DsPhi_fit_cats}. 
The total extended NLL for this fit is created from the sum of each NLL in each of the categories
\begin{equation}
-\log\mathcal{L}(n_{0}...n_{j},\vec{p}) = \sum_{\alpha} \sum_{\beta} \sum_{\gamma} \left(-\log\mathcal{L^{\alpha,\beta,\gamma}}(n_{0}^{\alpha,\beta,\gamma}...n_{j}^{\alpha,\beta,\gamma},\vec{p}) \right)
\end{equation} 
where $\alpha$, $\beta$ and $\gamma$ represent indexes over the \Dsp mode, $m(\Kp\Km)$ and $\cos\theta_{K}$ categories.
The NLL for each category is defined as before
\begin{equation}
-\log\mathcal{L^{\alpha,\beta,\gamma}}(n_{0}^{\alpha,\beta,\gamma}...n_{j}^{\alpha,\beta,\gamma},\vec{p}) = -\sum_{i}^{N^{\alpha,\beta,\gamma}} \log \left( \sum_{j} n_{j}^{\alpha,\beta,\gamma} f_{j}^{\alpha,\beta,\gamma}(m=m_{i},\vec{p}) \right) + \sum_{j}n_{j}^{\alpha,\beta,\gamma}.
\end{equation} 
As before, $j$ represents the index over each contribution to the fit model, and $i$ represents each of $N^{\alpha,\beta,\gamma}$ entries in the data set for category $\alpha,\beta,\gamma$. 
The composite extended NLL is minimised with respect to the parameters $\vec{p}$ to find the values for which the data is most likely.
{\color{Red}
\begin{itemize}
\item Comment about MINOS
\item also hesse and the others?
\end{itemize}
}

\subsection{Simultaneous categories}
\label{sec:B2DsPhi_fit_cats}

The data sample is split into categories primarily to aid the differentiation of different signal and background components that contribute in similar invariant mass ranges.

\subsubsection{\Dsp meson decay mode} 
The three \Dsp decays modes used to reconstruct the signal and normalisation decays (\decay{\Dsp}{\Kp\Km\pip}, \decay{\Dsp}{\pip\pim\pip}, and \decay{\Dsp}{\Kp\pim\pip}) are fitted simultaneously in different categories. This allows the invariant mass distributions for the three modes to vary slightly in ways that could not be easily accounted for if the modes were combined in a single data set. In principle, the widths and resolutions of the \Bp meson mass distributions could vary for the three different modes as a result of the different numbers of pions and kaons in the final state. The background levels also differ between the modes as a result of the smaller branching fractions for \decay{\Dsp}{\pip\pim\pip} and \decay{\Dsp}{\Kp\pim\pip}. This leads to the background from combinations of unrelated tracks having a larger relative contribution.

The \decay{\Dsp}{\Kp\Km\pip} decay mode is additionally split into two further categories; candidates consistent with \decay{\Dsp}{\phiz\pip} decays, and non-\phiz candidates. This exploits the high purity of \decay{\Dsp}{\phiz\pip} decays. 

\subsubsection{Invariant mass of \Kp\Km pair, $m(\Kp\Km)$} 
Three distinct ranges of $m(\Kp\Km)$ invariant mass are used to split the candidates. The first of these corresponds to normalisation \decay{\Bp}{\Dsp\Dzb} candidates within the range $|m(\Kp\Km)-m(\Dzb)|<25 \mevcc$. These are reconstructed separately to the signal decays, unlike in the search for \decay{\Bp}{\Dsp\Kp\Km}, as detailed in Chapter~\ref{ch:selection}. The candidates reconstructed as \decay{\Bp}{\Dsp\phiz} decays are split into two ranges; those within $|m(\Kp\Km)-m(\phiz)|<10\mevcc$, referred to as the \emph{inner \phiz mass category} and those candidates with $10<|m(\Kp\Km)-m(\phiz)|<40\mevcc$, referred to as the \emph{outer \phiz mass category}. These two categories for the signal mode allow contributions from decays that do not proceed via a \phiz meson to be distinguished from those that do. The \emph{inner \phiz mass category} contains 88\% of signal \decay{\Bp}{\Dsp\phiz} candidates, with the other 12\% in the \emph{outer \phiz mass category}. The $m(\Kp\Km)$ invariant mass distribution for simulated \decay{\Bp}{\Dsp\phiz} is shown in Fig~\ref{fig:B2DsPhi_hel_mass_MC}.



%%%%%%%%%%%%%%%%%%%%%%%%%%%%%%%%%%%%%%%%%%%%%%%%%%%%%%%%%%
\begin{figure}[!h]
    \centering
    \begin{subfigure}[t]{0.48\textwidth}
        \centering
        \includegraphics[width=1.0\textwidth]{figs/B2DsPhi/MC_Distributions_mass_B2DsPhi.pdf}
    \end{subfigure}
    \begin{subfigure}[t]{0.48\textwidth}
        \centering
        \includegraphics[width=1.0\textwidth]{figs/B2DsPhi/MC_Distributions_angle_B2DsPhi.pdf}
    \end{subfigure}
    \caption{The distributions of $m(\Kp\Km)$ (left) and $\cos\theta_{K}$ (right) in simulated \decay{\Bp}{\Dsp\phiz} decays. The vertical blue dashed lines represent the boundaries between categories defined in Sec.~\ref{sec:B2DsPhi_fit_cats}. The vertical red lines represent the mass window applied to candidates; those outside the red lines are not included in the data set.}
    \label{fig:B2DsPhi_hel_mass_MC}   
\end{figure}
%%%%%%%%%%%%%%%%%%%%%%%%%%%%%%%%%%%%%%%%%%%%%%%%%%%%%%%%%%


%MC_Distributions_mass_B2DsPhi.pdf



\subsubsection{Helicity angle, $\cos\theta_{K}$} 

The $\decay{\Bp}{\Dsp\phiz}$ decay involves the decay of a pseudoscalar particle to a pseudoscalar and vector particle. Therefore the \phiz vector meson ($J^{P} = 1^{-}$) must be produced longitudinally polarised. For a longitudinally polarised \phiz meson decaying to $\Kp\Km$, the distribution of the angle $\theta_{K}$, defined as the angle that the kaon meson forms with the \Bp momentum in the \phiz rest frame (Fig.~\ref{fig:B2DsPhi_helicity_angle}), is proportional to $\cos^{2}{\theta_{K}}$. The distribution of $\cos{\theta_{K}}$ for $\decay{\Bp}{\Dsp\phiz}$ as determined from simulated events is shown in Fig~\ref{fig:B2DsPhi_hel_mass_MC}. Candidates are split into two categories; $|\cos{\theta_{K}} |> 0.4$ and $|\cos{\theta_{K}} |< 0.4$. These categories contain 93\% and 7\% of the signal respectively.
%%%%%%%%%%%%%%%%%%%%%%%%%%%%%%%%%%%%%%%%%%%%%%%%%%%%%%%%%%
\begin{figure}[!h]
    \centering
    \includegraphics[width=0.48\textwidth]{figs/B2DsPhi/helicityangle.pdf}
    \caption{The angle $\theta_{K}$ (referred to as the helicity angle) is defined to be the angle that the kaon mesons forms with the \Bp meson momentum in the \phiz rest frame.}
    \label{fig:B2DsPhi_helicity_angle}   
\end{figure}
%%%%%%%%%%%%%%%%%%%%%%%%%%%%%%%%%%%%%%%%%%%%%%%%%%%%%%%%%%

This helicity angle is constructed using the momentum of the decay products calculated after the whole decay chain has been refitted with a \Dsp mass and \Bp direction constraint. This significantly increases the fraction of signal events expected in the first of the two categories.


\begin{table}[t]
   \centering
   \begin{tabular}{c|cc}
      \hline
      \multirow{2}{*}{$| m(\Kp\Km) - m_{\phi} |$ (\mevcc)}   & \multicolumn{2}{c}{Helicity Category} \\ 
                       & $|\cos{\theta_{K}} |> 0.4$          & $|\cos{\theta_{K}} |< 0.4$\\ 
      \hline
      $< 10$                           & 82\%           & 6\%                       \\
      (10, 40)                         & 11\%           & 1\%                       \\
      \hline
  \end{tabular}
  \caption{Fractions of $\decay{\Bp}{\Dsp\phiz}$ candidates expected in the helicity and $m(\Kp\Km)$ invariant mass categories of the simultaneous fit. }
  \label{tab:signal_ratios}
\end{table}


\subsection{\decay{\Bp}{\Dsp \Kp \Km} model and assumptions}
\label{sec:B2DsPhi_B2DsKKModel}
The search for \decay{\Bp}{\Dsp\phiz} decays includes a component for \decay{\Bp}{\Dsp\Kp\Km} decays that didn't proceed via a \phiz meson. This is necessary as the search documented in Chapter~\ref{ch:B2DsKK} determined there is a non-zero contribution from these decays in the range of $m(\Kp\Km)$ invariant mass considered here (Fig~\ref{fig:B2DsKK_twobodyprojections}). To avoid overestimating \decay{\Bp}{\Dsp\phiz} signal yield, separate components are included in the fit model for the \decay{\Bp}{\Dsp\phiz} and \decay{\Bp}{\Dsp\Kp\Km} decays. Although the invariant mass distributions of these contributions are identical, they can still be disentangled by exploiting the different fractions of these decays expected in each of the helicity angle and $m(\Kp\Km)$ categories. 
The fractions for the \decay{\Bp}{\Dsp\phiz} signal decays as listed in Table~\ref{tab:signal_ratios} show the decays are concentrated in the \emph{inner \phiz mass category} with $|\cos{\theta_{K}} |> 0.4$. 
To determine similar fractions for \decay{\Bp}{\Dsp\Kp\Km} decays the \laurapp package {\color{Red}(cite)} is used to generate a number of simulation samples for different intermediate resonance models.

Only resonances in the \Kp\Km system are considered as no significant structure is observed in the $m(\Dsp\Km)$ distribution in Fig.~\ref{fig:B2DsKK_twobodyprojections}. As such, all resonances are neutral mesons. The models are generated separately, therefore the effect of interference between any combination of states has been entirely neglected.    
The generated samples are described in the following sections.

\subsubsection{The $\phi(1020)$ resonance} 
Decays proceeding via a $\phiz(1020)$ are produced as a crosscheck. As the simulations generated with \laurapp have not been reconstructed with the full \lhcb detector model, this sample is compared to the existing full simulation samples. The differences in between the fraction of the decays in the different $m(\Kp\Km)$ and $\cos\theta_{K}$ categories in the two samples are taken as a proxy for the potential level of bias introduced by using these generator level samples instead of full simulation. The distribution of these simulated decays in $m(\Kp\Km)$ and $\cos\theta_{K}$ are shown in Fig.~\ref{fig:DsKK_model_phi1020}. This figure also include the Dalitz plot distribution of the decays parametrised with the variables $m^{2}(\Dsp\Km)$ and $m^{2}(\Kp\Km)$.
This resonance is generated with a Relativistic Breit-Wigner line shape~\cite{RelBWPhysRev.49.519}.
%%%%%%%%%%%%%
\begin{figure}[!h]
   \centering   
   \includegraphics[width=0.32\textwidth]{figs/B2DsPhi/phi_phi_mass.pdf}
   \includegraphics[width=0.32\textwidth]{figs/B2DsPhi/phi_Helicity.pdf}
   \includegraphics[width=0.32\textwidth]{figs/B2DsPhi/phi_Dalitz_plot.pdf}
   \caption{The distribution of $m(\Kp\Km)$ (left), Dalitz plot (middle) and the helicity angle $\cos\theta_{K}$ for generated for the $\phiz(1020)$ resonance.} 
   \label{fig:DsKK_model_phi1020}   
\end{figure}
%%%%%%%%%%%%%

\subsubsection{Non-resonant decays}

In addition to \Kp\Km resonances, a non-resonant model is considered. This model is defined to have a uniform amplitude across the allowed phase-space. The distribution in $m(\Kp\Km)$ of \decay{\Bp}{\Dsp\Kp\Km} decays in Fig.~\ref{fig:B2DsKK_twobodyprojections} is not consistent with this model as there are no candidates above $m(\Kp\Km) \sim 1900\mevcc$. However, this component is included in this study for comparative purposes. The distributions of decays generated with this flat model are shown in Fig.~\ref{fig:DsKK_model_NR}. 

%%%%%%%%%%%%%
\begin{figure}[!h]
   \centering   
   \includegraphics[width=0.32\textwidth]{figs/B2DsPhi/NR_phi_mass.pdf}
   \includegraphics[width=0.32\textwidth]{figs/B2DsPhi/NR_Helicity.pdf}
   \includegraphics[width=0.32\textwidth]{figs/B2DsPhi/NR_Dalitz_plot.pdf}
   \caption{The distribution of $m(\Kp\Km)$ (left), Dalitz plot (middle) and the helicity angle $\cos\theta_{K}$ for generated for non-resonant decays.} 
   \label{fig:DsKK_model_NR}   
\end{figure}
%%%%%%%%%%%%%

\subsubsection{The $f_{0}^{0}(980)$ resonance}

The $f_{0}^{0}(980)$ resonance is a light unflavoured $J^{P} = 0^{+}$ state with mass $990\pm20\mevcc$ and width 10--100\mevcc~\cite{PDG2016}. It has been observed to decay to $\Kp\Km$ making it a suitable resonance to consider. Although it's mass is at the lower end of the range considered here, it's significant width allows it to contribute at higher invariant masses. This component is modelled with the Flatt\'{e} line shape~\cite{FLATTE1976224} and the relevant distributions shown in Fig.~\ref{fig:DsKK_model_f0980}.
%%%%%%%%%%%%%
\begin{figure}[!h]
   \centering   
   \includegraphics[width=0.32\textwidth]{figs/B2DsPhi/f0_phi_mass.pdf}
   \includegraphics[width=0.32\textwidth]{figs/B2DsPhi/f0_Helicity.pdf}
   \includegraphics[width=0.32\textwidth]{figs/B2DsPhi/f0_Dalitz_plot.pdf}
   \caption{The distribution of $m(\Kp\Km)$ (left), Dalitz plot (middle) and the helicity angle $\cos\theta_{K}$ for generated for the $f_{0}^{0}(980)$ resonance.} 
   \label{fig:DsKK_model_f0980}   
\end{figure}
%%%%%%%%%%%%%

\subsubsection{The $a_{0}^{0}(980)$ resonance}
The $a_{0}^{0}(980)$ resonance is a light unflavoured $J^{P} = 0^{+}$ state with mass $980\pm20\mevcc$ and width 50--100\mevcc and has been observed to decay to $\PK\Kb$ final states~\cite{PDG2016}. This resonance is also modelled with the Flatt\'{e} line shape and the relevant distributions are shown in Fig~\ref{fig:DsKK_model_a0980}.
%%%%%%%%%%%%%
\begin{figure}[!h]
   \centering   
   \includegraphics[width=0.32\textwidth]{figs/B2DsPhi/a0_phi_mass.pdf}
   \includegraphics[width=0.32\textwidth]{figs/B2DsPhi/a0_Helicity.pdf}
   \includegraphics[width=0.32\textwidth]{figs/B2DsPhi/a0_Dalitz_plot.pdf}
   \caption{The distribution of $m(\Kp\Km)$ (left), Dalitz plot (middle) and the helicity angle $\cos\theta_{K}$ for generated for the $a_{0}^{0}(980)$ resonance.} 
   \label{fig:DsKK_model_a0980}   
\end{figure}
%%%%%%%%%%%%%

\subsubsection{The $f_{0}^{0}(1370)$ resonance}
The $f_{0}^{0}(1370)$ resonance is a light unflavoured $J^{P} = 0^{+}$ state with a mass in the range 1200--1500\mevcc and width in the range 200--500\mevcc. It has been observed to decay to the \kaon\Kb final states. It is modelled with a Relativistic Breit-Wigner line shape and the relevant distributions are shown in Fig.~\ref{fig:DsKK_model_f01370}.
%%%%%%%%%%%%%
\begin{figure}[!h]
   \centering   
   \includegraphics[width=0.32\textwidth]{figs/B2DsPhi/f0_1370_phi_mass.pdf}
   \includegraphics[width=0.32\textwidth]{figs/B2DsPhi/f0_1370_Helicity.pdf}
   \includegraphics[width=0.32\textwidth]{figs/B2DsPhi/f0_1370_Dalitz_plot.pdf}
   \caption{The distribution of $m(\Kp\Km)$ (left), Dalitz plot (middle) and the helicity angle $\cos\theta_{K}$ for generated for the $f_{0}^{0}(1370)$ resonance.} 
   \label{fig:DsKK_model_f01370}   
\end{figure}
%%%%%%%%%%%%%

\subsubsection{The $f_{2}^{0}(1270)$ resonance}
The $f_{2}^{0}(1270)$ resonance is a $J^{P} = 2^{+}$ state with mass $1275.5\pm08\mevcc$ and width $186.7^{+2.2}_{-2.5}\mevcc$ that has been observed to decay to \kaon\Kb final states. This resonance is modelled with a Relativistic Breit-Wigner line shape as shown in Fig.~\ref{fig:DsKK_model_f21270}.

%%%%%%%%%%%%%
\begin{figure}[!h]
   \centering   
   \includegraphics[width=0.32\textwidth]{figs/B2DsPhi/f2_phi_mass.pdf}
   \includegraphics[width=0.32\textwidth]{figs/B2DsPhi/f2_Helicity.pdf}
   \includegraphics[width=0.32\textwidth]{figs/B2DsPhi/f2_Dalitz_plot.pdf}
   \caption{The distribution of $m(\Kp\Km)$ (left), Dalitz plot (middle) and the helicity angle $\cos\theta_{K}$ for generated for the $f_{2}^{0}(1270)$ resonance.} 
   \label{fig:DsKK_model_f21270}   
\end{figure}
%%%%%%%%%%%%%


\subsubsection{The $a_{2}^{0}(1320)$ resonance}
The $a_{2}^{0}(1320)$ resonance is a $J^{P} = 2^{+}$ state with a mass $1318.1\pm0.7\mevcc$ and width $109.8\pm2.4\mevcc$ (both measured in the \kaon\Kb mode) observed decaying to the \kaon\Kb final state. This resonance is modelled with a Relativistic Breit-Wigner line shape and shown in Fig.~\ref{fig:DsKK_model_a21320}.
%%%%%%%%%%%%%
\begin{figure}[!h]
   \centering   
   \includegraphics[width=0.32\textwidth]{figs/B2DsPhi/a2_1320_phi_mass.pdf}
   \includegraphics[width=0.32\textwidth]{figs/B2DsPhi/a2_1320_Helicity.pdf}
   \includegraphics[width=0.32\textwidth]{figs/B2DsPhi/a2_1320_Dalitz_plot.pdf}
   \caption{The distribution of $m(\Kp\Km)$ (left), Dalitz plot (middle) and the helicity angle $\cos\theta_{K}$ for generated for the $a_{2}^{0}(1320)$ resonance.} 
   \label{fig:DsKK_model_a21320}   
\end{figure}
%%%%%%%%%%%%%



\subsubsection{Summary of models}

The fraction of decays expected in each $m(\Kp\Km)$ and $\cos\theta_{K}$ category for the different models considered are tabulated in Table~\ref{table:DsKK_rescfracs}. These are calculated by counting the numbers of entries in the corresponding ranges delineated by the vertical lines in Figs.~\ref{fig:DsKK_model_phi1020}-\ref{fig:DsKK_model_a21320}. For reference the $\phi(1020)$ fractions are included for both the fully simulated decays and those generated with \laurapp. The maximum difference between these fractions is included as a source of systematic uncertainty in Sec.~\ref{sec:B2DsPhi_systuncertainy}.
between For all of the models considered it is clear that the \phiz resonance has significantly different fractions to all of the models considered, allowing the component to be distinguished.   


%%%%%%%%%%%%%%%%%%%%%%%%%%%%%%%%%%%%%%%%%%%%%%%%%%%%%%%%%% 
% Table of things
%%%%%%%%%%%%%%%%%%%%%%%%%%%%%%%%%%%%%%%%%%%%%%%%%%%%%%%%%% 
\begin{table}[!ht]
   \centering
   \begin{tabular}{ c  c  c  c  c }

      \hline
      \multirow{ 2}{*}{\textbf{Model }} & \multicolumn{2}{c}{$|\Delta m|<10\mevcc$} & \multicolumn{2}{c}{$10<|\Delta m|<40\mevcc$} \\
         & $|\cos{\theta_{K}}|>0.4$ & $|\cos{\theta_{K}}|<0.4$ & $|\cos{\theta_{K}}|>0.4$ & $|\cos{\theta_{K}}|<0.4$ \\
      \hline 
      $\phi(1020)$ \laurapp          & 83.5  &  5.7  & 10.2  &  0.7   \\
      $\phi(1020)$ Full              & 82.4  &  5.9  & 10.9  &  0.8   \\
      \hline
      Non-resonant                   & 16.3  & 11.1  & 45.4  & 27.2   \\
      $f_{0}^{0}(980)$               & 16.5  & 11.2  & 43.3  & 29.0   \\
      $a_{0}^{0}(980)$               & 12.7  &  8.7  & 47.0  & 31.5   \\
      $f_{0}^{0}(1370)$              & 16.1  &  9.6  & 45.0  & 29.3   \\
      $f_{2}^{0}(1270)$              &  8.6  &  6.5  & 59.1  & 25.8   \\
      $a_{2}^{0}(1320)$              &  9.7  &  9.7  & 51.6  & 29.0   \\
      \hline
      Chosen fractions              & 14.6 $\pm$ 1.9  & 10.0 $\pm$ 1.3  & 45.2 $\pm$ 1.9  & 30.3 $\pm$ 1.3   \\
      \hline
   \end{tabular}
   \caption{Fractions of decays expected in each $m(\Kp\Km)$ and $\cos\theta_{K}$ category for the various resonance models considered in Sec.~\ref{sec:B2DsPhi_B2DsKKModel} ($\Delta m = m(\Kp\Km)- m(\phiz)$). }
   \label{table:DsKK_rescfracs}
\end{table}
%%%%%%%%%%%%%%%%%%%%%%%%%%%%%%%%%%%%%%%%%%%%%%%%%%%%%%%%%%

In order to choose suitable fractions for the \decay{\Bp}{\Dsp\Kp\Km} fit component a number of points are considered;

\begin{itemize}
\item Very few events are observed above 2000\mevcc in the background-subtracted $m(\Kp\Km)$ distribution, therefore the non-resonant model is neglected.
\item No significant peaking structure is observed in the $m(\Kp\Km)$ spectrum so on-shell resonances are neglected
\item The helicity distribution shows no distinctive structure so spin zero states are favoured.  
\item As this is not a full amplitude analysis no attempt is made to include the effects of interference, either between the remaining off-shell resonances or between these and any possible \decay{\Bp}{\Dsp\phiz} decays.
\end{itemize}

These considerations leave the $f_{0}^{0}(980)$ and $a_{0}^{0}(980)$ resonances. The fractions of \decay{\Bp}{\Dsp\Kp\Km} decays that have been used in the fit model are fixed to the average of these two, listed in the final row of Table~\ref{table:DsKK_rescfracs}. Uncertainties are assigned that correspond to half the difference of the two values. These uncertainties are propagated to the $\BF(\decay{\Bp}{\Dsp\phiz})$ branching fraction in Sec.~\ref{sec:B2DsPhi_systuncertainy}. 



\section{Fit components}
\label{sec:B2DsPhi_fitcomponents}

The yields of \decay{\Bp}{\Dsp\phiz} and \decay{\Bp}{\Dsp\Dzb} decays are extracted from the invariant mass distributions of the data sets by representing each component by probability density functions. The components are broadly very similar to those considered in the search for \decay{\Bp}{\Dsp\Kp\Km} decays detailed in Sec.~\ref{sec:B2DsKK_fitcomps}. There are a number of necessary differences:
\begin{itemize}
\item The \Bp invariant mass range considered for both the signal and normalisation channel is expanded to 4900--5900\mevcc. This allows a more stable determination of the various backgrounds contributing in the vicinity of the signal decays. In particular it stabilises the fraction of decays assigned to the combinatorial and partially reconstructed backgrounds at low \Bp invariant mass.   
\item More components are included in the model. This include both the signal mode, \decay{\Bp}{\Dsp\phiz}, and a closely related additional background mode \decay{\Bp}{\Dssp\phiz}. 
\end{itemize}


\subsection{Signal and normalisation decays}
\label{sec:B2DsPhi_signalcomps}

The invariant mass distributions of \decay{\Bp}{\Dsp\phiz} and \decay{\Bp}{\Dsp\Dzb} decays are parametrised using the same DCB function as in the search for \decay{\Bp}{\Dsp\Kp\Km} decays (Sec.~\ref{sec:B2DsKK_sigcomps}):
\begin{equation}
\text{DCB}(m|\mu,\sigma_1,\sigma_2,n,\alpha) = f_\sigma \times \text{CB}(m|\mu,\sigma_1,n,\alpha) + (1-f_\sigma) \times \text{CB}(m|\mu,\sigma_2,n,\alpha),
\label{eq:DoubleBD}
\end{equation}
where the CB function is defined as follows
\begin{equation}
\text{CB}(m|\mu,\sigma,n,\alpha) = \left \{
  \begin{aligned}
    &e^{-\frac{1}{2} \left(\frac{m-\mu}{\sigma}\right)^2} && \text{if}\ \left(\frac{m-\mu}{\sigma}\right) < -|\alpha|\\
    &\frac{\left(\frac{n}{|\alpha|}\right)^n\times e ^{-\frac{1}{2}|\alpha|^2} }{\left(\frac{n}{|\alpha|}-|\alpha| - \frac{m-\mu}{\sigma}\right)^n} && \text{otherwise.}
  \end{aligned} \right.
\end{equation}
Again, $\mu$, $\sigma$, $n$, $\alpha$ and $f_{\sigma}$ are adjustable parameters and $m$ is the \B meson invariant mass observable.
The tail parameter $\alpha$ is fixed to values determined from maximum likelihood fits to simulated candidates for the signal and normalisation decays. The parameter $n$ is fixed to unity in the fits to both simulation and data to increase the stability of the tails.     
The two CB function are allowed have different widths, $\sigma_{1}$ and $\sigma_{2}$, but the ratio $\sigma_{1}/\sigma_{2}$ is fixed from the fits to simulations, as is $f_{\sigma}$ that determines the fractional contribution of the narrower CB function ($\sigma_{1}<\sigma_{2}$). The values determined for each of these fixed parameters are tabulated in Table~\ref{tab:DsPhi_mc_fits}, along with the uncertainty obtained from the fits.

An extra constraint is added to the signal and normalisation DCB functions with respect to the configuration used in the search for \decay{\Bp}{\Dsp\Kp\Km} decays. As the number of signal candidates is likely to be small, the relative width of the narrower signal and normalisation CB functions, $\sigma_{1}(\Dsp\phi) / \sigma_{1}(\Dsp\Dzb)$, is also fixed to values obtained from the fits to simulations. All fixed parameters are determined separately for the different \Dsp decay modes, and the results of the fits to simulated decays are shown in Fig.~\ref{fig:B2DsPhi_signal_fits}.     


%%%%%%%%%%%%%%%%%%%%%%%%%%%%%%%%%%%%%%%%%%%%%%%%%%%%%%%%%% 
\begin{table}[h]
   \centering
   \begin{tabular}{ c c c c }
      \hline
      \multirow{2}{*}{Parameter}                   & \multicolumn{3}{c} {Value} \\
      \cline{2-4}
                                  & \decay{\Dsp}{\Kp\Km\pip}   & \decay{\Dsp}{\Kp\pim\pip} & \decay{\Dsp}{\pip\pim\pip}  \\
      \hline
      %\textbf{$\B \to \Ds \phi$}  &                    &                    &                        \\
      \multicolumn{4}{l} {\decay{\Bp}{\Dsp\phiz}}\\

      \hline
      $\sigma_1/\sigma_2$         & 0.49 $\pm$ 0.01    & 0.47 $\pm$ 0.01    & 0.46 $\pm$ 0.01        \\
      $f_\sigma$                  & 0.80 $\pm$ 0.01    & 0.84 $\pm$ 0.01    & 0.81 $\pm$ 0.01        \\
      $\alpha$                    & 2.76 $\pm$ 0.07    & 3.06 $\pm$ 0.16    & 3.71 $\pm$ 0.23        \\
      $n$                         & 1 $\pm$ 0          & 1  $\pm$ 0         & 1  $\pm$ 0             \\
      \hline
      %\textbf{ }  &                    &                    &                        \\
      \multicolumn{4}{l} {\decay{\Bp}{\Dsp\Dzb}}\\
      \hline
      $\sigma_1/\sigma_2$         & 0.43 $\pm$ 0.01    & 0.42 $\pm$ 0.01    & 0.40 $\pm$ 0.01        \\
      $f_\sigma$                  & 0.88 $\pm$ 0.01    & 0.88 $\pm$ 0.01    & 0.88 $\pm$ 0.01        \\
      $\alpha$                    & 2.91 $\pm$ 0.06    & 3.36 $\pm$ 0.26    & 3.53 $\pm$ 0.25        \\
      $n$                         & 1 $\pm$ 0          & 1 $\pm$ 0          & 1 $\pm$ 0              \\
      \hline 
      $\sigma_{1}(\Dsp\phi) / \sigma_{1}(\Dsp\Dzb)$ & 1.27 $\pm$ 0.02 & 1.31 $\pm$ 0.02 & 1.26 $\pm$ 0.02 \\
      \hline
   \end{tabular}
   \caption{Fixed values obtained in fits to MC used in the model for the signal pdf.} 
   \label{tab:DsPhi_mc_fits}  
\end{table}
%%%%%%%%%%%%%%%%%%%%%%%%%%%%%%%%%%%%%%%%%%%%%%%%%%%%%%%%%% 


%%%%%%%%%%%%%%%%%%%%%%%%%%%%%%%%%%%%%%%%%%%%%%%%%%%%%%%%%%
\begin{figure}[!h]
   \centering
   \begin{subfigure}[t]{1.0\textwidth}
      \centering
      \includegraphics[width=0.40\textwidth]{figs/B2DsPhi/Plot_Signal_Fit_All_B2PhiDs_Ds2KKPi.pdf}
      \includegraphics[width=0.40\textwidth]{figs/B2DsPhi/Plot_Signal_Fit_All_B2D0Ds_Ds2KKPi.pdf}
   \caption{\decay{\Dsp}{\Kp\Km\pip}}
   \end{subfigure}\\
   \begin{subfigure}[t]{1.0\textwidth}
      \centering
      \includegraphics[width=0.40\textwidth]{figs/B2DsPhi/Plot_Signal_Fit_All_B2PhiDs_Ds2PiPiPi.pdf}
      \includegraphics[width=0.40\textwidth]{figs/B2DsPhi/Plot_Signal_Fit_All_B2D0Ds_Ds2PiPiPi.pdf}
      \caption{\decay{\Dsp}{\pip\pim\pip}}
   \end{subfigure}\\
   \begin{subfigure}[t]{1.0\textwidth}
      \centering
      \includegraphics[width=0.40\textwidth]{figs/B2DsPhi/Plot_Signal_Fit_All_B2PhiDs_Ds2KPiPi.pdf}
      \includegraphics[width=0.40\textwidth]{figs/B2DsPhi/Plot_Signal_Fit_All_B2D0Ds_Ds2KPiPi.pdf}
      \caption{\decay{\Dsp}{\Kp\pim\pip}}
   \end{subfigure}\\
   \caption{Invariant mass fits to simulated signal (left) and normalisation (right) decays. The results of maximum likelihood fits using the signal PDFs are over laid, with the total function in black and the two contributing CB shapes in red and blue.}
   \label{fig:B2DsPhi_signal_fits}   
\end{figure}
%%%%%%%%%%%%%%%%%%%%%%%%%%%%%%%%%%%%%%%%%%%%%%%%%%%%%%%%%%


\subsection{Partially reconstructed backgrounds}
\label{sec:B2DsPhi_partrecocomps}

The accurate parametrisation of partially reconstructed backgrounds is particularly important in the search for \decay{\Bp}{\Dsp\phiz} decays as many different processes contribute to the low invariant mass range of the $m(\Dsp\phiz)$ spectrum. These processes involve decays of \Bs, \Bz or \Bp mesons in which the five final state tracks reconstructed in the search for \decay{\Bp}{\Dsp\phiz} decays are only a subset of the background modes final state. 
Typically, processes in which a low momentum pion or photon has not been reconstructed are found closest in mass to the signal decays. Decays of \Bs mesons are particularly dominant as the \Bs meson has a larger mass than the \Bp meson.   


\subsubsection{Backgrounds to the normalisation channel}

The modes \decay{\Bp}{\Dsp\Dstarzb} and \decay{\Bp}{\Dssp\Dzb} can both contribute as partially reconstructed backgrounds to the \decay{\Bp}{\Dsp\Dzb} normalisation mode. These are parametrised using the same PDFs as in the search for \decay{\Bp}{\Ds\Kp\Km} decays, detailed in Sec.~\ref{sec:B2DsKK_norm_partreco}. 

The invariant mass fit range is wider than in the fit to \decay{\Bp}{\Dsp\Kp\Km} candidates, therefore an extra contribution is included in the fit model to account for partially reconstructed \decay{\Bp}{\Dssp\Dstarzb} decays at lower invariant masses.

\begin{description}
\item \textbf{\decay{\Bp}{(\decay{\Dssp}{\Dsp[\piz]})\Dzb} and \decay{\Bp}{\Dsp(\decay{\Dstarzb}{\Dzb[\piz]})}:} these components are modelled by a parabola convolved with a resolution Gaussian. The parabola has a minimum in the centre and doesn't extend beyond endpoints $a$ and $b$
\begin{equation}
f(m|a,b,\sigma,\xi, \delta) = \int_{a}^{b}\left(\mu-\frac{a+b}{2}\right)^{2} \left( \frac{1-\xi}{b-a}\mu + \frac{b\xi-a}{b-a} \right) e^{-\frac{-(\mu-(m-\delta))^{2}}{2\sigma^{2}}} d\mu.
\label{eq:DsPhi_RooHorns}
\end{equation}
These components are shown by the black lines in Fig.~\ref{fig:B2DsPhi_DsD0_partreco}.

\item \textbf{\decay{\Bp}{(\decay{\Dssp}{\Dsp[\Pgamma]})\Dzb} and \decay{\Bp}{\Dsp(\decay{\Dstarzb}{\Dzb[\Pgamma]})}:} these components are modelled by a parabola convolved with a resolution Gaussian. The parabola has a maximum in the centre and doesn't extend beyond endpoints $a$ and $b$
\begin{equation}
f(m|a,b,\sigma,\xi, \delta) = \int_{a}^{b} -(\mu-a)(\mu-b)\left( \frac{1-\xi}{b-a}\mu + \frac{b\xi-a}{b-a} \right) e^{-\frac{-(\mu-(m-\delta))^{2}}{2\sigma^{2}}} d\mu.
\label{eq:DsPhi_RooHills}
\end{equation}
These components are shown by the blue lines in Fig.~\ref{fig:B2DsPhi_DsD0_partreco}.
\end{description}

%%%%%%%%%%%%%%%%%%%%%%%%%%%%%%%%%%%%%%%%%%%%%%%%%%%%%%%%%%
\begin{figure}[!h]
    \centering
    \includegraphics[width=0.80\textwidth]{figs/B2DsPhi/DsD0_part_reco_Shapes.pdf}
    \caption{Partially reconstructed $\Dsp\Dzb$ shapes.}
    \label{fig:B2DsPhi_DsD0_partreco}   
\end{figure}
%%%%%%%%%%%%%%%%%%%%%%%%%%%%%%%%%%%%%%%%%%%%%%%%%%%%%%%%%%


\begin{description}

\item \textbf{\decay{\Bp}{\Dssp\Dstarzb}:} the lower invariant mass range used in this search necessitates including a PDF for \decay{\Bp}{\Dssp\Dstarzb} decays in which two soft particles have been missed, one from each of the excited \D meson decays. It is possible for either a \piz or \Pgamma to be not reconstructed in the decays of both excited \D mesons. Additionally, as this process involves a psecudo-scalar meson decaying to two vector mesons, there should be two distinguishable helicity combinations for each process. This leads to a total of eight PDFs necessary to fully parametrise this contribution. Instead, however, this component is parametrised using a single function of the form given in Eq.~\ref{eq:DsPhi_RooHills}. The endpoints $a$ and $b$ are estimated by combining the effects of missing two neutral particles. The resulting distribution is shown in Fig.~\ref{fig:B2DsPhi_DsstarDstar0_partreco}.
The choice of PDF for this component is found to have negligible effect on the determination of the $\BF(\decay{\Bp}{\Dsp\phiz})$ branching fraction as detailed in Sec~\ref{sec:B2DsPhi_systuncertainy}.

\end{description}

%%%%%%%%%%%%%%%%%%%%%%%%%%%%%%%%%%%%%%%%%%%%%%%%%%%%%%%%%%
\begin{figure}[!h]
    \centering
    \includegraphics[width=0.80\textwidth]{figs/B2DsPhi/DsstarDstar0_part_reco_Shapes.pdf}
    \caption{Partially reconstructed $\Dsp\Dzb$ shapes.}
    \label{fig:B2DsPhi_DsstarDstar0_partreco}   
\end{figure}
%%%%%%%%%%%%%%%%%%%%%%%%%%%%%%%%%%%%%%%%%%%%%%%%%%%%%%%%%%



\subsubsection{Backgrounds to the signal channel}

%The signal channel invariant mass distribution receives contributions from the partially reconstructed modes considered in the search for \decay{\Bp}{\Dsp\Kp\Km} decays. The PDFs determined are 



\begin{description}
\item \textbf{\decay{\Bp}{(\decay{\Dssp}{\Dsp[\Pgamma]})\phiz} and \decay{\Bp}{(\decay{\Dssp}{\Dsp[\piz]})\phiz}:} the decays of \Bp mesons to an excited \Dsp meson and a \phiz meson could contribute to the $m(\Dsp\phiz)$ spectrum at low invariant masses when either a \piz or \Pgamma is missed from the excited meson decay. The resulting invariant mass distribution depends on the mass and spin of the non-reconstructed particle, as well as the helicity state of the \Dssp meson. This background involves the decay of a pseudo-scalar meson to two vector mesons, hence, as a result of angular momentum conservation, there are three helicity states of the \Dssp meson to consider. These are labelled \emph{001}, \emph{010} and \emph{100}. The two transversely polarised states, \emph{100} and \emph{001}, have identical invariant mass distributions and are therefore referred collectively as \emph{101}. This leads to a total of four contributions to consider for \decay{\Bp}{\Dssp\phiz} decays. These can be parametrised by parabolas convolved with resolution Gaussians in a similar way to the partially reconstructed backgrounds to the normalisation channel.
These each share the same functional form
\begin{equation}
f(m|a,b,\sigma,\xi, \delta) = \int_{a}^{b} g(\mu,a,b) \left( \frac{1-\xi}{b-a}\mu + \frac{b\xi-a}{b-a} \right) e^{-\frac{-(\mu-(m-\delta))^{2}}{2\sigma^{2}}} d\mu.
\label{eq:DsPhi_DsstarPhi_shapes}
\end{equation}
where $g(\mu,a,b)$ represents the parabola for each of the four components, listed in Table~\ref{tab:DsPhi_DsstarPhi_parabolas}. The resulting invariant mass distributions are shown in Fig.\ref{eq:DsPhi_DsstarPhi_shapes}.

This decay is unobserved, therefore the relative contribution to this decay in nature from the \emph{101} and \emph{010} helicty states is not known. The total PDF for this contribution is made by weighting the \piz and \Pgamma contributions by their corresponding branching fractions, and by assuming that the \emph{101} and \emph{010} helicty states contribute with equal magnitudes. This assumption is varied and included as a source of systematic uncertainty in the $\BF(\decay{\Bp}{\Dsp\phiz})$ determination.
\end{description}
 
%%%%%%%%%%%%%%%%%%%%%%%%%%%%%%%%%%%%%%%%%%%%%%%%%%%%%%%%%% 
\begin{table}[h]
   \centering
   \begin{tabular}{ c c c c c }
      \hline
      Missed particle   & Helicity  & $g(\mu,a,b)$                                      & $a$ (\mevcc) & $b$ (\mevcc)  \\
      \hline
      \piz              & 010       & $\left(\mu - \frac{a+b}{2}\right)^{2}$            & 5026.8       & 5124.8         \\
      \piz              & 101       & $-(\mu - a)(\mu-b)$                               & 5026.8       & 5124.8         \\
      \Pgamma           & 010       & $-(\mu - a)(\mu-b)$                               & 4936.4       & 5220.6         \\
      \Pgamma           & 101       & $(\mu - \frac{a+b}{2})^{2} +(\frac{a+b}{2})^{2} $ & 4936.4       & 5220.6         \\
      \hline
   \end{tabular}
   \caption{Parabolas.} 
   \label{tab:DsPhi_DsstarPhi_parabolas}  
\end{table}
%%%%%%%%%%%%%%%%%%%%%%%%%%%%%%%%%%%%%%%%%%%%%%%%%%%%%%%%%% 


%%%%%%%%%%%%%%%%%%%%%%%%%%%%%%%%%%%%%%%%%%%%%%%%%%%%%%%%%%
\begin{figure}[!h]
    \centering
    \includegraphics[width=0.80\textwidth]{figs/B2DsPhi/DsPhi_part_reco_Shapes.pdf}
    \caption{Partially reconstructed $\Dsp\phiz$ shapes.}
    \label{fig:B2DsPhi_DsPhi_partreco}   
\end{figure}
%%%%%%%%%%%%%%%%%%%%%%%%%%%%%%%%%%%%%%%%%%%%%%%%%%%%%%%%%%


{\color{Blue}
Additionally, the distribution of each shape in the helicity angle described in Section~\ref{subsec:Hel} is approximated to be $\cos^2(\theta)$ for the longitudinally polarized shapes (010) and $\sin^2(\theta)$ for the transversely (101). The fraction in each bin for 010 or 101 is taken from the appropriate integrals of these assumed distributions. 
}

\begin{description}
\item \textbf{\decay{\Bsb}{\Dsp\Km\Kstarz}:} this decay can form a background to \decay{\Bp}{\Dsp\phiz} decays when the soft pion from the \decay{\Kstarz}{\Kp\pim} decay is not reconstructed. The lower bound of the fit range is wide enough that a significant fraction of these decays are retained in the fitted data set. The $\Km\Kstarz$ is modelled as originating from the $a_1(1260)$ resonance. This resonance has a width of $250-600$ MeV~\cite{PDG2016}, allowing it to decay to $\Km\Kstarz$ even though it's pole mass is below the $\Km\Kstarz$ threshold. A PDF for this component is determined by reconstructing simulated \decay{\Bsb}{\Dsp\Km\Kstarz} decays through the identical reconstruction and selection steps as the signal. The \roofit class \texttt{RooKeysPDF} is used to create a kernel estimation of the partially reconstructed \Bp mass distribution for the candidates passing the selection. This is shown in Fig.~\ref{fig:B2DsPhi_part_reco_shapes_DsKKstar}. The fraction of these decays expected in each of the four $m(\Kp\Km)$ and $\cos\theta_{K}$ categories is determined from the simulations samples.
\end{description}

%%%%%%%%%%%%%%%%%%%%%%%%%%%%%%%%%%%%%%%%%%%%%%%%%%%%%%%%%%
\begin{figure}[!h]
    \centering
    \begin{subfigure}[t]{0.49\textwidth}
        \includegraphics[width=1.0\textwidth]{figs/B2DsPhi/Bs2Dsa1_4600_5900_Shape.pdf}
    \end{subfigure}
    \caption{Partially reconstructed mass PDFs determined from samples of \decay{\Bsb}{\Dsp\Km\Kstarz} simulations processed with the same reconstruction and selection as the signal decays. The \Bp meson mass is indicated by a vertical red line. The are below 4900\mevcc is not included in the fit range, but included for reference. The PDF colours follow the same convention used in the final fit plots shown in Fig.~\ref{fig:B2DsPhi_Signal_Fit}.}
    \label{fig:B2DsPhi_part_reco_shapes_DsKKstar}   
\end{figure}
%%%%%%%%%%%%%%%%%%%%%%%%%%%%%%%%%%%%%%%%%%%%%%%%%%%%%%%%%%

\begin{description}
\item \textbf{\decay{\Bsb}{\Dssp\Km\Kstarz}:} this decay similarly forms a background to the signal mode when both a low momentum neutral particle (\piz or \Pgamma) is not reconstructed in the decay of the \Dssp meson, in addition to the low momentum pion from the \Kstarz decay. {\color{Red} Add more words}
\end{description}

%%%%%%%%%%%%%%%%%%%%%%%%%%%%%%%%%%%%%%%%%%%%%%%%%%%%%%%%%%
\begin{figure}[!h]
    \centering
    \begin{subfigure}[t]{0.49\textwidth}
        \includegraphics[width=1.0\textwidth]{figs/B2DsPhi/Bs2DsstKKst_4600_5900_Shape.pdf}
    \end{subfigure}
    \caption{Partially reconstructed mass PDFs determined from samples of \decay{\Bsb}{\Dssp\Km\Kstarz} simulations processed with the same reconstruction and selection as the signal decays. The \Bp meson mass is indicated by a vertical red line. The are below 4900\mevcc is not included in the fit range, but included for reference. The PDF colours follow the same convention used in the final fit plots shown in Fig.~\ref{fig:B2DsPhi_Signal_Fit}.}
    \label{fig:B2DsPhi_part_reco_shapes_DsKKstar}   
\end{figure}
%%%%%%%%%%%%%%%%%%%%%%%%%%%%%%%%%%%%%%%%%%%%%%%%%%%%%%%%%%

\begin{description}
\item \textbf{\decay{\Bsb}{\Dsp\Dsm}, \decay{\Bzb}{\Dsp\Dm} and \decay{\Bsb}{\Dssp\Dsm}:} the decays of neutral \B mesons to two charged \D mesons can form a background to the signal decay if a low momentum pion is not reconstructed in one of the charm meson decays. In the case of \decay{\Bsb}{\Dssp\Dsm} decays an additional neutral particle is not reconstructed in the decay of the \Dssp meson.
The PDFs for these decays are determined from simulated decays that have been processed with the same reconstruction and selection steps as the signal. For \decay{\Bsb}{\Dsp\Dsm} and \decay{\Bsb}{\Dssp\Dsm} decays, the PDF is creating using a kernel estimation using the \roofit \texttt{RooKeysPDF} implementation, shown in Fig.\ref{fig:B2DsPhi_part_reco_shapes_DsDs}. Due to the similarities between \decay{\Bsb}{\Dsp\Dsm} and \decay{\Bzb}{\Dsp\Dm} decays, the PDF for the latter is created by using the kernel estimation for \decay{\Bsb}{\Dsp\Dsm} and shifting the shape down in mass by 40\mevcc to account for the kinematic differences. 

The branching fractions for these three decays are measured to be $\BF(\decay{\Bzb}{\Dsp\Dm}) = (7.2 \pm 0.8 ) \times 10^{-3}$, $\BF(\decay{\Bsb}{\Dsp\Dsm}) = (4.4 \pm 0.5 ) \times 10^{-3}$ and $\BF(\decay{\Bsb}{\Dssp\Dsm}) = (1.37 \pm 0.16 ) \%$. Therefore the relative contributions for these three processes are fixed using these branching fractions, estimates of the relative efficiencies for each mode and the production fraction of \Bsb mesons relative to \Bzb mesons, $f_{s}/f_{d}$. This helps to add stability to the fit.   
\end{description}


%%%%%%%%%%%%%%%%%%%%%%%%%%%%%%%%%%%%%%%%%%%%%%%%%%%%%%%%%%
\begin{figure}[!h]
    \centering
    \begin{subfigure}[t]{0.49\textwidth}
        \includegraphics[width=1.0\textwidth]{figs/B2DsPhi/Bs2DsDs_4600_5900_Shape.pdf}
        \caption{\decay{\Bsb}{\Dsp\Dsm} }
    \end{subfigure}
    \begin{subfigure}[t]{0.49\textwidth}
        \includegraphics[width=1.0\textwidth]{figs/B2DsPhi/Bs2DsstDs_4600_5900_Shape.pdf}
        \caption{\decay{\Bsb}{\Dssp\Dsm} }
    \end{subfigure}
    \caption{Partially reconstructed mass PDFs determined from samples of \decay{\Bsb}{\Dsp\Dsm} and \decay{\Bsb}{\Dssp\Dsm} simulations processed with the same reconstruction and selection as the signal decays. The \Bp meson mass is indicated by a vertical red line. The are below 4900\mevcc is not included in the fit range, but included for reference. The PDF colours follow the same convention used in the final fit plots shown in Fig.~\ref{fig:B2DsPhi_Signal_Fit}.}
    \label{fig:B2DsPhi_part_reco_shapes_DsDs}   
\end{figure}
%%%%%%%%%%%%%%%%%%%%%%%%%%%%%%%%%%%%%%%%%%%%%%%%%%%%%%%%%%

% %%%%%%%%%%%%%%%%%%%%%%%%%%%%%%%%%%%%%%%%%%%%%%%%%%%%%%%%%%
% \begin{figure}[!h]
%     \centering
%     \begin{subfigure}[t]{0.49\textwidth}
%         \includegraphics[width=1.0\textwidth]{figs/B2DsPhi/Bs2Dsa1_4600_5900_Shape.pdf}
%         \caption{\decay{\Bsb}{\Dsp\Km\Kstarz} }
%     \end{subfigure}
%     \begin{subfigure}[t]{0.49\textwidth}
%         \includegraphics[width=1.0\textwidth]{figs/B2DsPhi/Bs2DsstKKst_4600_5900_Shape.pdf}
%         \caption{\decay{\Bsb}{\Dsp\Km\Kstarz} }
%     \end{subfigure}
%     \begin{subfigure}[t]{0.49\textwidth}
%         \includegraphics[width=1.0\textwidth]{figs/B2DsPhi/Bs2DsDs_4600_5900_Shape.pdf}
%         \caption{\decay{\Bsb}{\Dsp\Dsm} }
%     \end{subfigure}
%     \begin{subfigure}[t]{0.49\textwidth}
%         \includegraphics[width=1.0\textwidth]{figs/B2DsPhi/Bs2DsstDs_4600_5900_Shape.pdf}
%         \caption{\decay{\Bsb}{\Dssp\Dsm} }
%     \end{subfigure}
%     \caption{Partially reconstructed mass PDFs determined from samples of simulations decays processed with the same reconstruction and selection as the signal decays. The \Bp meson mass is indicated by a vertical red line. The are below 4900\mevcc is not included in the fit range, but included for reference. The PDF colours follow the same convention used in the final fit plots shown in Fig.~\ref{fig:B2DsPhi_Signal_Fit}.}
%     \label{fig:B2DsPhi_part_reco_shapes}   
% \end{figure}
% %%%%%%%%%%%%%%%%%%%%%%%%%%%%%%%%%%%%%%%%%%%%%%%%%%%%%%%%%%


To account for difference between the simulation and data samples all PDFs determined using the kernel estimation method are convolved with an additional Gaussian. The mean position of this Gaussian is given by $\delta$, the same offset used for the analytically described partially decays. The width of the Gaussian is increased to account for the difference in resolution between simulation and data. 


\subsection{Combinatorial  backgrounds}
\label{sec:B2DsPhi_combcomps}

Combinations of unrelated tracks form a background to the signal decays and extend across the entire fitted \Bp meson mass range. 
This component is parametrised with a decaying exponential function 
\begin{equation}
f(m|c) = e^{-m\times c},
\end{equation}
where the parameter $c$ controls the slope and $m$ is the observable \Bp meson invariant mass. 
The combinatorial background is the dominant background under the signal decays, therefore it is important to accurately extrapolate distribution from the high and low mass ranges to the signal range. 

The yields of combinatorial background is allowed to vary freely in each of the separate simultaneous fit categories. To improve the stability of the fit, the single slope parameter, $c$, is shared between all of the categories. In categories with lower statistics, for example the \decay{\Dsp}{\Kp\pim\pip} decay mode, it would be possible for the partially reconstructed backgrounds to be incorrectly assigned to the combinatorial component. This can happen if the exponential slope for that mode gets to large, causing it to `kick up' at low invariant masses. This may bias the signal yield in this mode. Fixing the slope parameter to be the same between the different \Dsp modes and between the signal and normalisation channel allows the lower statistics modes to benefit from those with higher statistics.

\section{Free and constrained parameters}

The fit to \decay{\Bp}{\Dsp\phiz} and \decay{\Bp}{\Dsp\Dzb} decays contains a total of 54 free parameters. 
These are broadly divided into four different categories detailed here.

%These are listed in Tab.~\ref{tab:B2DsPhi_free_variables}. 

\subsection{Parameter of interest}

The parameter of interest (POI) is the branching fraction for \decay{\Bp}{\Dsp\phiz} decays. This is determined directly in the fit to the signal and normalisation decays. The same single branching fraction is used for all of the \Dsp decays modes.
The yield of signal candidates is calculated using the equation 
\begin{equation}
N(\decay{\Bp}{\Dsp\phiz}) = k_{\Dsp} \times \BF(\decay{\Bp}{\Dsp\phiz}) \times N(\decay{\Bp}{\Dsp\Dzb}),
\end{equation}
where $k_{\Dsp}$ is a constant defined for each \Dsp decay mode
\begin{equation}
k_{\Dsp} = \frac{\epsilon(\decay{\Bp}{\Dsp\Dzb})}{\epsilon(\decay{\Bp}{\Dsp\phiz})} \times  \frac{\BF(\decay{\phiz}{\Kp\Km})}{\BF(\decay{\Bp}{\Dsp\Dzb})\BF(\decay{\Dzb}{\Kp\Km})}.
\end{equation}
Here, the efficiencies $\epsilon$ are calculated for the specific \Dsp decay mode in question and external measurements are used for the additional branching fractions.
The yield $N(\decay{\Bp}{\Dsp\phiz})$ is the total signal yield in all four $m(\Kp\Km)$ and $\cos\theta_{K}$ categories. The yields in each category are constrained to be in the ratios previously listed in Table~\ref{tab:signal_ratios}, \ie the yield in the $|m(\Kp\Km)<10\mevcc$ and $|\cos\theta_{K}|>0.4$ category is $0.82\times N(\decay{\Bp}{\Dsp\phiz})$.



\subsection{Shape parameters}
As detailed in Sec~\ref{sec:B2DsPhi_fitcomponents}, the fit included eight parameters governing the shapes of various PDFs.
\begin{itemize}
\item The combinatorial background PDF is controlled by a single slope parameter $c$ for all categories.
\item The mean \Bp mass for the signal and normalisation mode is free to vary in the fit. The same value is used for all \decay{\Bp}{\Dsp\phiz} and \decay{\Bp}{\Dsp\Dzb} PDFs.
\item {\color{Red}Offset $\delta$}
\item The relative heights of the two peaks in partially reconstructed \decay{\Bp}{\Dssp\phiz}, \decay{\Bp}{\Dssp\Dzb} and \decay{\Bp}{\Dsp\Dstarzb} decays, $\xi$, is allowed to vary. The same value is used for all modes.
\item The smaller CB width parameter, $\sigma_{1}$, is allowed to vary for each of the \Dsp decay modes, leading to four more free parameters. 
\end{itemize}

\subsection{Yields}

The fit model contains a total of 36 free parameters that are yields.
\begin{itemize}
\item The total yield of normalisation decays in each \Dsp decay mode is left floating in the fit. These variables are the sum of the yields in the two helicity categories, $|\cos\theta_{K}| < 0.4$ and $|\cos\theta_{K}| > 0.4$. This results in four free parameters.
\item The yield of certain partially reconstructed decays in the normalisation channel is left floating for each \Dsp decay mode independently. This yield is defined to be $N(\decay{\Bp}{\Dsp\Dstarzb})+ N(\decay{\Bp}{\Dssp\Dzb})$. The yield of the doubly excited \decay{\Bp}{\Dssp\Dzb} is not included in this total. This results in four free parameters.
\item The yield of some partially reconstructed decays in the signal mode is left free in the fit. This is defined to be $N(\Dssp\phiz) + N(\Dsp\Km\Kstarz) + N(\Dssp\Km\Kstarz)$. This leads to four free parameters. 
\item All yields of combinatorial backgrounds are left free in the fit. This results in 24 free parameters.
\end{itemize}



\subsection{Fractions}




The majority of the partially reconstructed backgrounds do not have unconstrained yields in each of the 24 simultaneous fit categories. 
Instead, a single partially reconstructed background yield per \Dsp decay mode is left free in the fit. The relative contributions of other partially backgrounds are determined by floating fraction parameters multiplied by this yield. 
This allows the different background contributions to vary relative to one another, but keeps the ratios of the relative contributions the same across the different \Dsp decay modes.

There are four free parameters controlling the backgrounds contributions in the signal mode:
\begin{enumerate}
\item Ratio of yields of \decay{\Bp}{\Dsp\Kp\Km} decays to \decay{\Bp}{\Dsp\Dzb} decays
\begin{equation}
\frac{N(\decay{\Bp}{\Dsp\Kp\Km})}{N(\decay{\Bp}{\Dsp\Dzb})}. 
\end{equation}
\item The fraction of \decay{\Bsb}{\Dsp\Km\Kstarz} decays to the total of $\decay{\Bsb}{D_{s}^{(*)+}\Km\Kstarz}$ decays
\begin{equation}
\frac{N(\decay{\Bsb}{\Dsp\Km\Kstarz})}{N(\decay{\Bsb}{\Dssp\Km\Kstarz})+N(\decay{\Bsb}{\Dsp\Km\Kstarz})}.
\end{equation}

\item The fraction of \decay{\Bp}{\Dssp\phiz} decays in the total of these and $\decay{\Bsb}{D_{s}^{(*)+}\Km\Kstarz}$ decays
\begin{equation}
\frac{N(\decay{\Bp}{\Dssp\phiz})}{N(\decay{\Bp}{\Dssp\phiz})+N(\decay{\Bsb}{\Dssp\Km\Kstarz})+N(\decay{\Bsb}{\Dsp\Km\Kstarz})}   
\end{equation}
\item The ratio of yields of the partially reconstructed \decay{\Bs}{\Dsp\Dsm}, \decay{\Bs}{\Dsp\Dssm} and \decay{\Bz}{\Dsp\Dm} decays to $\decay{\Bp}{\Dssp\phiz}$ and $\decay{\Bsb}{D_{s}^{(*)+}\Km\Kstarz}$ decays
\begin{equation}
\frac{N(\decay{\Bs}{\Dsp\Dsm})+ N(\decay{\Bs}{\Dsp\Dssm}) + N(\decay{\Bz}{\Dsp\Dm})}{N(\decay{\Bp}{\Dssp\phiz})+N(\decay{\Bsb}{\Dssp\Km\Kstarz})+N(\decay{\Bsb}{\Dsp\Km\Kstarz})}   
\end{equation}


\end{enumerate}


Another five free parameters control the fractions of events in different categories for the normalisation mode:

\begin{enumerate}
\item Fraction of partially reconstructed \decay{\Bp}{\Dssp\Dzb} decays in the total of \decay{\Bp}{\Dssp\Dzb} and \decay{\Bp}{\Dsp\Dstarzb} decays in the $|\cos\theta_{K}|>0.4$ category
\begin{equation}
\left(\frac{N(\decay{\Bp}{\Dssp\Dzb})}{N(\decay{\Bp}{\Dssp\Dzb})+N(\decay{\Bp}{\Dsp\Dstarzb})}\right)_{|\cos\theta_{K}|>0.4}   
\end{equation}
\item Fraction of partially reconstructed \decay{\Bp}{\Dssp\Dzb} decays in the total of \decay{\Bp}{\Dssp\Dzb} and \decay{\Bp}{\Dsp\Dstarzb} decays in the alternative $|\cos\theta_{K}|<0.4$ category
\begin{equation}
\left(\frac{N(\decay{\Bp}{\Dssp\Dzb})}{N(\decay{\Bp}{\Dssp\Dzb})+N(\decay{\Bp}{\Dsp\Dstarzb})}\right)_{|\cos\theta_{K}|<0.4}   
\end{equation}
\item Fraction of partially reconstructed backgrounds in the $|\cos\theta_{K}|>0.4$ category
\begin{equation}
\frac{(N(\decay{\Bp}{\Dssp\Dzb})+N(\decay{\Bp}{\Dsp\Dstarzb}))_{|\cos\theta_{K}|>0.4}}{(N(\decay{\Bp}{\Dssp\Dzb})+N(\decay{\Bp}{\Dsp\Dstarzb}))_{\text{Total}}}   
\end{equation}
\item The ratio of \decay{\Bp}{\Dssp\Dstarzb} decays to the total of \decay{\Bp}{\Dssp\Dzb} and \decay{\Bp}{\Dsp\Dstarzb} decays
\begin{equation}\frac{N(\decay{\Bp}{\Dssp\Dstarzb})}{N(\decay{\Bp}{\Dssp\Dzb})+N(\decay{\Bp}{\Dsp\Dstarzb})}   
\end{equation}
\item Fraction of fully reconstructed normalisation decays in the $|\cos\theta_{K}|>0.4$ category
\begin{equation}
\frac{N(\decay{\Bp}{\Dsp\Dzb})_{|\cos\theta_{K}|>0.4}}{N(\decay{\Bp}{\Dsp\Dzb})_{\text{Total}}}.   
\end{equation}
This parameter is included as a cross check. The normalisation decays have a flat distribution in $\cos\theta_{K}$ so the paremeter would be expected to be around $0.6$. 
\end{enumerate}



% \begin{longtable}{ l l }
% \hline
% Type & Variable  \\
% \hline
% POI                  & Branching fraction $\BF(\decay{\Bp}{\Dsp\phiz})$                              \\
% \hline 
% Shape                & Combinatorial slope $c$                                                           \\ 
%                      & Mean \Bp mass (\mevcc)                                                        \\  
%                      & Mass offset $\delta$ (\mevcc)                         \\  
%                      & Relative heights $\xi$                               \\
%                      & $\sigma_{1}$ for \decay{\Dsp}{\Kp\Km\pip}  (\mevcc)                           \\  
%                      & $\sigma_{1}$ for \decay{\Dsp}{\Kp\pim\pip} (\mevcc)                           \\  
%                      & $\sigma_{1}$ for \decay{\Dsp}{\phiz\pip}  (\mevcc)                            \\  
%                      & $\sigma_{1}$ for \decay{\Dsp}{\pip\pim\pip}  (\mevcc)                         \\  
% \hline  
% Yields               & $N(\decay{\Bp}{(\decay{\Dsp}{\Kp\Km\pip}  )\Dzb})$                 \\  
%                      & $N(\decay{\Bp}{(\decay{\Dsp}{\Kp\pim\pip} )\Dzb})$                 \\  
%                      & $N(\decay{\Bp}{(\decay{\Dsp}{\phiz\pip}   )\Dzb})$                 \\  
%                      & $N(\decay{\Bp}{(\decay{\Dsp}{\pip\pim\pip})\Dzb})$                 \\
%                      & $N(\Dssp\Dzb + \Dsp\Dstarzb)$ in \decay{\Dsp}{\Kp\Km\pip}               \\  
%                      & $N(\Dssp\Dzb + \Dsp\Dstarzb)$ in \decay{\Dsp}{\Kp\pim\pip}              \\  
%                      & $N(\Dssp\Dzb + \Dsp\Dstarzb)$ in \decay{\Dsp}{\phiz\pip}                \\  
%                      & $N(\Dssp\Dzb + \Dsp\Dstarzb)$ in \decay{\Dsp}{\pip\pim\pip}             \\  
%                      & $N(\Dssp\phiz + D_{s}^{(*)+}\Km\Kstarz)$ in \decay{\Dsp}{\Kp\Km\pip}       \\  
%                      & $N(\Dssp\phiz + D_{s}^{(*)+}\Km\Kstarz)$ in \decay{\Dsp}{\Kp\pim\pip}      \\  
%                      & $N(\Dssp\phiz + D_{s}^{(*)+}\Km\Kstarz)$ in \decay{\Dsp}{\phiz\pip}        \\  
%                      & $N(\Dssp\phiz + D_{s}^{(*)+}\Km\Kstarz)$ in \decay{\Dsp}{\Kp\pim\pip}      \\

%                      & $N_{\text{comb}}((\decay{\Dsp}{\Kp\Km\pip})\Dzb)$  in H1             \\  
%                      & $N_{\text{comb}}(\Dsp\Dzb)$ \decay{\Dsp}{\Kp\Km\pip} in H2             \\  
%                      & $N_{\text{comb}}(\Dsp\Dzb)$ \decay{\Dsp}{\Kp\pim\pip} in H1            \\  
%                      & $N_{\text{comb}}(\Dsp\Dzb)$ \decay{\Dsp}{\Kp\pim\pip} in H2            \\  
%                      & $N_{\text{comb}}(\Dsp\Dzb)$ \decay{\Dsp}{\phiz\pip} in H1              \\  
%                      & $N_{\text{comb}}(\Dsp\Dzb)$ \decay{\Dsp}{\phiz\pip} in H2              \\  
%                      & $N_{\text{comb}}(\Dsp\Dzb)$ \decay{\Dsp}{\pip\pim\pip} in H1           \\  
%                      & $N_{\text{comb}}(\Dsp\Dzb)$ \decay{\Dsp}{\pip\pim\pip} in H2           \\    
%                      & $N_{\text{comb}}(\Dsp\phiz)$ \decay{\Dsp}{\Kp\Km\pip} in H1            \\  
%                      & $N_{\text{comb}}(\Dsp\phiz)$ \decay{\Dsp}{\Kp\Km\pip} in H2            \\  
%                      & $N_{\text{comb}}(\Dsp\phiz)$ \decay{\Dsp}{\Kp\pim\pip} in H1           \\  
%                      & $N_{\text{comb}}(\Dsp\phiz)$ \decay{\Dsp}{\Kp\pim\pip} in H2           \\  
%                      & $N_{\text{comb}}(\Dsp\phiz)$ \decay{\Dsp}{\phiz\pip} in H1             \\  
%                      & $N_{\text{comb}}(\Dsp\phiz)$ \decay{\Dsp}{\phiz\pip} in H2             \\  
%                      & $N_{\text{comb}}(\Dsp\phiz)$ \decay{\Dsp}{\pip\pim\pip} in H1          \\  
%                      & $N_{\text{comb}}(\Dsp\phiz)$ \decay{\Dsp}{\pip\pim\pip} in H2          \\
%                      & $N_{\text{comb}}(\Dsp\phiz)$ \phiz-sideband \decay{\Dsp}{\Kp\Km\pip} in H1   \\  
%                      & $N_{\text{comb}}(\Dsp\phiz)$ \phiz-sideband \decay{\Dsp}{\Kp\Km\pip} in H2   \\  
%                      & $N_{\text{comb}}(\Dsp\phiz)$ \phiz-sideband \decay{\Dsp}{\Kp\pim\pip} in H1  \\  
%                      & $N_{\text{comb}}(\Dsp\phiz)$ \phiz-sideband \decay{\Dsp}{\Kp\pim\pip} in H2  \\  
%                      & $N_{\text{comb}}(\Dsp\phiz)$ \phiz-sideband \decay{\Dsp}{\phiz\pip} in H1    \\  
%                      & $N_{\text{comb}}(\Dsp\phiz)$ \phiz-sideband \decay{\Dsp}{\phiz\pip} in H2    \\  
%                      & $N_{\text{comb}}(\Dsp\phiz)$ \phiz-sideband \decay{\Dsp}{\pip\pim\pip} in H1 \\  
%                      & $N_{\text{comb}}(\Dsp\phiz)$ \phiz-sideband \decay{\Dsp}{\pip\pim\pip} in H2 \\ 
% \hline 
% Fractions            & Ratio of \Dsp\Kp\Km to \Dsp\Dzb                                        \\  
%                      & Fraction of \Dsp\Km\Kstarz in (\Dssp\Km\Kstarz+\Dsp\Km\Kstarz)                    \\ 
%                      & Fraction of $\Dssp\phiz$ in ($D_{s}^{(*)+}\Km\Kstarz$ + $\Dssp\phiz$)                   \\  
%                      & Ratio of $\Dsp D_{(s)}^{(*)-}$ to ($\Dssp\phiz$ + $D_{s}^{(*)+}\Km\Kstarz$)           \\  

%                      & Fraction of \Dssp\Dzb in (\Dssp\Dzb+\Dsp\Dstarzb) in H1         \\  
%                      & Fraction of \Dssp\Dzb in (\Dssp\Dzb+\Dsp\Dstarzb) in H2         \\ 
%                      & Fraction of normalisation part reco in H1                       \\  
%                      & Fraction of normalisation peak in H1                            \\ 
%                      & Ratio of \Dssp\Dstarzb to (\Dssp\Dzb + \Dsp\Dstarzb)                          \\ 
% \hline
% \caption{The fit result with final values of all floating variables used in the fit model. Here H1 and H2 represent the $|\cos\theta_{K}|>0.4$ and $|\cos\theta_{K}|<0.4$ categories respectively.} 
% \label{tab:B2DsPhi_free_variables} 
% \end{longtable}

\section{Fit validation}
\label{sec:B2DsPhi_fitstrategy}

The simultaneous fitting framework is validated by generating pseudo-experiments (also referred to as \emph{toys}). The total fit model PDF is randomly sampled to create a simulation sample with the same number of candidates as the nominal fit. These are then fitted using the same fit model, determining the best estimate and uncertainty of each parameter. The parameter values used to generate the pseudo-experiments are chosen to be the final parameter values as determined in a fit to data. This is known as the plug-in method~\cite{plugin}.
The fitted value and associated error is used to determine the pull of each parameter of interest. As the errors are determined asymmetrically using \minos the is defined conditionally to incorporate the appropriate error 
\begin{equation}
  g_{\text{pull}} = \left \{
  \begin{aligned}
    &\frac{x_{\text{gen}} - x_{\text{fit}} }{\sigma_{+}}, && \text{if}\ x_{\text{fit}} < x_{\text{gen}}\\
    &\frac{x_{\text{fit}} - x_{\text{gen}} }{\sigma_{-}}, && \text{otherwise},
  \end{aligned} \right.
\end{equation} 
where $x_{\text{fit}}$ and $x_{\text{gen}}$ are the fitted and generated values of the variable, and $\sigma_{+}$ and $\sigma_{-}$ are the high and low asymmetric errors.

The distributions of the values, errors and pulls for the yields of the normalisation decay in each of the \Dsp decays are shown in Fig.~\ref{fig:B2DsPhi_Pulls_normalisation}. The mean and widths are determined using simple fits to the pull distributions. The results and PDFs for these fits are overlaid on the distributions. The normalisation yield means and widths are found to be within $2\sigma$ of zero and one respectively.


%%%%%%%%%%%%%%%%%%%%%%%%%%%%%%%%%%%%%%%%%%%%%%%%%%%%%%%%%%
\begin{figure}[!h]
   \centering
   \begin{subfigure}[t]{1.0\textwidth}
      \includegraphics[width=0.32\textwidth]{figs/B2DsPhi/Plots_DsKK_Value_yield_peak_DsD0_Ds2PhiPi_toy_both_DsBDTbin1_PhiBDTbin1_both_both.pdf}
      \includegraphics[width=0.32\textwidth]{figs/B2DsPhi/Plots_DsKK_Error_yield_peak_DsD0_Ds2PhiPi_toy_both_DsBDTbin1_PhiBDTbin1_both_both.pdf}
      \includegraphics[width=0.32\textwidth]{figs/B2DsPhi/Plots_DsKK_Pull_yield_peak_DsD0_Ds2PhiPi_toy_both_DsBDTbin1_PhiBDTbin1_both_both.pdf}
      \caption{\decay{\Dsp}{\phiz\pip}}
   \end{subfigure}\\
   \begin{subfigure}[t]{1.0\textwidth}
      \includegraphics[width=0.32\textwidth]{figs/B2DsPhi/Plots_DsKK_Value_yield_peak_DsD0_Ds2KKPi_toy_both_DsBDTbin1_PhiBDTbin1_both_both.pdf}
      \includegraphics[width=0.32\textwidth]{figs/B2DsPhi/Plots_DsKK_Error_yield_peak_DsD0_Ds2KKPi_toy_both_DsBDTbin1_PhiBDTbin1_both_both.pdf}
      \includegraphics[width=0.32\textwidth]{figs/B2DsPhi/Plots_DsKK_Pull_yield_peak_DsD0_Ds2KKPi_toy_both_DsBDTbin1_PhiBDTbin1_both_both.pdf}
      \caption{\decay{\Dsp}{\Kp\Km\pip}}
   \end{subfigure}\\
   \begin{subfigure}[t]{1.0\textwidth}
      \includegraphics[width=0.32\textwidth]{figs/B2DsPhi/Plots_DsKK_Value_yield_peak_DsD0_Ds2PiPiPi_toy_both_DsBDTbin1_PhiBDTbin1_both_both.pdf}
      \includegraphics[width=0.32\textwidth]{figs/B2DsPhi/Plots_DsKK_Error_yield_peak_DsD0_Ds2PiPiPi_toy_both_DsBDTbin1_PhiBDTbin1_both_both.pdf}
      \includegraphics[width=0.32\textwidth]{figs/B2DsPhi/Plots_DsKK_Pull_yield_peak_DsD0_Ds2PiPiPi_toy_both_DsBDTbin1_PhiBDTbin1_both_both.pdf}
      \caption{\decay{\Dsp}{\pip\pim\pip}}
   \end{subfigure}\\
   \begin{subfigure}[t]{1.0\textwidth}
      \includegraphics[width=0.32\textwidth]{figs/B2DsPhi/Plots_DsKK_Value_yield_peak_DsD0_Ds2KPiPi_toy_both_DsBDTbin1_PhiBDTbin1_both_both.pdf}
      \includegraphics[width=0.32\textwidth]{figs/B2DsPhi/Plots_DsKK_Error_yield_peak_DsD0_Ds2KPiPi_toy_both_DsBDTbin1_PhiBDTbin1_both_both.pdf}
      \includegraphics[width=0.32\textwidth]{figs/B2DsPhi/Plots_DsKK_Pull_yield_peak_DsD0_Ds2KPiPi_toy_both_DsBDTbin1_PhiBDTbin1_both_both.pdf}
      \caption{\decay{\Dsp}{\Kp\pim\pip}}
   \end{subfigure}

   \caption{Pulls normalisation}
   \label{fig:B2DsPhi_Pulls_normalisation}
\end{figure}
%%%%%%%%%%%%%%%%%%%%%%%%%%%%%%%%%%%%%%%%%%%%%%%%%%%%%%%%%%


Similarly, the distributions of the signal yield, error and pull for each of the different \Dsp decay modes are shown in Fig.~\ref{fig:B2DsPhi_Pulls_signal}. The pull means and widths are all within $2\sigma$ of zero and one respectively.


%%%%%%%%%%%%%%%%%%%%%%%%%%%%%%%%%%%%%%%%%%%%%%%%%%%%%%%%%%
\begin{figure}[!h]
   \centering
   \begin{subfigure}[t]{1.0\textwidth}
      \includegraphics[width=0.32\textwidth]{figs/B2DsPhi/Plots_DsKK_Value_yield_peak_total_DsPhi_Ds2PhiPi_toy_both_DsBDTbin1_PhiBDTbin1_both_both.pdf}
      \includegraphics[width=0.32\textwidth]{figs/B2DsPhi/Plots_DsKK_Error_yield_peak_total_DsPhi_Ds2PhiPi_toy_both_DsBDTbin1_PhiBDTbin1_both_both.pdf}
      \includegraphics[width=0.32\textwidth]{figs/B2DsPhi/Plots_DsKK_Pull_yield_peak_total_DsPhi_Ds2PhiPi_toy_both_DsBDTbin1_PhiBDTbin1_both_both.pdf}
      \caption{\decay{\Dsp}{\phiz\pip}}
   \end{subfigure}\\
   \begin{subfigure}[t]{1.0\textwidth}
      \includegraphics[width=0.32\textwidth]{figs/B2DsPhi/Plots_DsKK_Value_yield_peak_total_DsPhi_Ds2KKPi_toy_both_DsBDTbin1_PhiBDTbin1_both_both.pdf}
      \includegraphics[width=0.32\textwidth]{figs/B2DsPhi/Plots_DsKK_Error_yield_peak_total_DsPhi_Ds2KKPi_toy_both_DsBDTbin1_PhiBDTbin1_both_both.pdf}
      \includegraphics[width=0.32\textwidth]{figs/B2DsPhi/Plots_DsKK_Pull_yield_peak_total_DsPhi_Ds2KKPi_toy_both_DsBDTbin1_PhiBDTbin1_both_both.pdf}
      \caption{\decay{\Dsp}{\Kp\Km\pip}}
   \end{subfigure}\\
   \begin{subfigure}[t]{1.0\textwidth}
      \includegraphics[width=0.32\textwidth]{figs/B2DsPhi/Plots_DsKK_Value_yield_peak_total_DsPhi_Ds2PiPiPi_toy_both_DsBDTbin1_PhiBDTbin1_both_both.pdf}
      \includegraphics[width=0.32\textwidth]{figs/B2DsPhi/Plots_DsKK_Error_yield_peak_total_DsPhi_Ds2PiPiPi_toy_both_DsBDTbin1_PhiBDTbin1_both_both.pdf}
      \includegraphics[width=0.32\textwidth]{figs/B2DsPhi/Plots_DsKK_Pull_yield_peak_total_DsPhi_Ds2PiPiPi_toy_both_DsBDTbin1_PhiBDTbin1_both_both.pdf}
      \caption{\decay{\Dsp}{\pip\pim\pip}}
   \end{subfigure}\\
   \begin{subfigure}[t]{1.0\textwidth}
      \includegraphics[width=0.32\textwidth]{figs/B2DsPhi/Plots_DsKK_Value_yield_peak_total_DsPhi_Ds2KPiPi_toy_both_DsBDTbin1_PhiBDTbin1_both_both.pdf}
      \includegraphics[width=0.32\textwidth]{figs/B2DsPhi/Plots_DsKK_Error_yield_peak_total_DsPhi_Ds2KPiPi_toy_both_DsBDTbin1_PhiBDTbin1_both_both.pdf}
      \includegraphics[width=0.32\textwidth]{figs/B2DsPhi/Plots_DsKK_Pull_yield_peak_total_DsPhi_Ds2KPiPi_toy_both_DsBDTbin1_PhiBDTbin1_both_both.pdf}
      \caption{\decay{\Dsp}{\Kp\pim\pip}}
   \end{subfigure}

   \caption{Pulls signal}
   \label{fig:B2DsPhi_Pulls_signal}
\end{figure}
%%%%%%%%%%%%%%%%%%%%%%%%%%%%%%%%%%%%%%%%%%%%%%%%%%%%%%%%%%

These distributions were generated using pseudo-experiments in which the yields for each \Dsp mode were free variables rather than being determined using a single free branching fraction parameter.

The model is also studied using a single branching fraction $\BF(\decay{\Bp}{\Dsp\phiz})$ that determines the yields in each signal mode. This parameter is calculated directly in the fit so it can be assessed for any possible bias. The distribution of the measured values, uncertainties and pulls are shown in Fig.~\ref{fig:B2DsPhi_Pulls_signal}.


%%%%%%%%%%%%%%%%%%%%%%%%%%%%%%%%%%%%%%%%%%%%%%%%%%%%%%%%%%
\begin{figure}[!h]
   \centering
   \begin{subfigure}[t]{1.0\textwidth}
      \includegraphics[width=0.32\textwidth]{figs/B2DsPhi/Plots_DsKK_Error_Branching_fraction.pdf}
      \includegraphics[width=0.32\textwidth]{figs/B2DsPhi/Plots_DsKK_Error_Branching_fraction.pdf}
      \includegraphics[width=0.32\textwidth]{figs/B2DsPhi/Plots_DsKK_Pull_Branching_fraction.pdf}
   \end{subfigure}
   \caption{Pulls branching fraction}
   \label{fig:B2DsPhi_Pulls_signal}
\end{figure}
%%%%%%%%%%%%%%%%%%%%%%%%%%%%%%%%%%%%%%%%%%%%%%%%%%%%%%%%%%
The pull mean and width are both within $2\sigma$ of zero and one respectively. 
As no significant biases are observed in the pulls of any of the yields or branching fraction no corrections are applied to the values determined fit.


\section{Normalisation and signal fits}

The result of the simultaneous fit to the signal and normalisation decays is shown in Figs.~\ref{fig:B2DsPhi_Norm_Fit} and \ref{fig:B2DsPhi_Signal_Fit}. These figures show the distribution of candidates passing all selection requirements along with the total PDF model resulting from the minimisation of the log-likelihood. The four \Dsp decay mode categories have been merged into a single distribution. The fits in each different \Ds decay mode category can be found in Appendix~\ref{ch:appendix_fits}.



%%%%%%%%%%%%%%%%%%%%%%%%%%%%%%%%%%%%%%%%%%%%%%%%%%%%%%%%%%
\begin{figure}[!h]
    \centering
    \begin{subfigure}[t]{1.0\textwidth}
        \includegraphics[width=1.0\textwidth]{figs/Appendix_FitCategories/canvas_DsD0_merged_both_summed_splitHel_splitKKPi_s21_s21r1_s24_s26.pdf}
    \end{subfigure}
    \caption{Invariant mass fits to \decay{\Bp}{\Dsp\Dzb} candidates}
    \label{fig:B2DsPhi_Norm_Fit}
\end{figure}
%%%%%%%%%%%%%%%%%%%%%%%%%%%%%%%%%%%%%%%%%%%%%%%%%%%%%%%%%%

The high purity of the normalisation mode reconstruction can be seen in Fig.~\ref{fig:B2DsPhi_Norm_Fit}, as the contribution from the combinatorial background shape is very small. Additionally, the double-peaked structure of the partially reconstructed \decay{\Bp}{\Dssp\Dzb} and \decay{\Bp}{\Dsp\Dstarzb} decays is clearly visible. 
%The decays are split into two $\cos\theta_{K}$ categories, 


%%%%%%%%%%%%%%%%%%%%%%%%%%%%%%%%%%%%%%%%%%%%%%%%%%%%%%%%%%
\begin{figure}[!h]
    \centering
    \begin{subfigure}[t]{1.0\textwidth}
        \includegraphics[width=1.0\textwidth]{figs/B2DsPhi/Fig4a.eps}
    \end{subfigure}
    \begin{subfigure}[t]{1.0\textwidth}
        \includegraphics[width=1.0\textwidth]{figs/B2DsPhi/Fig4b.eps}
    \end{subfigure}
    \caption{Invariant mass fits to \decay{\Bp}{\Dsp\phiz} candidates}
    \label{fig:B2DsPhi_Signal_Fit}
\end{figure}
%%%%%%%%%%%%%%%%%%%%%%%%%%%%%%%%%%%%%%%%%%%%%%%%%%%%%%%%%%

The fit to the signal mode is shown in Fig.~\ref{fig:B2DsPhi_Signal_Fit}. This figure includes the \decay{\Bp}{\Dsp\phiz} candidates in both the $|m(\Kp\Km)| < 10\mevcc$ (\phiz-region) and $10< |m(\Kp\Km)| < 40\mevcc$ (\phiz-sideband) categories. These are further split into the two $\cos\theta_{K}$ categories in an analogous way to the normalisation decays.



The final values of the parameters as determined by the fit are tabulated in Tab.~\ref{tab:B2DsPhi_free_variables}, including the branching fraction $\BF(\decay{\Bp}{\Dsp\phiz})$. This parameter includes a correction to account for the efficiencies of the signal and normalisation modes as discussed in Sec.~\ref{sec:B2DsPhi_effcorrections}. 

\begin{longtable}{ l l c }
\hline
Type       & Parameter                                                                    & Fit result\\
\hline
POI        & Branching fraction $\BF(\decay{\Bp}{\Dsp\phiz}) (\times 10^{-7})$            &   $1.17^{+ 1.56}_{-1.37} $\\
\hline
Shape      & Combinatorial slope $c$                                                      &   $(-3.3^{+0.4}_{-0.4})\times10^{-3}$\\
           & Mean \Bp mass (\mevcc)                                                       &   $5279.10^{+0.17}_{-0.17} $\\
           & Mass offset $\delta$ (\mevcc)                                                &   $-1.975^{+0.362}_{-0.362} $\\
           & Relative heights $\xi$                                                       &   $0.68^{+0.05}_{-0.05} $\\
           & $\sigma_{1}$ for \decay{\Dsp}{\Kp\Km\pip}  (\mevcc)                          &   $7.61^{+0.18}_{-0.18} $\\
           & $\sigma_{1}$ for \decay{\Dsp}{\Kp\pim\pip} (\mevcc)                          &   $8.77^{+0.59}_{-0.55} $\\
           & $\sigma_{1}$ for \decay{\Dsp}{\phiz\pip}  (\mevcc)                           &   $7.53^{+0.22}_{-0.21} $\\
           & $\sigma_{1}$ for \decay{\Dsp}{\pip\pim\pip}  (\mevcc)                        &   $8.30^{+0.36}_{-0.34} $\\
\hline
Yields     & $N(\decay{\Bp}{(\decay{\Dsp}{\Kp\Km\pip}  )\Dzb})$                           &   $1324^{+37}_{-36} $\\
           & $N(\decay{\Bp}{(\decay{\Dsp}{\Kp\pim\pip} )\Dzb})$                           &   $182^{+14}_{-13} $\\
           & $N(\decay{\Bp}{(\decay{\Dsp}{\phiz\pip}   )\Dzb})$                           &   $801^{+29}_{-28} $\\
           & $N(\decay{\Bp}{(\decay{\Dsp}{\pip\pim\pip})\Dzb})$                           &   $369^{+20}_{-19} $\\
           & $N(\Dssp\Dzb + \Dsp\Dstarzb)$ in \decay{\Dsp}{\Kp\Km\pip}                    &   $2827^{+52}_{-51} $\\
           & $N(\Dssp\Dzb + \Dsp\Dstarzb)$ in \decay{\Dsp}{\Kp\pim\pip}                   &   $394^{+20}_{-20} $\\
           & $N(\Dssp\Dzb + \Dsp\Dstarzb)$ in \decay{\Dsp}{\phiz\pip}                     &   $1733^{+38}_{-37} $\\
           & $N(\Dssp\Dzb + \Dsp\Dstarzb)$ in \decay{\Dsp}{\pip\pim\pip}                  &   $804^{+25}_{-25} $\\
           & $N(\Dssp\phiz + D_{s}^{(*)+}\Km\Kstarz)$ in \decay{\Dsp}{\Kp\Km\pip}         &   $139^{+19}_{-20} $\\
           & $N(\Dssp\phiz + D_{s}^{(*)+}\Km\Kstarz)$ in \decay{\Dsp}{\Kp\pim\pip}        &   $7^{+4}_{-4} $\\
           & $N(\Dssp\phiz + D_{s}^{(*)+}\Km\Kstarz)$ in \decay{\Dsp}{\phiz\pip}          &   $67^{+10}_{-10} $\\
           & $N(\Dssp\phiz + D_{s}^{(*)+}\Km\Kstarz)$ in \decay{\Dsp}{\Kp\pim\pip}        &   $25^{+7}_{-6} $\\

           & $N_{\text{comb}}(\Dsp\Dzb)$ \decay{\Dsp}{\Kp\Km\pip} in H1                   &   $112^{+30}_{-25} $\\
           & $N_{\text{comb}}(\Dsp\Dzb)$ \decay{\Dsp}{\Kp\Km\pip} in H2                   &   $50^{+17}_{-14} $\\
           & $N_{\text{comb}}(\Dsp\Dzb)$ \decay{\Dsp}{\Kp\pim\pip} in H1                  &   $71^{+20}_{-17} $\\
           & $N_{\text{comb}}(\Dsp\Dzb)$ \decay{\Dsp}{\Kp\pim\pip} in H2                  &   $41^{+12}_{-10} $\\
           & $N_{\text{comb}}(\Dsp\Dzb)$ \decay{\Dsp}{\phiz\pip} in H1                    &   $31^{+14}_{-11} $\\
           & $N_{\text{comb}}(\Dsp\Dzb)$ \decay{\Dsp}{\phiz\pip} in H2                    &   $12^{+9}_{-6} $\\
           & $N_{\text{comb}}(\Dsp\Dzb)$ \decay{\Dsp}{\pip\pim\pip} in H1                 &   $41^{+14}_{-11} $\\
           & $N_{\text{comb}}(\Dsp\Dzb)$ \decay{\Dsp}{\pip\pim\pip} in H2                 &   $35^{+14}_{-11} $\\
           & $N_{\text{comb}}(\Dsp\phiz)$ \decay{\Dsp}{\Kp\Km\pip} in H1                  &   $107^{+23}_{-20} $\\
           & $N_{\text{comb}}(\Dsp\phiz)$ \decay{\Dsp}{\Kp\Km\pip} in H2                  &   $52^{+12}_{-11} $\\
           & $N_{\text{comb}}(\Dsp\phiz)$ \decay{\Dsp}{\Kp\pim\pip} in H1                 &   $19^{+7}_{-6} $\\
           & $N_{\text{comb}}(\Dsp\phiz)$ \decay{\Dsp}{\Kp\pim\pip} in H2                 &   $18^{+5}_{-4} $\\
           & $N_{\text{comb}}(\Dsp\phiz)$ \decay{\Dsp}{\phiz\pip} in H1                   &   $38^{+14}_{-12} $\\
           & $N_{\text{comb}}(\Dsp\phiz)$ \decay{\Dsp}{\phiz\pip} in H2                   &   $27^{+8}_{-7} $\\
           & $N_{\text{comb}}(\Dsp\phiz)$ \decay{\Dsp}{\pip\pim\pip} in H1                &   $42^{+10}_{-9} $\\
           & $N_{\text{comb}}(\Dsp\phiz)$ \decay{\Dsp}{\pip\pim\pip} in H2                &   $25^{+7}_{-6} $\\
           & $N_{\text{comb}}(\Dsp\phiz)$ \phiz-sideband \decay{\Dsp}{\Kp\Km\pip} in H1   &   $33^{+13}_{-10} $\\
           & $N_{\text{comb}}(\Dsp\phiz)$ \phiz-sideband \decay{\Dsp}{\Kp\Km\pip} in H2   &   $26^{+11}_{-9} $\\
           & $N_{\text{comb}}(\Dsp\phiz)$ \phiz-sideband \decay{\Dsp}{\Kp\pim\pip} in H1  &   $18^{+7}_{-6} $\\
           & $N_{\text{comb}}(\Dsp\phiz)$ \phiz-sideband \decay{\Dsp}{\Kp\pim\pip} in H2  &   $14^{+5}_{-5} $\\
           & $N_{\text{comb}}(\Dsp\phiz)$ \phiz-sideband \decay{\Dsp}{\phiz\pip} in H1    &   $5^{+6}_{-4} $\\
           & $N_{\text{comb}}(\Dsp\phiz)$ \phiz-sideband \decay{\Dsp}{\phiz\pip} in H2    &   $12^{+8}_{-7} $\\
           & $N_{\text{comb}}(\Dsp\phiz)$ \phiz-sideband \decay{\Dsp}{\pip\pim\pip} in H1 &   $16^{+8}_{-6} $\\
           & $N_{\text{comb}}(\Dsp\phiz)$ \phiz-sideband \decay{\Dsp}{\pip\pim\pip} in H2 &   $14^{+8}_{-8} $\\
\hline
Fractions  & Ratio of \Dsp\Kp\Km to \Dsp\Dzb                                              &   $0.024^{+0.004}_{-0.004} $\\
           & Fraction of \Dsp\Km\Kstarz in (\Dssp\Km\Kstarz+\Dsp\Km\Kstarz)               &   $0.664^{+0.046}_{-0.044} $\\
           & Fraction of $\Dssp\phiz$ in ($D_{s}^{(*)+}\Km\Kstarz$ + $\Dssp\phiz$)        &   $0.172^{+0.103}_{-0.135} $\\
           & Ratio of $\Dsp D_{(s)}^{(*)-}$ to ($\Dssp\phiz$ + $D_{s}^{(*)+}\Km\Kstarz$)  &   $0.567^{+0.129}_{-0.107} $\\
           & Fraction of \Dssp\Dzb in (\Dssp\Dzb+\Dsp\Dstarzb) in H1                      &   $0.302^{+0.031}_{-0.031} $\\
           & Fraction of \Dssp\Dzb in (\Dssp\Dzb+\Dsp\Dstarzb) in H2                      &   $0.342^{+0.037}_{-0.037} $\\
           & Fraction of normalisation part reco in H1                                    &   $0.593^{+0.005}_{-0.005} $\\
           & Fraction of normalisation peak in H1                                         &   $0.592^{+0.010}_{-0.010} $\\
           & Ratio of \Dssp\Dstarzb to (\Dssp\Dzb + \Dsp\Dstarzb)                         &   $0.607^{+0.016}_{-0.015} $\\
\hline
\caption{The fit result with final values of all floating variables used in the fit model. Here H1 and H2 represent the $|\cos\theta_{K}|>0.4$ and $|\cos\theta_{K}|<0.4$ categories respectively.} 
\label{tab:B2DsPhi_free_variables} 
\end{longtable}



\section{Efficiency corrections}
\label{sec:B2DsPhi_effcorrections}

The branching fraction for \decay{\Bp}{\Dsp\phiz} decays is determined by correcting the yields for the signal and normalisation channels by their respective efficiencies. In determining the efficiencies both the \decay{\Bp}{\Dsp\phiz} and \decay{\Bp}{\Dsp\Dzb} decays are assumed to be pseudo-two-body decays in which variations as a function of phase-space are negligible. This means the relative efficiency can be calculated as a simple ratio for each \Dsp decay mode, simplifying the correction considerably.    

The efficiencies for each stage of selection are either determined from the appropriate simulation samples for the signal and normalisation decays, or from dedicated calibration data samples for the efficiencies involving particle identification requirements. 


\subsection{Efficiencies from simulation}

The efficiencies for each step of selection are listed separately for the different \Dsp decay modes in Table~\ref{tab:B2DsPhi_eff}. These steps are closely related to the stages previously described in Chapter~\ref{ch:selection}, however more specific details of what is included in each step is described here. The efficiencies are calculated separately for the different years of data taking, however they are combined here, weighting according to the relative contributions to the total data set. 

% \begin{table}[h]
% \centering
% \begin{tabular}{ l c c c }

% \hline
% Requirement             & $\epsilon(\decay{\Bp}{\Dsp\Dzb})$  & $\epsilon(\decay{\Bp}{\Dsp\phiz})$ & Ratio \\ 
% \hline
% Acceptance              & 17.42 $\pm$ 0.03         & 18.35 $\pm$ 0.04      & 0.950 $\pm$ 0.003  \\
% Reconstruction          &  2.11 $\pm$ 0.02         &  1.96 $\pm$ 0.02      & 1.077 $\pm$ 0.013  \\
% Trigger                 & 17.78 $\pm$ 0.04         & 18.63 $\pm$ 0.04      & 0.954 $\pm$ 0.003  \\
% Mass window             & 17.78 $\pm$ 0.04         & 18.63 $\pm$ 0.04      & 0.954 $\pm$ 0.003  \\
% Vetoes                  & 17.78 $\pm$ 0.04         & 18.63 $\pm$ 0.04      & 0.954 $\pm$ 0.003  \\
% $\chi^{2}_{\text{IP}}$  & 17.78 $\pm$ 0.04         & 18.63 $\pm$ 0.04      & 0.954 $\pm$ 0.003  \\
% Charmless               & 17.78 $\pm$ 0.04         & 18.63 $\pm$ 0.04      & 0.954 $\pm$ 0.003  \\
% \hline
% \end{tabular}
% \caption{Efficiencies (in \%) determined from simulation samples for \decay{\Dsp}{\Kp\Km\pip} decays and the ratio $\epsilon(\decay{\Bp}{\Dsp\Dzb})/\epsilon(\decay{\Bp}{\Dsp\phiz})$. The errors are statistical.} 
% \label{tab:B2DsPhi_eff_KKPi} 
% \end{table}

% KKPi   17.42352941    0.034705882    18.34588235    0.035588235    0.9495      0.002558824
% KKPi  2.111764706    0.02     1.958823529    0.015588235    1.077323529    0.013117647

% \begin{table}[h]
% \centering
% \begin{tabular}{ l c c c }

% \hline
% Requirement             & $\epsilon(\decay{\Bp}{\Dsp\Dzb})$  & $\epsilon(\decay{\Bp}{\Dsp\phiz})$ & Ratio \\  
% \hline
% Acceptance              & 16.53 $\pm$ 0.04         & 17.36 $\pm$ 0.04      & 0.952 $\pm$ 0.003  \\
% Reconstruction          & 17.78 $\pm$ 0.04         & 18.63 $\pm$ 0.04      & 0.954 $\pm$ 0.003  \\
% Trigger                 & 17.78 $\pm$ 0.04         & 18.63 $\pm$ 0.04      & 0.954 $\pm$ 0.003  \\
% Mass window             & 17.78 $\pm$ 0.04         & 18.63 $\pm$ 0.04      & 0.954 $\pm$ 0.003  \\
% Vetoes                  & 17.78 $\pm$ 0.04         & 18.63 $\pm$ 0.04      & 0.954 $\pm$ 0.003  \\
% $\chi^{2}_{\text{IP}}$  & 17.78 $\pm$ 0.04         & 18.63 $\pm$ 0.04      & 0.954 $\pm$ 0.003  \\
% Charmless               & 17.78 $\pm$ 0.04         & 18.63 $\pm$ 0.04      & 0.954 $\pm$ 0.003  \\
% \hline
% \end{tabular}
% \caption{Efficiencies (in \%) determined from simulation samples for \decay{\Dsp}{\Kp\pim\pip} decays and the ratio $\epsilon(\decay{\Bp}{\Dsp\Dzb})/\epsilon(\decay{\Bp}{\Dsp\phiz})$. The errors are statistical.} 
% \label{tab:B2DsPhi_eff_KPiPi} 
% \end{table}

% KPiPi  16.52647059    0.035588235    17.36147059    0.035588235    0.951941176    0.002558824
% KPiPi 2.263823529    0.02     2.063823529    0.02     1.095117647    0.014
% \begin{table}[h]
% \centering
% \begin{tabular}{ l c c c }

% \hline
% Requirement             & $\epsilon(\decay{\Bp}{\Dsp\Dzb})$  & $\epsilon(\decay{\Bp}{\Dsp\phiz})$ & Ratio \\   
% \hline
% Acceptance              & 15.91 $\pm$ 0.03         & 16.77 $\pm$ 0.04      & 0.949 $\pm$ 0.003  \\
% Reconstruction          & 17.78 $\pm$ 0.04         & 18.63 $\pm$ 0.04      & 0.954 $\pm$ 0.003  \\
% Trigger                 & 17.78 $\pm$ 0.04         & 18.63 $\pm$ 0.04      & 0.954 $\pm$ 0.003  \\
% Mass window             & 17.78 $\pm$ 0.04         & 18.63 $\pm$ 0.04      & 0.954 $\pm$ 0.003  \\
% Vetoes                  & 17.78 $\pm$ 0.04         & 18.63 $\pm$ 0.04      & 0.954 $\pm$ 0.003  \\
% $\chi^{2}_{\text{IP}}$  & 17.78 $\pm$ 0.04         & 18.63 $\pm$ 0.04      & 0.954 $\pm$ 0.003  \\
% Charmless               & 17.78 $\pm$ 0.04         & 18.63 $\pm$ 0.04      & 0.954 $\pm$ 0.003  \\
% \hline
% \end{tabular}
% \caption{Efficiencies (in \%) determined from simulation samples for \decay{\Dsp}{\pip\pim\pip} decays and the ratio $\epsilon(\decay{\Bp}{\Dsp\Dzb})/\epsilon(\decay{\Bp}{\Dsp\phiz})$. The errors are statistical.} 
% \label{tab:B2DsPhi_eff_PiPiPi} 
% \end{table}

% PiPiPi   15.91264706    0.031764706    16.76882353    0.034705882    0.949088235    0.002558824
% PiPiPi   2.364117647    0.02     2.135    0.02     1.106411765    0.013852941



%%%%%%%
\begin{table}[h]
   \centering
   \begin{tabular}{ l l c c c }
      \hline
      Requirement             & \Dsp mode         & $\epsilon(\decay{\Bp}{\Dsp\Dzb})$  & $\epsilon(\decay{\Bp}{\Dsp\phiz})$ & Ratio \\
      \hline
      Acceptance              & $\Kp\Km\pip$      & 17.42 $\pm$ 0.03         & 18.35 $\pm$ 0.04      & 0.950 $\pm$ 0.003  \\
                              & $\Kp\pim\pip$     & 16.53 $\pm$ 0.04         & 17.36 $\pm$ 0.04      & 0.952 $\pm$ 0.003  \\
                              & $\pip\pim\pip$    & 15.91 $\pm$ 0.03         & 16.77 $\pm$ 0.04      & 0.949 $\pm$ 0.003  \\
      \hline
      Reconstruction          & $\Kp\Km\pip$      & 2.11 $\pm$ 0.02         & 1.96 $\pm$ 0.02     & 1.078 $\pm$ 0.013  \\
                              & $\Kp\pim\pip$     & 2.27 $\pm$ 0.02         & 2.06 $\pm$ 0.02     & 1.095 $\pm$ 0.014  \\
                              & $\pip\pim\pip$    & 2.36 $\pm$ 0.02         & 2.13 $\pm$ 0.02     & 1.105 $\pm$ 0.014  \\
      \hline
      Trigger                 & $\Kp\Km\pip$      & 93.3 $\pm$ 0.2         & 93.1 $\pm$ 0.2     & 1.003 $\pm$ 0.003  \\
                              & $\Kp\pim\pip$     & 95.1 $\pm$ 0.2         & 93.5 $\pm$ 0.2     & 1.017 $\pm$ 0.003  \\
                              & $\pip\pim\pip$    & 95.3 $\pm$ 0.2         & 93.7 $\pm$ 0.2     & 1.017 $\pm$ 0.003  \\
      \hline
      Mass window             & $\Kp\Km\pip$      & 96.2 $\pm$ 0.2         & 94.1 $\pm$ 0.2     & 1.022 $\pm$ 0.003  \\
                              & $\Kp\pim\pip$     & 94.9 $\pm$ 0.2         & 94.1 $\pm$ 0.2     & 1.008 $\pm$ 0.003  \\
                              & $\pip\pim\pip$    & 93.7 $\pm$ 0.2         & 92.4 $\pm$ 0.2     & 1.015 $\pm$ 0.004  \\
      \hline
      Vetos                   & $\Kp\Km\pip$      & 99.9 $\pm$ 0.0         & 95.2 $\pm$ 0.2     & 1.049 $\pm$ 0.002  \\
                              & $\Kp\pim\pip$     & 99.9 $\pm$ 0.0         & 92.4 $\pm$ 0.3     & 1.080 $\pm$ 0.004  \\
                              & $\pip\pim\pip$    & 99.9 $\pm$ 0.0         & 90.3 $\pm$ 0.3     & 1.105 $\pm$ 0.004  \\
      \hline
      Charmless               & $\Kp\Km\pip$      & 72.6 $\pm$ 0.4         & 100.0 $\pm$ 0.0     & 0.726 $\pm$ 0.004  \\
                              & $\Kp\pim\pip$     & 65.8 $\pm$ 0.4         & 63.0 $\pm$ 0.5     & 1.044 $\pm$ 0.011  \\
                              & $\pip\pim\pip$    & 66.0 $\pm$ 0.4         & 82.9 $\pm$ 0.4     & 0.796 $\pm$ 0.006  \\
      \hline
      $\chi^{2}_{\text{IP}}$  & $\Kp\Km\pip$      & 100.0 $\pm$ 0.0         & 96.2 $\pm$ 0.2     & 1.039 $\pm$ 0.002  \\
                              & $\Kp\pim\pip$     & 100.0 $\pm$ 0.0         & 95.7 $\pm$ 0.3     & 1.045 $\pm$ 0.003  \\
                              & $\pip\pim\pip$    & 100.0 $\pm$ 0.0         & 95.9 $\pm$ 0.3     & 1.043 $\pm$ 0.003  \\
      \hline
   \end{tabular}
   \caption{Efficiencies (in \%) determined from simulation samples for signal and normalisation decays and the ratio $\epsilon(\decay{\Bp}{\Dsp\Dzb})/\epsilon(\decay{\Bp}{\Dsp\phiz})$ for each \Dsp decay mode. The errors are statistical.} 
   \label{tab:B2DsPhi_eff} 
\end{table}



\begin{description}

\item \textbf{Acceptance:} this accounts for the fraction of generated decays in which all five final state tracks end up within the \lhcb detector's acceptance. For all \Dsp decay modes the signal decay has a slightly higher efficiency than the normalisation channel. This may because of the kinematics of the \decay{\phiz}{\Kp\Km} decay in which the two kaon tracks are typically close to one another and therefore more likely to both be in the acceptance.   

\item \textbf{Reconstruction:} this efficiency determines the fraction of decays in which all tracks are well reconstructed and combined into a suitable candidate. The candidate must pass all requirements outlined in the \emph{Stripping Line} for the specific decay. The reconstruction selections used for both the signal and normalisation decays explicitly require that the event in which the candidate was found had fired the trigger. Therefore this efficiency includes some, but not all, of the trigger efficiency. This results in small efficiencies around 2\% for each mode. Here, the efficiencies are slightly larger for the normalisation channel than the signal. This may be because the kaons from the \phiz decay are typically lower momentum than those from the \Dzb, and therefore less likely to be reconstructed.  

\item \textbf{Trigger:} this efficiency effectively accounts for the likelihood that the candidate passed at least one of the requirements in Sec.~\ref{sec:selection_trigger}, given a trigger fired in that event. These efficiencies have very high values around 94\%.  

\item \textbf{Mass windows:} this represents the efficiency for the candidates to be within the invariant mass windows for the \Dsp and \phiz or \Dzb mesons. Again, this is slightly high for the normalisation than the signal. The \phiz meson invariant mass peak a fairly long tail extending to higher invariant masses that may be the cause.  

\item \textbf{Vetoes:} the efficiency of the kinematic vetoes described in Sec.\ref{sec:kinematicvetos} is included in this quantity. The misidentified \D and \Lc hadron vetoes target the \Dsp meson, present in both the signal and normalisation decays. The relative efficiency is assumed to be one for these specific vetoes therefore not included in this efficiency. The systematic uncertainty resulting from this assumption is discussed in Sec.\ref{sec:B2DsPhi_systuncertainy}. Most of the kinematic vetoes are only applied to the signal mode, hence why the normalisation channel efficiencies are almost 100\%.

\item \textbf{Charmless:} the requirements applied to the flight distance significance of the \Dsp meson is tuned differently for each \Dsp decay and signal and normalisation. As a result, the efficiencies have large variations between the different modes.

\item \textbf{$\chi^{2}_{\text{IP}}$:} the efficiency of the $\chi^{2}_{\text{IP}}$ requirements for the \Bp and \Dsp candidates as detailed in Sec.\ref{sec:selection_IPCHI2} are included in this quantity. These are only applied to the signal mode therefore the normalisation mode is 100\% efficient.

\end{description}





\subsection{Efficiencies requiring calibration samples}

The efficiencies of the particle identification and MVA requirements are both determined using input from dedicated calibration samples as the distributions are known to be poorly represented in simulations. The method used here differs from that already described in Sec.~\ref{sec:B2DsKK_eff_from_calib} as the \decay{\Bp}{\Dsp\phiz} decay can be assumed to be a pseudo-two-body decay in which phase-space dependent efficiencies are not required. 

%%%%%%%
\begin{table}[h]
   \centering
   \begin{tabular}{ l l c c c }
      \hline
      Requirement             & \Dsp mode         & $\epsilon(\decay{\Bp}{\Dsp\Dzb})$  & $\epsilon(\decay{\Bp}{\Dsp\phiz})$ & Ratio \\
      \hline
      PID                     & $\Kp\Km\pip$      & 88.1 $\pm$ 0.1         & 89.9 $\pm$ 0.0     & 0.980 $\pm$ 0.001  \\
                              & $\Kp\pim\pip$     & 86.7 $\pm$ 0.2         & 88.8 $\pm$ 0.1     & 0.977 $\pm$ 0.002  \\
                              & $\pip\pim\pip$    & 85.3 $\pm$ 0.1         & 87.0 $\pm$ 0.0     & 0.980 $\pm$ 0.001  \\
      \hline
      MVA                     & $\Kp\Km\pip$      & 53.2 $\pm$ 0.4         & 57.1 $\pm$ 0.4     & 0.932 $\pm$ 0.010  \\
                              & $\Kp\pim\pip$     & 42.9 $\pm$ 0.4         & 46.0 $\pm$ 0.5     & 0.932 $\pm$ 0.013  \\
                              & $\pip\pim\pip$    & 46.3 $\pm$ 0.4         & 49.7 $\pm$ 0.4     & 0.933 $\pm$ 0.012  \\
      \hline
   \end{tabular}
   \caption{Efficiencies (in \%) determined from the relevant calibration and validation samples using input from simulation and the ratio $\epsilon(\decay{\Bp}{\Dsp\Dzb})/\epsilon(\decay{\Bp}{\Dsp\phiz})$ for each \Dsp decay mode. The errors are statistical.} 
   \label{tab:B2DsPhi_eff_from_calib} 
\end{table}

\subsubsection{PID efficiency}

The efficiency of the particle identification requirements as described in Sec.~\ref{sec:pidrequirements} are determined using a package called \texttt{PIDCalib}. This uses calibrations samples for the different particle species to determine the fraction of candidates passing the various PID variable requirements. The samples are background-subtracted to isolate the distributions of the PID variables for the tracks of interest. The calibration samples for both \Kp and \pip mesons are collected from a sample of \decay{\Dstarp}{(\decay{\Dz}{\Kp\pim})\pip} decays, using the decay products of the \Dz decay. 

The PID variable distributions depend on both the kinematics of the track in question and the occupancy of the detector as a whole. 
These can both affect the characteristics of the hits in the \rich detectors and therefore result in different PID variable distributions. The calibrations samples are binned in three variables; the momentum of the track \ptot, the pseudo-rapidity \Peta, and the number of tracks in an event $n_{\text{Tracks}}$. Input is used from simulation samples to determine the \ptot, \Peta and $n_{\text{Tracks}}$ distributions for the signal decays and weight the calibration samples accordingly. The efficiencies of the PID requirements on the signal and normalisation for the different \Dsp decay modes are listed in Table.~\ref{tab:B2DsPhi_eff_from_calib}.


\subsubsection{MVA efficiency}

The efficiency of the MVA requirements is determined using the method outlined in Sec.~\ref{sec:selection_MVA_eff}. The validation samples of \decay{\Bsb}{\Dsp\pim} and \decay{\Bs}{\jpsi\phiz} decays are binned in transverse momentum \pt and flight distance significance $\chi^{2}_{\text{FD}}$. The background-subtracted MVA responses are extracted for each bin. Samples of simulated decays are iterated though, finding the \pt and $\chi^{2}_{\text{FD}}$ values for each candidate and calculating the MVA cut efficiency from the corresponding validation samples. The total efficiency for each candidate is the product of the \Dsp and \phiz MVA selection efficiencies. The total efficiency for the whole sample is then given by the sum of the per-candidate efficiencies as listed in Table.~\ref{tab:B2DsPhi_eff_from_calib}.




\subsection{Total efficiencies}






\section{Systematic uncertainties}
\label{sec:B2DsPhi_systuncertainy}

The sources of systematic uncertainty are broadly similar to those considered in the search for \decay{\Bp}{\Dsp\Kp\Km} decays. The more complex fit strategy and model requires most of these systematics to be reassessed and additional source included. 

\subsection{Relative efficiencies}

\subsection{Signal and normalisation PDFs}

\subsection{Background PDFs}

\subsection{Charmless contribution}

\subsection{$\decay{\Bp}{\Dsp\Kp\Km}$ model assumption}


\section{Results}
\label{sec:B2DsPhi_results}

{\color{Red}
\begin{itemize}
\item Copy most of results section from paper
\end{itemize}
}

The fit to $\decay{\Bp}{\Dsp\phiz}$ candidates finds a total yield of $N(\decay{\Bp}{\Dsp\phiz}) = 5.3 \pm 6.7$, summed across all categories and \Dsp meson decay modes. 
A yield of $N(\decay{\Bp}{\Dsp\Km\Kp}) = 65 \pm 10 $ is found, consistent with the yield obtained from the full $\decay{\Bp}{\Dsp\Kp\Km}$ measurement. 
The branching fraction for $ \decay{\Bp}{\Dsp\phiz}$ decays is calculated as
\begin{equation}
\BF(\decay{\Bp}{\Dsp\phiz}) = R \times \frac{\BF(\decay{\Dzb}{\Kp\Km})}{\BF(\decay{\phi}{\Kp\Km})} \times \BF(\decay{\Bp}{\Dsp\Dzb}),
\label{eq:branching_fraction_calc}
\end{equation}
where the branching fraction $\BF(\decay{\phi}{\Kp\Km})= 0.489 \pm 0.005$ has been used~\cite{PDG2016}. 

The free variable $R$ is defined to be the ratio of the signal and normalisation yields, corrected for the selection efficiencies.
The yield of signal candidates in each \Dsp mode is constructed from $R$ and the normalisation yield for the given \Dsp decay mode, $N(\decay{\Bp}{\Dsp\Dzb})$. The product of these two quantities is corrected by the ratio of selection efficiencies
\begin{equation}
N(\decay{\Bp}{\Dsp\phiz}) = R \times N(\decay{\Bp}{\Dsp\Dzb}) \times \frac{\epsilon(\decay{\Bp}{\Dsp\phiz})}{\epsilon(\decay{\Bp}{\Dsp\Dzb})}.
\label{eq:branching_fraction_R}
\end{equation}

The simultaneous fit measures a single value of $R$ for all \Dsp decay mode categories. From an ensemble of pseudoexperiments, $R$ is distributed normally. It can be written as the ratio of signal and normalisation branching fractions using Eq.~{\ref{eq:branching_fraction_calc}. The value is determined to be 
\begin{equation}
R = \frac{\BF(\decay{\Bp}{\Dsp\phiz})}{\BF(\decay{\Bp}{\Dsp\Dzb})}\times \frac{\BF(\decay{\phi}{\Kp\Km})}{\BF(\decay{\Dzb}{\Kp\Km})} =(1.6^{+2.2}_{-1.9}\pm 1.1) \times 10^{-3}, 
\end{equation}
where the first uncertainty is statistical and the second systematic. This corresponds to a branching fraction for $\decay{\Bp}{\Dsp\phiz}$ decays of

\begin{equation}
\BF(\decay{\Bp}{\Dsp\phiz}) = (1.2^{+1.6}_{-1.4} \pm 0.8  \pm 0.1)\times 10^{-7},
\label{eq:branching_fraction}
\end{equation}
where the first uncertainty is statistical, the second systematic, and the third results from the uncertainty on the branching fractions $\BF(\decay{\Bp}{\Dsp\Dzb})$, $\BF(\decay{\phi}{\Kp\Km})$ and $\BF(\decay{\Dzb}{\Kp\Km})$. Considering only the statistical uncertainty, the significance of the $\decay{\Bp}{\Dsp\phiz}$ signal is 0.8 standard deviations ($\sigma$). 


Th branching fraction for \decay{\Bp}{\Dsp\phiz} decays is also determined separately for the different \Dsp decays modes included in the search. These are found to be
\begin{equation}
  \left .
  \begin{aligned}
    &\decay{\Dsp}{\phiz\pip}      && : \BF(\decay{\Bp}{\Dsp\phiz}) = &+2.7^{+2.9}_{-2.3}\times 10^{-7}\\
    &\decay{\Dsp}{\Kp\Km\pip}     && : \BF(\decay{\Bp}{\Dsp\phiz}) = &+1.2^{+2.2}_{-1.8}\times 10^{-7}\\
    &\decay{\Dsp}{\pip\pim\pip}   && : \BF(\decay{\Bp}{\Dsp\phiz}) = &-9.4^{+3.6}_{-2.8}\times 10^{-7}\\
    &\decay{\Dsp}{\Kp\pim\pip}    && : \BF(\decay{\Bp}{\Dsp\phiz}) = &+3.7^{+1.2}_{-7.6}\times 10^{-7},\\
  \end{aligned} \right.
\end{equation} 
where these results correspond to the following yields for each \Dsp decay mode
\begin{equation}
  \left .
  \begin{aligned}
    &\decay{\Dsp}{\phiz\pip}      && : N(\decay{\Bp}{\Dsp\phiz}) = &+3.9^{+4.2}_{-3.3} \\
    &\decay{\Dsp}{\Kp\Km\pip}     && : N(\decay{\Bp}{\Dsp\phiz}) = &+2.7^{+5.0}_{-4.2} \\
    &\decay{\Dsp}{\pip\pim\pip}   && : N(\decay{\Bp}{\Dsp\phiz}) = &-5.2^{+2.0}_{-1.6} \\
    &\decay{\Dsp}{\Kp\pim\pip}    && : N(\decay{\Bp}{\Dsp\phiz}) = &+0.8^{+2.6}_{-1.7}. \\
  \end{aligned} \right.
\end{equation} 
A visual representation of these measurements are shown in Fig.~\ref{fig:B2DsPhi_split_Ds_results} along with the value determined using all modes simultaneously.

%%%%%%%%%%%%%%%%%%%%%%%%%%%%%%%%%%%%%%%%%%%%%%%%%%%%%%%%%%
\begin{figure}[!h]
    \centering
        \includegraphics[width=0.8\textwidth]{figs/B2DsPhi/Split_Ds_modes.pdf}
        \caption{Results split for different \Dsp decay modes.}
    \label{fig:B2DsPhi_split_Ds_results}   
\end{figure}
%%%%%%%%%%%%%%%%%%%%%%%%%%%%%%%%%%%%%%%%%%%%%%%%%%%%%%%%%%

\subsection{Limit setting}
\label{sec:B2DsPhi_limitsetting}

The measured branching fraction, $\BF(\decay{\Bp}{\Dsp\phiz}) = (1.2^{+1.6}_{-1.4} \pm 0.8  \pm 0.1)\times 10^{-7}$, is not significant enough to constitute evidence or an observation for the \decay{\Bp}{\Dsp\phiz} decay and is consistent with a branching fraction of zero. Whilst this measurement is useful in itself, for example it could provide constraints in combination with other results, it is also useful to set a limit on the branching fraction for a more straightforward comparison with theoretical predictions.
Three different methods of limit estimation are attempted. These methods make different assumptions are therefore applicable in slightly different situations.

\subsubsection{The $\text{CL}_{\text{S}}$ method}

The first method tried is the $\text{CL}_{\text{S}}$ method, widely used in the high energy physics community. This method tests the p-value of a signal plus background hypothesis, $\text{CL}_{\text{S+B}}$, against a background only hypothesis, $\text{CL}_{\text{B}}$,
\begin{equation}
\text{CL}_{\text{S}}  = \frac{\text{CL}_{\text{S}+\text{B}}}{\text{CL}_{\text{B}}}.
\end{equation}
The free parameters in the fit other than the POI are considered nuisance parameters.
This method is implemented using the \roostats package within the \root framework. 

The $\text{CL}_{\text{S}}$ distribution is shown in Fig.~\ref{fig:B2DsPhi_limit_CLS} which includes the $1\sigma$ bands in green and $2\sigma$ bands in yellow.
The 95\% upper limit is determined as the point where the $\text{CL}_{\text{S}}$ value falls below 5\% as illustrated by the red line.
This corresponds to a 95\% limit of
\begin{equation}
\BF(\decay{\Bp}{\Dsp\phiz}) < 4.2 \times 10^{-7}.
\label{eq:B2DsPhi_upperlimit_CLS}
\end{equation}


{\color{Red}
\begin{itemize}
\item Methods 
\item Freq calc, Hybrid Calc and Aym Calc.
\item citations
\end{itemize}
}

%%%%%%%%%%%%%%%%%%%%%%%%%%%%%%%%%%%%%%%%%%%%%%%%%%%%%%%%%%
\begin{figure}[!h]
    \centering
        \includegraphics[width=0.7\textwidth]{figs/B2DsPhi/CLs_Branching_fraction_better.pdf}
        \caption{$\text{CL}_{\text{S}}$ limit determination.}
    \label{fig:B2DsPhi_limit_CLS}   
\end{figure}
%%%%%%%%%%%%%%%%%%%%%%%%%%%%%%%%%%%%%%%%%%%%%%%%%%%%%%%%%%

\subsubsection{The profile likelihood method}
 
An upper limit at 95\% confidence limit is determined for the branching fraction $\BF(\decay{\Bp}{\Dsp\phiz})$ using the profile likelihood method. This is calculated by determining the value of the branching fraction $x_{\text{U}}$ that satisfies the equation
 
\begin{equation}
\frac{\int_{0}^{x_{\text{U}}} \mathcal{L}(x) dx}{\int_{0}^{\infty} \mathcal{L}(x) dx} = 0.95,
\label{eq:likelihood}
\end{equation}
where $\mathcal{L}(x)$ is the profile likelihood as a function of the branching fraction $\BF(\decay{\Bp}{\Dsp\phiz})$.
This Bayesian method integrates the prior knowledge about the branching fraction, namely that the value must be greater than or equal to zero; the profile likelihood is integrated from zero upwards.  
The 95\% CL limit determined when considering only statistical uncertainties is
\begin{equation}
\BF(\decay{\Bp}{\Dsp\phiz}) < 4.1 \times 10^{-7}.
\label{eq:stat_upperlimit}
\end{equation}
 
To account for systematic uncertainty, the likelihood is convolved with a Gaussian distribution with a width given by the systematic uncertainty. The likelihood and difference in the log-likelihood are shown in Fig.~\ref{fig:B2DsPhi_limit_likelihood}, with and without the systematic uncertainty included.
The limit at 95\% CL including the systematic uncertainty is determined to be
\begin{equation}
\BF(\decay{\Bp}{\Dsp\phiz}) < 4.4 \times 10^{-7}.
\label{eq:upperlimit}
\end{equation}


%%%%%%%%%%%%%%%%%%%%%%%%%%%%%%%%%%%%%%%%%%%%%%%%%%%%%%%%%%
\begin{figure}[!h]
    \centering
        \includegraphics[width=0.9\textwidth]{figs/B2DsPhi/Likelihood_limits.eps}
         \caption{Bayesian profile likelihood limit determination: the (left) difference in the log-likelihood and (right) the likelihood as a function of the assumed \decay{\Bp}{\Dsp\phiz} branching fraction. The black distributions include only the statistical uncertainty, whilst the blue also include the systematic uncertainty. The shaded regions represent the areas integrated to determine the 95\% CL limits. }
    \label{fig:B2DsPhi_limit_likelihood}   
\end{figure}
%%%%%%%%%%%%%%%%%%%%%%%%%%%%%%%%%%%%%%%%%%%%%%%%%%%%%%%%%%

{\color{Red}
\begin{itemize}
\item include assumptions and possible issues
\item asymmetric likelihood 
\end{itemize}
}
\subsubsection{The Feldman-Cousins method}

Upper limits at 95\% and 90\% confidence levels (CL) are also determined using the Feldman-Cousins approach~\cite{FeldmanCousins}. An ensemble of pseudo-experiments is generated for different values of the branching fraction $\BF(\decay{\Bp}{\Dsp\phiz})$. These generated pseudo-experiments are then fitted with the nominal fit model to calculate the fitted branching fraction and associated statistical uncertainty, $\sigma_{\text{stat}}$. This method constructs confidence bands based on a likelihood ratio method, calculating the probability of fitting a branching fraction for a given generated branching fraction. This probability is assumed to follow a Gaussian distribution with width $\sigma = \sqrt{\sigma_{\text{stat}}^{2}+\sigma_{\text{syst}}^{2}}$, where $\sigma_{\text{stat}}$ and $\sigma_{\text{syst}}$ are the statistical and systematic uncertainties. The dominant source of systematic uncertainty in this measurement is from the background PDFs. As the size of this uncertainty is not expected to vary as a function of the generated branching fraction, $\sigma_{\text{syst}}$ is assumed to be constant. Nuisance parameters are accounted for using the plug-in method~\cite{plugin}. The generated confidence bands are shown in Fig.~\ref{fig:B2DsPhi_limit_FC}, where the statistical-only 90\% and 95\% confidence level bands are shown, along with the 95\% confidence level band with systematic uncertainty included. 
This corresponds to a statistical-only 95\% (90\%) confidence level of $\BF(\decay{\Bp}{\Dsp\phiz}) < 4.4 \times 10^{-7}~(3.9 \times 10^{-7})$, and a 95\% (90\%) confidence level including systematic uncertainties of
\begin{equation}
\BF(\decay{\Bp}{\Dsp\phiz}) < 4.9 \times 10^{-7}~(4.2 \times 10^{-7}).
\label{eq:B2DsPhi_upperlimit_FC}
\end{equation}
%%%%%%%%%%%%%%%%%%%%%%%%%%%%%%%%%%%%%%%%%%%%%%%%%%%%%%%%%%
\begin{figure}[!h]
    \centering
        \includegraphics[width=0.8\textwidth]{figs/B2DsPhi/Sensitivity_plot.eps}
        \caption{Confidence bands produced using the Feldman-Cousins approach. The green and yellow bands represent the statistical-only 90\% and 95\% confidence level bands and the black dotted line represents the 95\% limit including systematic uncertainties. The measured value of the branching fraction is shown by the vertical red line, and the corresponding 95\% confidence levels, with and without systematic uncertainties, are represented by the dotted red lines.}
    \label{fig:B2DsPhi_limit_FC}   
\end{figure}
%%%%%%%%%%%%%%%%%%%%%%%%%%%%%%%%%%%%%%%%%%%%%%%%%%%%%%%%%%



{\color{Red}
\begin{itemize}
\item Document all methods attempted
\item CLs plots
\item likelihood
\item FC bands
\item Table of comparison 
\end{itemize}
}

\subsection{Comparison to 2011?}
