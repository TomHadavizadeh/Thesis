
%ABSTRACT
%
%The abstract environment puts a large, bold, centered "Abstract" label at
%the top of the page. The abstract itself appears in a quote environment,
%i.e. tabbed in at both sides, and on its own page.

% \begin{alwayssingle} 
%  \thispagestyle{empty}
%  \begin{center}
%    \vspace*{1.5cm}
%    {\Large \bfseries  Abstract}
%  \end{center}
%  \vspace{0.5cm}
%  \begin{quote}
%  Brief summary of things
%  \end{quote}
% \end{alwayssingle}


%The abstractlong environment puts a large, bold, centered "Abstract" label at
%the top of the page. The abstract itself does not appears in a quote
%environment so you can get more in.
% \begin{center}
%     { \Huge {\bfseries {\@title}} \par}
% {\large \vspace*{40mm} {\logo \par} \vspace*{25mm}}
%     {{\Large \@author} \par}
% {\large \vspace*{1ex}
%     {{\@college} \par}
% \vspace*{1ex}
%     {University of Oxford \par}
% \vspace*{25mm}
%     {{\submittedtext} \par}
% \vspace*{1ex}
%     {\it {\@degree} \par}
% \vspace*{2ex}
%     {\@degreedate}}
%   \end{center}

% \begin{alwayssingle} 
%  \thispagestyle{empty}
%  \begin{center}
%     {\@title}
%    \vspace*{1.5cm}
%    {\Large \bfseries  Abstract}
%  \end{center}
%  \vspace{0.5cm}
%  Brief summary of things
% \end{alwayssingle}


%The abstractseparate environment is for running of a page with the abstract
%on including title and author etc as required to be handed in separately

% \title{Rare hadronic decays of charged B meson at LHCb}
% \author{Tom Hadavizadeh}
% \college{St Peter's College}

\begin{alwayssingle} 
 \thispagestyle{empty}
 \vspace*{-1in}
 \begin{center}
   { \Large {\bfseries {Rare hadronic decays of charged B~mesons at LHCb}} \par}
   {{\large \vspace*{1ex} Thomas Hadavizadeh} \par}
   {\large \vspace*{1ex}
   {{St Peter's College} \par}
   {University of Oxford \par}
   \vspace*{1ex}
   {{\it \submittedtext} \par}
   {\it {Doctor of Philosophy} \par}
   {\it {at the University of Oxford} \par}
    \vspace*{2ex}
    {Hilary 2018}\par}
    \vspace*{1.5cm}
    {\Large \bfseries  Abstract}
  \end{center}

  This thesis documents two searches for rare hadronic decays of \Bp mesons with the \lhcb experiment at the Large Hadron Collider. 
  Both are performed using proton-proton collision data corresponding to an integrated luminosity of 4.8 fb$^{-1}$, collected at centre-of-mass energies of 7, 8 and 13\tev during 2011--2016.
  
  The first is a search for \decay{\Bp}{\Dsp\Kp\Km} decays. A significant signal is observed for the first time and the branching fraction is determined to be
  \begin{equation*}
  \mathcal{B}(B^{+} \to D_s^{+}K^{+}K^{-} ) = (7.1 \pm 0.5 \pm 0.6 \pm 0.7) \times 10^{-6}, 
  \end{equation*}
  \noindent where the first uncertainty is statistical, the second systematic and the third due to the uncertainty on 
  the branching fraction of the normalisation mode \decay{\Bp}{\Dsp\Dzb}.
  
  The second search is performed for the rare pure annihilation decay \decay{\Bp}{\Dsp\phiz}.
  No significant signal is observed and a limit of
  \begin{equation*}
  \mathcal{B}(B^{+} \to D_s^{+}\phi) < 4.9 \times 10^{-7}
  \end{equation*}
  is set on the branching fraction at 95\% confidence level.



\end{alwayssingle}