\chapter{Event selection} 
\label{ch:selection}

\minitoc

In this chapter the procedure used to reconstruct and select candidate $\Bp \to \Dsp \phi $ and $\Bp \to \Dsp \Kp \Km$ decays is described. 


\section{Online selection}

The online event selection is performed by a trigger, which consists of a hardware stage, based on information from the calorimeter and muon
systems, followed by a software stage, which applies a full event reconstruction.


At the hardware trigger stage, events are required to have a muon with high \pt or a hadron, photon or electron with high transverse energy in the calorimeters. For hadrons, the transverse energy threshold is 3.5\gev. 
Two different algorithms are used in the software trigger to select candidates for this analysis.
The first one uses a multivariate algorithm~\cite{BBDT} to identify the presence of a secondary vertex that has two, three or four tracks and is displaced from any PV. At least one of these charged particles must have a transverse momentum $\pt > 1.7\gevc$ and be inconsistent with originating from a PV. 
The second algorithm selects $\phiz$ candidates decaying to two charged kaons. Each kaon must have a transverse momentum $\pt > 0.8\gevc$ and be inconsistent with originating from a PV. The invariant mass of the kaon pair must be within $20\mevcc$ of the known \phiz mass~\cite{PDG2016}.


\section{Offline selection}
\subsection{Particle identification requirements}
Loose requirements are made on particle identification (PID) variables to reduce the contribution from other types of hadrons and background from other \bquark-hadron decays with misidentified hadrons. 
For the signal, the overall efficiency of the PID requirements varies from 80\% to 90\%, depending on the \Dsp mode.
\subsection{Charmless and single-charm backgrounds}

Background from decays of \Bp mesons to the same final state that did not proceed via a \Dsp meson (referred to as charmless decays) are suppressed by applying a requirement on the significance of the \Bp meson and \Dsp meson vertex separation, $\chi^{2}_{FD}$. 
The residual yields of charmless decays are estimated by fitting the \Bp yields in the sidebands of the \Dsp meson ($25\mevcc < |m(h^{+}h'^{-}\pip) - m(\Dsp)| < 50\mevcc $, where $m(h^{+}h'^{-}\pip)$ is the \Dsp candidate mass and $h,h'=K,\pi$). This background estimation is performed separately for the $\Bp \to \Dsp \phiz$ and $\Bp \to \Dsp \Kp \Km$ searches. For the $\Bp \to \Dsp \Dzb$ normalisation channel, a two-dimensional optimisation is performed to calculate the contribution from decays without a \Dsp meson, \Dzb meson or both. The optimal selection requirements are chosen such that the maximal signal efficiency is achieved for a residual charmless contribution of 2\% of the normalisation yield.

\subsection{Background vetoes}
\subsection{Multivariate analysis}



