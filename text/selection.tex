\chapter{Event selection} 
\label{ch:selection}

\minitoc

In this chapter the procedure developed to reconstruct and select \decay{\Bp}{\Dsp\phiz} and \decay{\Bp}{\Dsp\Kp\Km} candidates is described. 
In both cases the branching fractions are measured relative to the normalisation channel \decay{\Bp}{\Dsp\Dzb}.
The corresponding selection for the normalisation channel \decay{\Bp}{\Dsp\Dzb} is also described.  


\section{Data samples and Simulations}



\subsection{Data samples}
\label{sec:data}
{\color{Red} The searches for \decay{\Bp}{\Dsp\phiz} and \decay{\Bp}{\Dsp\Kp\Km} decays are performed using a data sample collected by the \lhcb experiment.
This data sample includes the entire Run I sample and the first two years of the Run II data collection period.}
The total integrated luminosity obtained for each year is listed in Table~\ref{tab:lumi}, along with the corresponding centre-of-mass energies. 

\begin{table}[h]
   \begin{center}
      \begin{tabular}{ccc}
         \hline
         Year                    & Integrated luminosity (\invfb)  & $\sqrt{s}$ (\tev) \\ 
         \hline
         2011                    & 1.0  &  7 \\
         2012                    & 2.0  &  8 \\
         2015                    & 0.3  & 13 \\
         2016                    & 1.5  & 13 \\
         \hline
      \end{tabular}
   \end{center}
   \caption{The integrated luminosities obtained during the different data taking periods used in this analysis and the corresponding centre-of-mass energies ($\sqrt{s}$).}
   \label{tab:lumi}
\end{table}

\subsection{Simulation samples}
\label{sec:mc}

Simulated events are used to determine the relative efficiencies of the signal and normalisation channels. 
The samples are generated separately for each of the running periods listed in Table~\ref{tab:lumi} and for both polarities of the \lhcb dipole magnet. In these simulations, $pp$ collisions are generated using \pythia~\cite{Sjostrand:2007gs,Sjostrand:2006za} with a specific \lhcb configuration~\cite{LHCb-PROC-2010-056}.  Decays of hadronic particles are described by \evtgen~\cite{Lange:2001uf}, in which final-state radiation is generated using \photos~\cite{Golonka:2005pn}. The interaction of the generated particles with the detector, and its response, are implemented using the \geant toolkit~\cite{Allison:2006ve, *Agostinelli:2002hh} as described in Ref.~\cite{LHCb-PROC-2011-006}.

{\color{Red}
\begin{itemize}
\item Probably need more detail?
\item MC for \decay{\Bp}{\Dsp\phiz} and \decay{\Bp}{\Dsp\Kp\Km}
\item phase-space generation
\item MC for background components
\end{itemize}
}

\section{Online selection}
\label{sec:selection_trigger}
The online event selection is performed by the \lhcb trigger, which consists of a hardware stage, based on information from the calorimeter and muon
systems, followed by a software stage, which reconstructs the full event.
% {\color{Red}
% \begin{itemize}
% \item More relevant information on the trigger rather than jumping straight into TIS/TOS
% \end{itemize}
% }
Trigger decisions are made globally for a given event; either the event is saved or it's discarded. However, for events that are retained more detailed information about the specific interaction that initiated the trigger can be used to help isolate events containing signal candidates. At both hardware and software level there are a number of different triggers that can lead to an event being retained. In hardware this is broadly separated according to the sub-detector that contained the triggering deposit. Deposits in the muon chambers or hadronic calorimeter would fire \texttt{L0Muon} or \texttt{L0Hadron} respectively. Electromagnetic calorimeter deposits could result in \texttt{L0Electron} or \texttt{L0Photon} firing. The \decay{\Bp}{\Dsp\Kp\Km} and \decay{\Bp}{\Dsp}{\phiz} decays are both reconstructed in fully hadronic final states, therefore events retained as a result of the \texttt{L0Electron}, \texttt{L0Photon} or \texttt{L0Muon} trigger firing are unlikely to have been caused by the signal decays. 

In addition to specifying which triggers fired, it is possible to use more granular information about the specific hits initiating the trigger to improve the selection of signal candidates. The tracks constituting the reconstructed candidate are matched to the deposits in each sub-detector. The fraction of hits required to match in order for the trigger decision to be associated to that object vary depending on the subdetector in question, as listed in Table~\ref{tab:tosfrac}.
\begin{table}[h]
   \begin{center}
      \begin{tabular}{cc}
         \hline
         Subdetector    &  Fraction of matching hits \\
         \hline 
         \hcal          & 1\%    \\ 
         \ecal          & 1\%    \\ 
         Muon           & 0.01\% \\ 
         \ttracker      & 0\%    \\ 
         \intr and \ot  & 70\%   \\ 
         \velo          & 70\%   \\ 
         \hline
      \end{tabular}
   \end{center}
   \caption{The fraction of matching hits required for a offline reconstructed object to be associated to a online subdetector deposit.}
   \label{tab:tosfrac}
   %% From: https://gitlab.cern.ch/lhcb/Phys/blob/83e7320119561a4e4996e51a96601f65f1ff5cfe/Phys/TisTosTobbing/src/lib/TisTos.cpp
   %% https://gitlab.cern.ch/lhcb/Phys/blob/2016-patches/Phys/TisTosTobbing/src/lib/TisTos.cpp
\end{table}
Reconstructed objects are classified into categories when considering their relationship to the various triggers that fired in a given event. 
\begin{description}
\item \textbf{Triggered on Signal:} if the deposited matched to a reconstructed object was sufficient to have fired a given trigger then this object is said to be \emph{Triggered on Signal} (\texttt{TOS}) with respect to that trigger (Fig.~\ref{fig:TOS}). As such, if the hits that this object caused were removed from the events the trigger would no longer fire.
\item \textbf{Triggered Independently of Signal:} conversely, if the specific trigger fired solely due to other interactions within the same event the object is \emph{Triggered Independently of Signal} (\texttt{TIS}) with respect to that trigger (Fig.~\ref{fig:TIS}). In this case, removal of hits matched to the object would not affect the trigger decision.  It is possible for a candidate to be both \texttt{TIS} and \texttt{TOS} (Fig.~\ref{fig:TOSandTIS}).
\item \textbf{Triggered on Both:} a third category is possible, \emph{Triggered on Both} (\texttt{TOB}), where both the signal object and another object are required to reach the threshold to fire a trigger (Fig.~\ref{fig:TOB}). In this situation neither are sufficient to individually fire the trigger, but removing either one of them would prevent the trigger firing. Events in this category are not considered in this analysis.
\end{description}

%%%%%%%%%%%%%%%%%%%%%%%%%%%%%%%%%%%%%%%%%%%%%%%%%%%%%%%%%%
\begin{figure}[!h]
    \centering
    \begin{subfigure}[t]{0.4\textwidth}
        \includegraphics[width=1.0\textwidth]{figs/Selection/Tos.pdf}
        \caption{\texttt{TOS}}
        \label{fig:TOS}
    \end{subfigure}%
    \begin{subfigure}[t]{0.4\textwidth}
        \includegraphics[width=1.0\textwidth]{figs/Selection/Tis.pdf}
        \caption{\texttt{TIS}}
        \label{fig:TIS}
    \end{subfigure}\\
    \begin{subfigure}[t]{0.4\textwidth}
        \includegraphics[width=1.0\textwidth]{figs/Selection/Tob.pdf}
        \caption{\texttt{TOB}}
        \label{fig:TOB}
    \end{subfigure}
    \begin{subfigure}[t]{0.4\textwidth}
        \includegraphics[width=1.0\textwidth]{figs/Selection/TisTos.pdf}
        \caption{\texttt{TOS} and \texttt{TIS}}
        \label{fig:TOSandTIS}
    \end{subfigure}\\
    \caption{Simple schematics of the different trigger matching possibilities, illustrated with one signal track and one non-signal track. Cells with a deposit above the trigger threshold are shown in red. }
    \label{fig:tistostobbing}   
\end{figure}
%%%%%%%%%%%%%%%%%%%%%%%%%%%%%%%%%%%%%%%%%%%%%%%%%%%%%%%%%%


At the hardware trigger stage, the selected candidates are required to be \texttt{TOS} with respect to the hadronic trigger \texttt{L0Hadron}. This ensures the selected candidates were retained due to corresponding deposits in the hadronic calorimeter. Alternatively, candidates are selected if they are \texttt{TIS} with respect to the global hardware trigger \texttt{L0Global}. Any of the hardware trigger subsystems can contribute to the \texttt{L0Global} decision. This allows candidates that have been retained due another highly energetic decay in the same event to contribute. This could be the decay of hadron resulting from the other \bquark quark in a \bquark\bquarkbar pair production. 

{\color{Red}
\begin{itemize}
\item Include approximate TIS TOS fractions
\end{itemize}
}

%At the hardware trigger stage, events are required to have a muon with high \pt or a hadron, photon or electron with high transverse energy in the calorimeters. For hadrons, the transverse energy threshold is 3.5\gev. 

The software trigger stage is split into two parts, \hltone and \hlttwo.
The first stage, \hltone, requires that the selected candidates are associated with at least one well reconstructed track. This track must be significantly displaced from the primary interaction and has a large transverse momentum. In Run II, this trigger was reimplementing using a multivariate analysis trained to select a well reconstructed, high transverse momentum track that was significantly displaced from the primary interaction. An additional trigger was also included to select pairs of high transverse momentum tracks that form a vertex that is displaced from the primary interaction. The signal candidate is required to be \texttt{TOS} with respect to these trigger lines such that one of the candidate tracks passes this requirement.   

{\color{Red}
\begin{itemize}
\item Include more details on HltTrackAllL0? 
\item Trigger 2011 paper and Run 2 algorithms
\item add citations
\end{itemize}
}



At the second software stage, \hlttwo, two different algorithms are used to select candidates for this analysis.
The first one uses a multivariate algorithm~\cite{BBDT} to identify the presence of a secondary vertex that has two, three or four tracks and is displaced from any PV. At least one of these charged particles must have a transverse momentum $\pt > 1.7\gevc$ and be inconsistent with originating from a PV. 
The second algorithm selects $\phiz$ candidates decaying to two charged kaons. Each kaon must have a transverse momentum $\pt > 0.8\gevc$ and be inconsistent with originating from a PV. The invariant mass of the kaon pair must be within $20\mevcc$ of the known \phiz mass~\cite{PDG2016}.
The inclusive \phiz line is used to maximise the selection efficiencies as \phiz mesons can contribute directly to \decay{\Bp}{\Dsp\phiz} decays, as well as via the large fractions of \decay{\Dsp}{\phiz\pip} decays that contribute to the \decay{\Dsp}{\Kp\Km\pip} final state.

{\color{Red}
\begin{itemize}
\item fractions of events added by each category
\end{itemize}
}

\section{Offline selection}

Events passing any trigger requirement are saved to tape for processing offline. The stages of the offline reconstruction are detailed in this section.

\subsection{Selection requirements}
\label{sec:selectionrequirements}

The large offline data samples passing the online trigger selection are habitually processed in a procedure know within \lhcb as \emph{Stripping}. This centrally managed processing builds candidates from tracks and neutral calorimeter objects in each event according to a set of predefined \emph{Stripping Lines}. Each line builds a specific candidate decay, applying a loose set of preselection requirements, including kinematic, geometric and invariant mass selections. 

The candidate \decay{\Bp}{\Dsp\phiz} and \decay{\Bp}{\Dsp\Kp\Km} decays are built using almost identical set of requirements due to the similar topologies of these decays. The \phiz meson has a lifetime of $(1.55\pm0.01)\times10^{-22}\sec$~\cite{PDG2016}, therefore the kaon decay products effectively originate at the \Bp decay vertex in a similar way to \decay{\Bp}{\Dsp\Kp\Km} decays.  The normalisation channel, \decay{\Bp}{\Dsp\Dzb} is built similarly, however the difference in lifetime between the \Dzb and \phiz mesons necessitates a slightly different topology. The \Dzb meson has a lifetime of $(4.101\pm0.015)\times10^{-13}\sec$, allowing the meson to travel away from the \Bp decay vertex. Schematics of the signal and normalisation decay topologies are shown in Fig.~\ref{fig:topo}.

%%%%%%%%%%%%%%%%%%%%%%%%%%%%%%%%%%%%%%%%%%%%%%%%%%%%%%%%%%
\begin{figure}[!h]
    \centering
    \begin{subfigure}[t]{0.4\textwidth}
        \includegraphics[width=1.0\textwidth]{figs/Selection/B2DsKK_topology.pdf}
        \caption{Signal decays}
    \end{subfigure}%
    \begin{subfigure}[t]{0.4\textwidth}
        \includegraphics[width=1.0\textwidth]{figs/Selection/B2DsD0_topology.pdf}
        \caption{Normalisation decays}
    \end{subfigure}\\
    \caption{Schematics of the signal and normalisation decay topologies. The collision of the two protons (blue) results in the primary collision vertex (PV). Many promptly produced tracks (red) originate at the PV. In both cases, the long-lived \Bp meson decay vertex is displaced from the PV. The long-lived charm mesons are also displaced from the \Bp meson decay vertex.}
    \label{fig:topo}   
\end{figure}
%%%%%%%%%%%%%%%%%%%%%%%%%%%%%%%%%%%%%%%%%%%%%%%%%%%%%%%%%%

The candidate \decay{\Bp}{\Dsp\Kp\Km} and \decay{\Bp}{\Dsp\phiz} signal decays and \decay{\Bp}{\Dsp\Dzb} normalisation decay are reconstructed in fully hadronic final states. Both the \Dzb and \phiz mesons are reconstructed using pairs of oppositely charged kaons. The branching fractions are $\BF(\decay{\Dzb}{\Kp\Km})= (3.97 \pm 0.07)\times 10^{-3}$ and $\BF(\decay{\phiz}{\Kp\Km})= (48.9 \pm 0.5)\%$ respectively~\cite{PDG2016}. Although the similar two-body hadronic decay \decay{\Dz}{\Km\pip} has a larger branching fraction than mode chosen for the normalisation channel, $\BF(\decay{\Dz}{\Km\pip})= (3.89 \pm 0.04)\%$, sharing the same final state helps to reduced systematic uncertainties in the ratio of selection efficiencies. Many differences in how pions and kaons interact with the detector can be neglected as they would affect the signal and normalisation channel in the same way.
The \Dsp mesons used in the search for \decay{\Bp}{\Dsp\Kp\Km} decays are reconstructed using the \decay{\Dsp}{\Kp\Km\pip} decay. The search for \decay{\Bp}{\Dsp\phiz} additionally includes the modes \decay{\Dsp}{\pip\pim\pip} and \decay{\Dsp}{\Kp\pim\pip} to increase the sensitivity of the search. The branching fractions for these decays are listed in Table~\ref{tab:dsbranchingfractions}. In each instance the normalisation channel is reconstructed using the same \Dsp decay mode as the signal.  


\begin{table}[h]
   \begin{center}
      \begin{tabular}{lc}
         \hline
         Decay                         &  Branching fraction \\
         \hline 
         \decay{\Dsp}{\Kp\Km\pip}      & $5.45 \pm 0.17 \%$ \\
         \decay{\Dsp}{\pip\pim\pip}    & $1.09 \pm 0.05 \%$ \\
         \decay{\Dsp}{\Kp\pim\pip}     & $0.66 \pm 0.04 \%$ \\
         \hline
      \end{tabular}
   \end{center}
   \caption{The branching fractions for the different \Dsp decay modes used to reconstruct the signal and normalisation modes.}
   \label{tab:dsbranchingfractions}
\end{table}

%The searches for \decay{\Bp}{\Dsp\phiz} and \decay{\Bp}{\Dsp\Kp\Km} decays employ the use of two different \emph{Stripping Lines} as detailed in Table~\ref{tab:strippinglines}. These differ only in the invariant mass window applied to the $\Kp\Km$ pair used to reconstruct the $\phi$ meson, and in the number of \Dsp decay modes included: the \decay{\Bp}{\Dsp\Kp\Km} line only reconstructs the Cabibbo Favoured (CF) \decay{\Dsp}{\Kp\Km\pip} decay. The \emph{Stripping Line} used to select the normalisation channel \decay{\Bp}{\Dsp\Dzb} is also included in Table~\ref{tab:strippinglines}. This has slightly different requirements allowing the \Dzb meson decay vertex to be be displaced from the \Bp meson decay vertex.


The selection of candidates makes requirements on many different distinguishing characteristics. The definitions of the relevant quantities are as follows:  
\begin{description}
\item \textbf{Mass, $m(X)$:} the invariant mass of the indicated particle or combination of particles. 
\item \textbf{Momentum, \ptot:} the total momentum of the particle or combination of particles.
\item \textbf{Transverse momentum, \pt:} The projection of the total momentum, \ptot, on the plane perpendicular to the proton beam axis.
\item \textbf{Lifetime, $\tau$:} the time taken for the candidate to travel from its production vertex to its decay vertex.
\item \textbf{Products \pt scalar sum, $\sum{|\pt|}$:} the sum of the magnitudes of the transverse momentum for each of the decay products.
\item \textbf{Vertex quality, $\chi^{2}_{\text{VTX}}$:} the minimised value of $\chi^{2}/N_{\text{DOF}}$ as determined in the fit to the vertex location.
\item \textbf{Track quality, $\chi^{2}_{\text{TRK}}$:} the minimised value of $\chi^{2}/N_{\text{DOF}}$ as determined in the fit track hits.
\item \textbf{Impact parameter, $\text{IP}$:} The shortest distance between a given track and a vertex location, as shown in Fig.~\ref{fig:impact_parameter}. 

%%%%%%%%%%%%%%%%%%%%%%%%%%%%%%%%%%%%%%%%%%%%%%%%%%%%%%%%%%
\begin{figure}[!h]
    \centering
    \includegraphics[width=0.4\textwidth]{figs/Selection/Impact_parameter.pdf}
    \caption{Impact parameter (dotted black line) between a track (black) and a vertex (red lines).}
    \label{fig:impact_parameter}   
\end{figure}
%%%%%%%%%%%%%%%%%%%%%%%%%%%%%%%%%%%%%%%%%%%%%%%%%%%%%%%%%%

\item \textbf{Impact parameter significance, $\chi^{2}_{\text{IP}}$:} The difference of a given vertex's $\chi^{2}/N_{\text{DOF}}$ with and without a specific track included in the fitting procedure.
\item \textbf{Flight distance significance, $\chi^{2}_{\text{FD}}$:} A measure of how significant the flight distance of a combination of particle is. Defined as the $\chi^{2}$ associated with the difference in position of the two vertices, $\vec{\mathbf{d}} = \vec{\mathbf{v}}_2 - \vec{\mathbf{v}}_1$, where $\vec{\mathbf{v}}_1$ and $\vec{\mathbf{v}}_1$ are positions of the first and the second vertices. 
\item \textbf{Distance of closest approach, $\text{DOCA}(h,h')$:} The shortest distance between two tracks as shown in Fig~\ref{fig:doca}.
%%%%%%%%%%%%%%%%%%%%%%%%%%%%%%%%%%%%%%%%%%%%%%%%%%%%%%%%%%
\begin{figure}[!h]
    \centering
    \includegraphics[width=0.4\textwidth]{figs/Selection/DOCA.pdf}
    \caption{Distance of closest approach (dotted black line) between two tracks (black).}
    \label{fig:doca}   
\end{figure}
%%%%%%%%%%%%%%%%%%%%%%%%%%%%%%%%%%%%%%%%%%%%%%%%%%%%%%%%%%


\item \textbf{Ghost track probability, $P_{\text{Ghost}}$:} {\color{Red}Look up definition}
\item \textbf{Direction angle:} The angle between the particles momentum vector and the vector connecting the primary vertex and decay vertex as shown in Fig.~\ref{fig:dira}.

%%%%%%%%%%%%%%%%%%%%%%%%%%%%%%%%%%%%%%%%%%%%%%%%%%%%%%%%%%
\begin{figure}[!h]
    \centering
    \includegraphics[width=0.4\textwidth]{figs/Selection/DIRA.pdf}
    \caption{Direction angle, $\theta$, between the line joining two vertices, primary vetex (PV) and secondary vertex (SV), and the momentum vector shown in green. The momentum vector corresponds to the momentum of the three tracks contributing to the SV. }
    \label{fig:dira}   
\end{figure}
%%%%%%%%%%%%%%%%%%%%%%%%%%%%%%%%%%%%%%%%%%%%%%%%%%%%%%%%%%


\item \textbf{Particle identification, $\text{PIDK}$:} {\color{red} Something }  
\end{description}


As the final states are fully hadronic, the candidates are built from the combination of five tracks. Only \emph{long} tracks (those with hits in the \velo and tracking stations) are used to build these mesons. The track $\chi^{2}/N_{\text{DOF}}$ is required to be below $4.0$ to ensure these are well reconstructed. Additionally, they are required to have a total momentum $\ptot > 1000 \mevc$ and the transverse momentum is required to be $\pt > 100 \mevc$.
Due to the relatively long lifetime of the \Bp and \D mesons, the decay products originate from a vertex that is displaced from the proton-proton collision vertex. Therefore it is possible that the trajectory that the decay products followed didn't pass through the collision position as shown in Fig.~\ref{fig:impact_parameter}. A requirement is placed on the significance of the impact parameter between the track and the proton-proton collision vertex of $\chi^{2}_{\text{IP}} > 4$ to ensure all of the tracks used are inconsistent with originating at the primary interaction.  


Loose requirements are placed on \emph{Particle Identification} variables to ensure the tracks are of the required species. These are further tightened as detailed in Section~\ref{sec:pidrequirements}. Incorrect track candidates created by combining unrelated \velo and tracking station {\color{red}stubs} are suppressed by requiring the ghost track probability, $P_{\text{Ghost}}$, to be less than $0.4$. 

%%%%%% Ds/ D0 and phi construction

The tracks passing these requirements are combined in pairs or threes to form the \Dsp and \phiz or \Dzb meson candidates.  
To create the \phiz candidates (or \Kp\Km pair), all combinations of two tracks are considered. The two tracks are assigned the mass hypothesis of kaons. Only pairs which pass within $0.5\mm$ of one another at their closest point are retained. A number of requirements are imposed to ensure the pair are consistent with coming from a \phiz meson originating from a \Bp meson decay. The vertex is required to be of good quality, the scalar sum of the transverse momentums must be greater than $1000\mevc$ and the flight distance significance is required to be $\chi^{2}_{\text{FD} } > 16$. Additionally the cosine of the direction angle, defined to be the angle between the momentum vector and flight vector, is required to be $\cos{\theta}>0$, preventing the vertex being backward of the PV when the momentum is in the forward direction. In the search for \decay{\Bp}{\Dsp\phiz} mesons, the invariant mass of the two kaons is required to be within $150\mevcc$ of the known \phiz meson mass.  

The \Dzb candidates are selecting using a similar parameters, however the exact values of the cuts are changed to reflect the differences in properties of the \phiz and \Dzb mesons.


% & Products \pt scalar sum        &  $\sum{|\pt|} > 1000 \mevc$         \\  
% & Direction angle                &  $\cos{\theta}>0$                 \\  
% & Vertex quality                 &  $\chi^{2}/N_{\text{DOF}} < 16$   \\   
% & Flight distance significance   &  $\chi^{2}_{\text{FD} }  > 16$    \\


% \Dzb           & Mass                           &  $1765 < m(h^{+}h^{-}h^{+}) < 1965\mevcc$\\  
%                & Products \pt scalar sum        &  $\sum{|\pt|} > 1800 \mevc$         \\  
%                & Distance of closest approach   &  $\text{DOCA}(\Kp,\Km) < 0.5\mm$  \\  
%                & Direction angle                &  $\cos{\theta}>0$                 \\  
%                & Vertex quality                 &  $\chi^{2}/N_{\text{DOF}} < 10$   \\   
%                & Flight distance significance   &  $\chi^{2}_{\text{FD} }  > 36$    \\ 

The \Dsp mesons for the signal and normalisation modes are created using combinations of three tracks given the mass hypotheses kaons or pions, depending on the decay mode being constructed. All combinations of three tracks are considered {\color{Red} maybe with the exception of ones used in other candidates?}. 


% \Dsp           & Mass                           &  $1770 < m(h^{+}h^{-}h^{+}) < 2068\mevcc$            \\  
%                & Products \pt scalar sum        &  $\sum{|\pt|} > 1800 \mevc$         \\ 
%                & Distance of closest approach   &  $\text{DOCA}(h^{+},h^{-}) < 0.5\mm$     \\  
%                & Distance of closest approach   &  $\text{DOCA}(h^{+},h^{+}) < 0.5\mm$     \\  
%                & Direction angle                &  $\cos{\theta}>0$                 \\  
%                & Vertex quality                 &  $\chi^{2}/N_{\text{DOF}} < 10$   \\   
%                & Flight distance significance   &  $\chi^{2}_{\text{FD} }  > 36$    \\ 


%% B meson construction

The \Bp meson candidates are constructed by creating all possible combinations of the \Dsp and \phiz or \Dzb meson candidates in each event.
Requirements are placed on these combination to select just those consistent with \Bp mesons.


{\color{Red}
\begin{itemize}
\item something about how combineparticles works?
\item \Dsp and \phiz construction 
\item \Bp construction
\end{itemize}
}


The selection requirements imposed on candidate \decay{\Bp}{\Dsp\phiz} and \decay{\Bp}{\Dsp\Kp\Km} decays in their respective \emph{Stripping Lines} is detailed in Table~\ref{tab:strippinglinecuts}. 


\begin{table}[h]
   \begin{center}
      \begin{tabular}{l l}
         \hline
         Mode & Stripping line \\ 
         \hline
         \decay{\Bp}{\Dsp\phiz}        & \texttt{StrippingB2DPhiD2HHHPIDBeauty2CharmLine}    \\
         \decay{\Bp}{\Dsp\Kp\Km}       & \texttt{StrippingB2DKKD2HHHCFPIDBeauty2CharmLine}   \\
         \decay{\Bp}{\Dsp\Dzb}         & \texttt{StrippingB2D0DBeauty2CharmLine}             \\
         \hline
      \end{tabular}
   \end{center}
   \caption{\emph{Stripping Lines} used in this analysis.}
   \label{tab:strippinglines}
\end{table}





\begin{table}[h]
\begin{center}
\begin{tabular}{ l l l}
\hline
Particles      & Quantity                       & Requirement                       \\ 
\hline
\Bp            & Mass                           &  $4750 < m(\Dsp\phiz) < 7000\mevcc$    \\  
%               & Transverse Momentum            &  $\pt > 4000 \gevc$               \\  
               & Products \pt scalar sum        &  $\sum{|\pt|} > 5000 \mevc$         \\  
               & Vertex quality                 &  $\chi^{2}/N_{\text{DOF}} < 10$   \\  
               & Lifetime                       &  $\tau_{\Bp} > 0.2\ps$            \\  
               & Impact parameter significance  &  $\chi^{2}_{\text{IP}} < 25$      \\  
               & Direction angle                &  $\cos{\theta}>0.999$             \\  
               & \textit{$>$0 decay products with:}    &                                   \\
               & Momentum                       &  $\ptot > 10000 \mevc$            \\  
               & Transverse momentum            &  $\pt > 1700 \mevc$               \\  
               & Impact parameter significance  &  $\chi^{2}_{\text{IP}} > 16$      \\  
               & Impact parameter               &  $\text{IP} > 0.1\mm$             \\  
               & \textit{$>$1 decay products with:}   &                                   \\
               & Momentum                       &  $\ptot > 5000 \mevc$             \\  
               & Transverse momentum            &  $\pt > 500 \mevc$                \\
%               &                                &                                   \\
\hline  
\Dsp           & Mass                           &  $1770 < m(h^{+}h^{-}h^{+}) < 2068\mevcc$            \\  
               & Products \pt scalar sum        &  $\sum{|\pt|} > 1800 \mevc$         \\ 
               & Distance of closest approach   &  $\text{DOCA}(h^{+},h^{-}) < 0.5\mm$     \\
               & Distance of closest approach   &  $\text{DOCA}(h^{-},h'^{+}) < 0.5\mm$     \\    
               & Distance of closest approach   &  $\text{DOCA}(h^{+},h'^{+}) < 0.5\mm$     \\  
               & Direction angle                &  $\cos{\theta}>0$                 \\  
               & Vertex quality                 &  $\chi^{2}/N_{\text{DOF}} < 10$   \\   
               & Flight distance significance   &  $\chi^{2}_{\text{FD} }  > 36$    \\   
%               &                                &                                   \\  
\hline
\phiz          & Mass (only for \decay{\Bp}{\Dsp\phiz})&  $|m(\Kp\Km)-m_{\phiz}| < 150\mevcc$\\ 
               & Products \pt scalar sum        &  $\sum{|\pt|} > 1000 \mevc$         \\  
               & Distance of closest approach   &  $\text{DOCA}(\Kp,\Km) < 0.5\mm$  \\  
               & Direction angle                &  $\cos{\theta}>0$                 \\  
               & Vertex quality                 &  $\chi^{2}/N_{\text{DOF}} < 16$   \\   
               & Flight distance significance   &  $\chi^{2}_{\text{FD} }  > 16$    \\   
%               &                                &                                   \\  
\hline
\Kpm (\pipm)   & Track quality                  &  $\chi^{2}/N_{\text{DOF}}<4.0$    \\  
               & Transverse momentum            &  $\pt > 100 \mevc$                \\  
               & Momentum                       &  $\ptot > 1000 \mevc$             \\  
               & Impact parameter significance  &  $\chi^{2}_{\text{IP}} > 4$       \\  
               & Ghost track probability        &  $P_{\text{Ghost}} < 0.4$         \\
               & Particle identification        &  $\text{PIDK}>-10$ ($\text{PIDK}<20$)\\  
\hline
\end{tabular}
\end{center}
\caption{Selection requirements for \decay{\Bp}{\Dsp\phiz} and \decay{\Bp}{\Dsp\Kp\Km} candidates.}
\label{tab:strippinglinecuts}
\end{table}

%%%%%%%%%%%%%%%%%%%%%% DONE %%%%%%%%%%%%%%%%%%%%%%
%B CombCut
%(
% ASUM(
%       SUMTREE(
%                PT,(
%                      ISBASIC | 
%                      (ID=='gamma')
%                   )
%                ,0.0
%             )
%       )>5000*MeV) & 
% (AM<7000*MeV) & 
% (AM>4750*MeV)

% B MotherCut

% (VFASPF(VCHI2/VDOF)<10) &
%(BPVLTIME()>0.2*ps) & 
%(BPVIPCHI2()<25) & 
%(BPVDIRA>0.999)
% (INTREE(
%          HASTRACK & 
%          (P>10000*MeV) & 
%          (PT>1700*MeV) & 
%          (TRCHI2DOF<4.) & 
%          (MIPCHI2DV(PRIMARY)>16) & 
%          (MIPDV(PRIMARY)>0.1*mm) )) & 
% (NINTREE(
%          (
%             ISBASIC & 
%             HASTRACK & 
%             (TRCHI2DOF<4.) & 
%             (PT > 500*MeV) & 
%             (P > 5000*MeV)
%          ) > 1
% )


%X2PiPi
%(ASUM(PT)>1000*MeV) & 
% (AM < 5.2*GeV) & 
% (AHASCHILD(
%             (
%                ISBASIC & 
%                HASTRACK & 
%                (TRCHI2DOF<4.) & 
%                (PT > 500*MeV) & 
%                (P > 5000*MeV)
%             )
%          )
% ) & 
% (ADOCA(1,2)<0.5*mm)

%ADMASS('phi(1020)') < 150*MeV

%PiInput
%(TRCHI2DOF<4.0) & 
% (PT>100*MeV) & 
% (P>1000*MeV) & 
% (MIPCHI2DV(PRIMARY)>4.0) & 
% (TRGHP<0.4)


%D2HHHFilter
%
% (NINGENERATION(
%                ('p+'==ABSID) & 
%                (PIDp < -10),1
%                ) == 0
% ) & 
% (NINGENERATION(   
%                ('K+'==ABSID) & 
%                (PIDK < -10)
%                , 1) == 0
% ) & 
% (NINGENERATION(
%                ('pi+'==ABSID) & 
%                (PIDK > 20)
%                , 1) == 0
% )

% D2HHH CombCut
% (ASUM(PT)>1800*MeV) & 
% (in_range(1769.62*MeV,AWM('K+','K+','pi-'),2068.49*MeV)) & 
% (AHASCHILD(
              
%             ISBASIC & 
%             HASTRACK & 
%             (TRCHI2DOF<4.) & 
%             (PT > 500*MeV) & 
%             (P > 5000*MeV)          
%          )
% ) & 
% (ADOCA(1,3)<0.5*mm) & 
% (ADOCA(2,3)<0.5*mm)

% D2HHH MotherCut
% (VFASPF(VCHI2/VDOF)<10) & 
% (BPVVDCHI2>36) & 
% (BPVDIRA>0)

% PhiMotherCut
% (VFASPF(VCHI2/VDOF)<16) & 
% (BPVVDCHI2>16) & 
% (BPVDIRA>0)


%%%%%%%%%%%%%%%%%%%%%%%%%%%%%%%%%%%%%%%%%%%%%%%%%


%HHPionsInput
%(PT>100*MeV) & (P>2000*MeV)


Two slightly different strategies are used for the normalisation channel selection in the search for \decay{\Bp}{\Dsp\phiz} and \decay{\Bp}{\Dsp\Kp\Km} events.
In the former, the dedicated \decay{\Bp}{\Dsp\Dzb} \emph{Stripping Line} listed in Table~\ref{tab:strippinglines} is used to reconstruct the normalisation channel decays.
The \emph{Stripping Line} selection for this line is listed in Table~\ref{tab:strippinglinecuts_norm}.

The \emph{Stripping Line} used in the search for \decay{\Bp}{\Dsp\Kp\Km} decays covers the full $m(\Kp\Km)$ phasespace. This includes the \Dzb mass such that this line reconstructs both the signal and normalisation channels simultaneously. 
Both modes are selected using this line to reduce systematic uncertainty in the ratio of selection efficiencies.

\begin{table}[h]
\begin{center}
\begin{tabular}{ l l l}
\hline
Particles      & Quantity                       & Requirement                       \\ 
\hline
\Bp            & Mass                           &  $4750 < m(\Dsp\Dzb) < 7000\mevcc$    \\ 
               & Products \pt scalar sum        &  $\sum{|\pt|} > 5000 \mevc$         \\  
               & Vertex quality                 &  $\chi^{2}/N_{\text{DOF}} < 10$   \\  
               & Lifetime                       &  $\tau_{\Bp} > 0.2\ps$            \\  
               & Impact parameter significance  &  $\chi^{2}_{\text{IP}} < 25$      \\  
               & Direction angle                &  $\cos{\theta}>0.999$             \\  
               & \textit{$>$0 decay products with:}    &                                   \\
               & Momentum                       &  $\ptot > 10000 \mevc$            \\  
               & Transverse momentum            &  $\pt > 1700 \mevc$               \\  
               & Impact parameter significance  &  $\chi^{2}_{\text{IP}} > 16$      \\  
               & Impact parameter               &  $\text{IP} > 0.1\mm$             \\  
               & \textit{$>$1 decay products with:}   &                                   \\
               & Momentum                       &  $\ptot > 5000 \mevc$             \\  
               & Transverse momentum            &  $\pt > 500 \mevc$                \\
%               &                                &                                   \\  
\hline
\Dsp           & Mass                           &  $1770 < m(h^{+}h^{-}h^{+}) < 2068\mevcc$            \\  
               & Products \pt scalar sum        &  $\sum{|\pt|} > 1800 \mevc$         \\ 
               & Distance of closest approach   &  $\text{DOCA}(h^{+},h^{-}) < 0.5\mm$     \\
               & Distance of closest approach   &  $\text{DOCA}(h^{-},h'^{+}) < 0.5\mm$     \\    
               & Distance of closest approach   &  $\text{DOCA}(h^{+},h'^{+}) < 0.5\mm$     \\   
               & Direction angle                &  $\cos{\theta}>0$                 \\  
               & Vertex quality                 &  $\chi^{2}/N_{\text{DOF}} < 10$   \\   
               & Flight distance significance   &  $\chi^{2}_{\text{FD} }  > 36$    \\   
%               &                                &                                   \\  
\hline
\Dzb           & Mass                           &  $1765 < m(h^{+}h^{-}h^{+}) < 1965\mevcc$\\  
               & Products \pt scalar sum        &  $\sum{|\pt|} > 1800 \mevc$         \\  
               & Distance of closest approach   &  $\text{DOCA}(\Kp,\Km) < 0.5\mm$  \\  
               & Direction angle                &  $\cos{\theta}>0$                 \\  
               & Vertex quality                 &  $\chi^{2}/N_{\text{DOF}} < 10$   \\   
               & Flight distance significance   &  $\chi^{2}_{\text{FD} }  > 36$    \\   
%               &                                &                                   \\  
\hline
\Kpm (\pipm)   & Track quality                  &  $\chi^{2}/N_{\text{DOF}}<4.0$    \\  
               & Transverse momentum            &  $\pt > 100 \mevc$                \\  
               & Momentum                       &  $\ptot > 1000 \mevc$             \\  
               & Impact parameter significance  &  $\chi^{2}_{\text{IP}} > 4$       \\  
               & Ghost track probability        &  $P_{\text{Ghost}} < 0.4$         \\
               & Particle identification        &  $\text{PIDK}>-10$ ($\text{PIDK}<20$)\\
\hline
\end{tabular}
\end{center}
\caption{Selection requirements for \decay{\Bp}{\Dsp\Dzb} candidates.}

\label{tab:strippinglinecuts_norm}
\end{table}



% D2KKPi Mother
% (VFASPF(VCHI2/VDOF)<10) & (BPVVDCHI2>36) & (BPVDIRA>0)



% D2KKPi Comb12 
% (ADOCA(1,2)<0.5*mm)


% D2KKPi Comb
% 
% (ASUM(PT)>1800*MeV) & 
% (in_range(1769.62*MeV,AWM('K-','K-','pi+'),2068.49*MeV)) & 
% (AHASCHILD(
%             ISBASIC & 
%             HASTRACK & 
%             (TRCHI2DOF<4.) & 
%             (PT > 500*MeV) & 
%             (P > 5000*MeV)  
%          )  
% ) & 
% (ADOCA(1,3)<0.5*mm) & 
% (ADOCA(2,3)<0.5*mm)



% K input 
% (TRCHI2DOF<4.0) & (PT>100*MeV) & (P>1000*MeV) & (MIPCHI2DV(PRIMARY)>4.0) & (TRGHP<0.4)


% D2KK Mother
% 
% (VFASPF(VCHI2/VDOF)<10) & 
% (BPVVDCHI2>36) & 
% (BPVDIRA>0)




% D2KK Comb
% 
% (ASUM(PT)>1800*MeV) & 
% (in_range(1764.84*MeV,AWM('K+','K-'),1964.84*MeV)) & 
% (AHASCHILD(
%          ISBASIC & 
%          HASTRACK & 
%          (TRCHI2DOF<4.) & 
%          (PT > 500*MeV) & 
%          (P > 5000*MeV)    
%       )
% ) & 
% (ADOCA(1,2)<0.5*mm)

% D2HH
% (NINGENERATION(('p+'==ABSID) & (PIDp < -10),1) == 0) & (NINGENERATION(('K+'==ABSID) & (PIDK < -10), 1) == 0) & (NINGENERATION(('pi+'==ABSID) & (PIDK > 20), 1) == 0)


% B mother cut
% (VFASPF(VCHI2/VDOF)<10) & 
% (
%    INTREE(HASTRACK & 
%    (P>10000*MeV) & 
%    (PT>1700*MeV) & 
%    (TRCHI2DOF<4.) & 
%    (MIPCHI2DV(PRIMARY)>16) & 
%    (MIPDV(PRIMARY)>0.1*mm))
% ) & 
% (
%    NINTREE( 
%             ISBASIC & 
%             HASTRACK & 
%             (TRCHI2DOF<4.) & 
%             (PT > 500*MeV) & 
%             (P > 5000*MeV)     
%          ) > 1
% ) & 
% (BPVLTIME()>0.2*ps) & 
% (BPVIPCHI2()<25) & 
% (BPVDIRA>0.999)


% B comb
% (ASUM(
%    SUMTREE(PT,ISBASIC,0.0) )>5000*MeV
% ) & 
% (AM<7000*MeV) & 
% (AM>4750*MeV)

%%%%%%%%%%%%% done %%%%%%%%%%%


\clearpage


\subsection{Particle identification requirements}
\label{sec:pidrequirements}
Particle identification variables help to determine the species of tracks passing though the \lhcb detector. Using information from the RICH sub-detectors, the likelihood of different mass hypotheses are compared to the pion hypothesis. Loose requirements are made on the kaon hypothesis PID variable to reduce the contribution from other types of hadrons and background from other \bquark-hadron decays with misidentified hadrons. 
%For the signal decays, the overall efficiency of the PID requirements varies from 80\% to 90\%, depending on the \Dsp mode.

% if(d_Phi_K0_PIDK < 0 ) continue;
% if(d_Phi_K1_PIDK < 0 ) continue;

% if(d_mode == Ds2KKPi){
%     if(d_D_K0_PIDK < -5) continue;
%     if(d_D_K1_PIDK < -5) continue;
%     if(d_D_P_PIDK  >  5) continue;

% } else if(d_mode == Ds2KPiPi){
%     if(d_D_K_PIDK  < -5) continue; 
%     if(d_D_P0_PIDK >  5) continue; 
%     if(d_D_P1_PIDK >  5) continue;     
% } else if(d_mode == Ds2PiPiPi){
%     if(d_D_P0_PIDK >  5) continue; 
%     if(d_D_P1_PIDK >  5) continue;
%     if(d_D_P2_PIDK >  5) continue;



\begin{table}[h]
   \begin{center}
      \begin{tabular}{l l l}
         \hline
         Decay mode & Species & PID requirement\\ 
         \hline
         \decay{\phiz}{\Kp\Km}      & \Kp    & $\text{PIDK} > 0$  \\
                                    & \Km    & $\text{PIDK} > 0$  \\
         \hline
         \decay{\Dzb}{\Kp\Km}       & \Kp    & $\text{PIDK} > 0$  \\
                                    & \Km    & $\text{PIDK} > 0$  \\
         \hline
         \decay{\Dsp}{\Kp\Km\pip}   & \Kp    & $\text{PIDK} > -5$ \\
                                    & \Km    & $\text{PIDK} > -5$ \\
                                    & \pip   & $\text{PIDK} < 5$  \\
         \hline
         \decay{\Dsp}{\pip\pim\pip} & \pip   & $\text{PIDK} < 5$  \\
                                    & \pim   & $\text{PIDK} < 5$  \\
                                    & \pip   & $\text{PIDK} < 5$  \\
         \hline
         \decay{\Dsp}{\Kp\pim\pip}  & \Kp    & $\text{PIDK} > -5$ \\
                                    & \pim   & $\text{PIDK} < 5$  \\
                                    & \pip   & $\text{PIDK} < 5$  \\
         \hline
      \end{tabular}
   \end{center}
   \caption{Particle idetification requirements applied to kaons and pions.}
   \label{tab:strippinglines}
\end{table}



{\color{Red}
\begin{itemize}
\item Include actual cuts
\item description of what DLLs mean
\end{itemize}
}

\subsection{Charmless and single-charm backgrounds}


Decays of \Bp mesons that didn't proceed via \D mesons could form a peaking background below the signal invariant mass distributions when they decay to the same final state.
The signal mode could receive contributions from the decays $\decay{\Bp}{h^{+}h^{-}h^{+}\phiz}$ or $\decay{\Bp}{h^{+}h^{-}h^{+}h^{+}h^{-}}$, referred to as charmless backgrounds. Here $h^{\pm}$ is used to represent \Kpm or \pipm in the specific \Dsp, \Dzb or \phiz final state.
The normalisation mode is also susceptible, however as it involves two charm mesons it could receive contributions from the decays $\decay{\Bp}{h^{+}h^{-}h^{+}\Dzb}$ or $\decay{\Bp}{\Dsp h^{+}h^{-}}$, referred to as single-charm backgrounds, and $\decay{\Bp}{h^{+}h^{-}h^{+} h^{+}h^{-}}$ referred to as a charmless background.
These backgrounds can be suppressed by requiring the \D meson decay vertex to be displaced from the \Bp meson decay vertex. Requirements are applied to the significance of the vertex separation ($\chi^{2}_{\text{FD}}$).

The residual yields of charmless backgrounds in the signal mode are estimated by performing a fit to the \Bp invariant mass for candidates with $25 < |m(h^{+}h^{-}h^{+}) - m(\Dsp)| < 50\mevcc $. This background estimation is performed separately for the \decay{\Bp}{\Dsp\phiz} and \decay{\Bp}{\Dsp\Kp\Km} searches. 

For the \decay{\Bp}{\Dsp\Dzb} normalisation channel, a two-dimensional optimisation is performed to calculate the contribution from decays without a \Dsp meson, \Dzb meson or both. 
The two-dimensional space defined by the \Dsp and \Dzb masses is split into four types of area as shown in Fig~\ref{fig:2d_normalisation}.

%%%%%%%%%%%%%%%%%%%%%%%%%%%%%%%%%%%%%%%%%%%%%%%%%%%%%%%%%%
\begin{figure}[!h]
    \centering
        \includegraphics[width=0.6\textwidth]{figs/Selection/B2DsD0_2D_mass_Ds2KKPiRun2.pdf}
        \caption{Two dimensional normalisation}
    \label{fig:2d_normalisation}   
\end{figure}
%%%%%%%%%%%%%%%%%%%%%%%%%%%%%%%%%%%%%%%%%%%%%%%%%%%%%%%%%%

\begin{enumerate}
\item Areas in which only $\decay{\Bp}{h^{+}h^{-}h^{+}h^{+}h^{-}}$ decays contribute (red).
\item Areas in which either $\decay{\Bp}{D_{s}^{+}h^{+}h^{-}}$  or $\decay{\Bp}{h^{+}h^{-}h^{+} h^{+}h^{-}}$ decays can contribute (blue). 

\item Areas in which either $\decay{\Bp}{h^{+}h^{-}h^{+}\Dzb}$ or $\decay{\Bp}{h^{+}h^{-}h^{+} h^{+}h^{-}}$ decays can contribute (green). 
\item The signal region in which $\decay{\Bp}{D_{s}^{+} \Dzb}$, $\decay{\Bp}{h^{+}h^{-}h^{+}\Dzb}$, $\decay{\Bp}{D_{s}^{+}h^{+}h^{-}}$ or $\decay{\Bp}{h^{+}h^{-}h^{+}h^{+}h^{-}}$ decays could contribute (black).
\end{enumerate}   

Asymmetric \Dzb sidebands are used to prevent misidentified \decay{\Bp}{\Dsp (\decay{\Dzb}{\Km\pip})} decays from being included in the sideband sample.
The optimal selection requirements are chosen such that the maximal signal efficiency is achieved for a residual charmless contribution of 2\% of the normalisation yield.


{\color{Red}
\begin{itemize}
\item Optmised separately for \decay{\Bp}{\Dsp\Kp\Km} and \decay{\Bp}{\Dsp\phiz}
\item Include plots of selected cuts
\item include table of expected residual yields 
\end{itemize}
}


\subsection{Misidentified \D and \Lc hadrons}
\label{sec:pidvetos}

It is possible for the samples \Dsp mesons to be contaminated by other misidentified decays of \Dp mesons or \Lc baryons in which one of the decay products has been incorrectly identified.
The invariant mass of the \Dsp meson is recalculated, swapping the mass hypothesis of the ambiguous track to that of the \kaon, \pion or \proton, depending on the decay mode. 
The particle identification requirements are tightened within a mass window around the \Dp or \Lc mass, effectively removing this crossfeed. For the mode \decay{\Dsp}{\Kp\Km\pip}, the vetoes are not applied to candidates for which $m|(\Km\Kp)-m_{\phiz}| < 10\mevcc$ as there are a high purity of \decay{\Dsp}{\Kp\Km\pip} decays in this region.

The specific vetoes included in this selection are listed in Table~\ref{table:pidvetos}. 
\begin{table*}[!ht]
\begin{center}
\begin{tabular}{ l l l }
\hline
Decay Mode & Misidentified decay\\
\hline
\decay{\Dsp}{{\color{Red}\Kp}\Km\pip}   & \decay{\Dp}{{\color{Red}\pip}\Km\pip}    \\
                           & \decay{\Lc}{{\color{Red}\Pp}\Km\pip}     \\
%                           &                             \\
\hline
\decay{\Dsp}{{\color{Red}\Kp}\pim\pip}  & \decay{\Dp}{{\color{Red}\pip}\pim\pip}   \\
%                           &                             \\

\hline
\end{tabular}
\caption{Misidentified decays targeted by vetoes. The ambiguous track is highlighted in red in each case.}
\label{table:pidvetos}
\end{center}
\end{table*}
The invariant mass distributions for each the misidentified \decay{\Dsp}{\Kp\Km\pip} decays are shown with and without the MVA requirements in Figs.~\ref{fig:PIDVetos_Ds2KKPi_D_Veto} and \ref{fig:PIDVetos_Ds2KKPi_Lc_Veto} for both the signal \decay{\Bp}{\Dsp\phiz} and normalisation \decay{\Bp}{\Dsp\Dzb} decays.



%%%%%%%%%%%%%%%%%%%%%%%%%%%%%%%%%%%%%%%%%%%%%%%%%%%%%%%%%%
\begin{figure}[!h]
    \centering
    \begin{subfigure}[t]{0.4\textwidth}
        \includegraphics[width=1.0\textwidth]{figs/Selection/B2DsD0_Ds2KKPi_D_Veto_NoBDT.pdf}
        \caption{Normalisation without selection}
    \end{subfigure}%
    \begin{subfigure}[t]{0.4\textwidth}
        \includegraphics[width=1.0\textwidth]{figs/Selection/B2DsPhi_Ds2KKPi_D_Veto_NoBDT.pdf}
        \caption{Signal without selection}
    \end{subfigure}\\
    \begin{subfigure}[t]{0.4\textwidth}
        \includegraphics[width=1.0\textwidth]{figs/Selection/B2DsD0_Ds2KKPi_D_Veto_WithBDT.pdf}
        \caption{Normalisation with selection}
    \end{subfigure}%
    \begin{subfigure}[t]{0.4\textwidth}
        \includegraphics[width=1.0\textwidth]{figs/Selection/B2DsPhi_Ds2KKPi_D_Veto_WithBDT.pdf}
        \caption{Signal with selection}
    \end{subfigure}\\
    \caption{Invariant mass distributions of \decay{\Dsp}{\Kp\Km\pip} samples reconstructed as \decay{\Dp}{\pip\Km\pip} for the signal and normalisation samples. The samples are shown with (red) and without (black) the veto described in Sec.~\ref{sec:pidvetos}. The distributions are shown before (top) and after (bottom) the MVA requirements have been applied.}
    \label{fig:PIDVetos_Ds2KKPi_D_Veto}   
\end{figure}
%%%%%%%%%%%%%%%%%%%%%%%%%%%%%%%%%%%%%%%%%%%%%%%%%%%%%%%%%%

%%%%%%%%%%%%%%%%%%%%%%%%%%%%%%%%%%%%%%%%%%%%%%%%%%%%%%%%%%
\begin{figure}[!h]
    \centering
    \begin{subfigure}[t]{0.4\textwidth}
        \includegraphics[width=1.0\textwidth]{figs/Selection/B2DsD0_Ds2KKPi_Lc_Veto_NoBDT.pdf}
        \caption{Normalisation without MVA cut}
    \end{subfigure}%
    \begin{subfigure}[t]{0.4\textwidth}
        \includegraphics[width=1.0\textwidth]{figs/Selection/B2DsPhi_Ds2KKPi_Lc_Veto_NoBDT.pdf}
        \caption{Signal without MVA cut}
    \end{subfigure}\\
    \begin{subfigure}[t]{0.4\textwidth}
        \includegraphics[width=1.0\textwidth]{figs/Selection/B2DsD0_Ds2KKPi_Lc_Veto_WithBDT.pdf}
        \caption{Normalisation with MVA cut}
    \end{subfigure}%
    \begin{subfigure}[t]{0.4\textwidth}
        \includegraphics[width=1.0\textwidth]{figs/Selection/B2DsPhi_Ds2KKPi_Lc_Veto_WithBDT.pdf}
        \caption{Signal with MVA cut}
    \end{subfigure}\\
    \caption{Invariant mass distributions of \decay{\Dsp}{\Kp\Km\pip} samples reconstructed as \decay{\Lc}{\Pp\Km\pip} for the signal and normalisation samples. The samples are shown with (red) and without (black) the veto described in Sec.~\ref{sec:pidvetos}. The distributions are shown before (top) and after (bottom) the MVA requirements have been applied.}
    \label{fig:PIDVetos_Ds2KKPi_Lc_Veto}   
\end{figure}
%%%%%%%%%%%%%%%%%%%%%%%%%%%%%%%%%%%%%%%%%%%%%%%%%%%%%%%%%%

\subsection{Invariant mass vetoes}
\label{sec:kinematicvetos}

Sharp peaking structures are observed in subsets of the final state particles. These are removed with simple invariant mass cuts to remove combinatorial or partially reconstructed backgrounds that result from these incorrectly reconstructed decays. 
For simplicity the final state particles for each mode are labelled with a number between 1--5 as described in Table~\ref{table:vetolabels}.

\begin{table*}[!ht]
\begin{center}
\begin{tabular}{ l c c c c c c }
\hline
Decay Mode & 1  & 2 & 3 & 4 & 5 \\
\hline
\decay{\Bp}{(\decay{\Dsp}{\Kp\Km\pip})\phiz}       & \Kp    & \Km    & \pip  & \Kp  & \Km \\
\decay{\Bp}{(\decay{\Dsp}{\pip\pim\pip})\phiz}     & \pip   & \pim   & \pip  & \Kp  & \Km \\
\decay{\Bp}{(\decay{\Dsp}{\Kp\pim\pip})\phiz}      & \Kp    & \pim   & \pip  & \Kp  & \Km \\
\hline
\decay{\Bp}{(\decay{\Dsp}{\Kp\Km\pip})\Kp\Km}      & \Kp    & \Km    & \pip  & \Kp  & \Km \\
\hline
\end{tabular}
\caption{Particle labels used when studying invariant mass vetoes for \decay{\Bp}{\Dsp\phiz} and \decay{\Bp}{\Dsp\Kp\Km} candidates.}
\label{table:vetolabels}
\end{center}
\end{table*}

All combinations of the final state particles that create a neutral or singly-charged candidate are investigated.
Significant structures are observed for all three \Dsp decay modes in some combination. 

The following vetos are applied to remove these incorrectly reconstructed decays.
\begin{itemize}
\item For the mode \decay{\Bp}{(\decay{\Dsp}{\Kp\Km\pip})\phiz}
\begin{itemize}
\item $|m(\text{1245})- m(\Bs)| > 50\mevcc$
\item $|m(\text{345})- m(\Dsp)| > 25\mevcc$ and $|m(\text{345})- m(\Dp)| > 25\mevcc$
\end{itemize}

\item For the mode \decay{\Bp}{(\decay{\Dsp}{\pip\pim\pip})\phiz}
\begin{itemize}
\item $|m(\text{145})- m(\Dsp)| > 25\mevcc$ and $|m(\text{145})- m(\Dp)| > 25\mevcc$
\item $|m(\text{245})- m(\Dsp)| > 25\mevcc$ and $|m(\text{245})- m(\Dp)| > 25\mevcc$
\item $|m(\text{345})- m(\Dsp)| > 25\mevcc$ and $|m(\text{345})- m(\Dp)| > 25\mevcc$
\end{itemize}
\item For the mode \decay{\Bp}{(\decay{\Dsp}{\Kp\pim\pip})\phiz}
\begin{itemize}
\item $|m(\text{245})- m(\Dsp)| > 25\mevcc$ and $|m(\text{245})- m(\Dp)| > 25\mevcc$
\item $|m(\text{345})- m(\Dsp)| > 25\mevcc$ and $|m(\text{345})- m(\Dp)| > 25\mevcc$
\end{itemize}
\end{itemize}

%%%%%%%%%%%%%%%%%%%%%%%%%%%%%%%%%%%%%%%%%%%%%%%%%%%%%%%%%%
\begin{figure}[!h]
   \centering
   \begin{subfigure}[t]{1.0\textwidth}
      \centering
      \begin{subfigure}[t]{0.32\textwidth}
         \includegraphics[width=1.0\textwidth]{figs/Selection/Veto_Comparison_B2DsPhi_Ds2KKPi_m1245.pdf}
      \end{subfigure}
      \begin{subfigure}[t]{0.32\textwidth}
         \includegraphics[width=1.0\textwidth]{figs/Selection/Veto_Comparison_B2DsPhi_Ds2KKPi_m345.pdf}
      \end{subfigure}
      \caption{\decay{\Bp}{(\decay{\Dsp}{\Kp\Km\pip})\phiz}}
   \end{subfigure}
   \begin{subfigure}[t]{1.0\textwidth}
      \centering
      \begin{subfigure}[t]{0.32\textwidth}
         \includegraphics[width=1.0\textwidth]{figs/Selection/Veto_Comparison_B2DsPhi_Ds2PiPiPi_m145.pdf}
      \end{subfigure}
      \begin{subfigure}[t]{0.32\textwidth}
         \includegraphics[width=1.0\textwidth]{figs/Selection/Veto_Comparison_B2DsPhi_Ds2PiPiPi_m245.pdf}
      \end{subfigure}
      \begin{subfigure}[t]{0.32\textwidth}
         \includegraphics[width=1.0\textwidth]{figs/Selection/Veto_Comparison_B2DsPhi_Ds2PiPiPi_m345.pdf}
      \end{subfigure}
      \caption{\decay{\Bp}{(\decay{\Dsp}{\pip\pim\pip})\phiz}}
   \end{subfigure}
   \begin{subfigure}[t]{1.0\textwidth}
      \centering
      \begin{subfigure}[t]{0.32\textwidth}
         \includegraphics[width=1.0\textwidth]{figs/Selection/Veto_Comparison_B2DsPhi_Ds2KPiPi_m245.pdf}
      \end{subfigure}
      \begin{subfigure}[t]{0.32\textwidth}
         \includegraphics[width=1.0\textwidth]{figs/Selection/Veto_Comparison_B2DsPhi_Ds2KPiPi_m345.pdf}
      \end{subfigure}
      \caption{\decay{\Bp}{(\decay{\Dsp}{\Kp\pim\pip})\phiz}}
   \end{subfigure}

   \caption{Invariant mass distributions for subsets of decay products in data (black) and simulation (red). The green region show the regions removed by the vetoes listed in Sec~\ref{sec:kinematicvetos}.}
   \label{fig:invariantmassvetoes}   
\end{figure}
%%%%%%%%%%%%%%%%%%%%%%%%%%%%%%%%%%%%%%%%%%%%%%%%%%%%%%%%%%


In the search for \decay{\Bp}{\Dsp\Kp\Km} decays the increased size of the $m(\Kp\Km)$ phase-space means more of the combinations of final state particles are susceptible to sharp peaking structure from incorrectly reconstructed backgrounds. Those spectra found to have significant peaking structures are additionally vetoed, as shown in Fig.~\ref{fig:invariantmassvetoes_DsKK}.

\begin{itemize}
\item Vetoes for the mode \decay{\Bp}{(\decay{\Dsp}{\Kp\Km\pip})\Kp\Km}:
\begin{itemize}
\item $|m(\text{1245})- m(\Bs)| > 50\mevcc$
\item $|m(\text{345})- m(\Dsp)| > 25\mevcc$ and $|m(\text{345})- m(\Dp)| > 25\mevcc$
\item $|m(\text{135})- m(\Dsp)| > 25\mevcc$
\item $|m(\text{234})- m(\Dsp)| > 25\mevcc$
\end{itemize}
\end{itemize}

%%%%%%%%%%%%%%%%%%%%%%%%%%%%%%%%%%%%%%%%%%%%%%%%%%%%%%%%%%
\begin{figure}[!h]
   \centering
   \begin{subfigure}[t]{1.0\textwidth}
      \centering
      \begin{subfigure}[t]{0.32\textwidth}
         \includegraphics[width=1.0\textwidth]{figs/Selection/Veto_Comparison_B2DsKK_Ds2KKPi_m1245.pdf}
      \end{subfigure}\\
      \begin{subfigure}[t]{0.32\textwidth}
         \includegraphics[width=1.0\textwidth]{figs/Selection/Veto_Comparison_B2DsKK_Ds2KKPi_m345.pdf}
      \end{subfigure}
      \begin{subfigure}[t]{0.32\textwidth}
         \includegraphics[width=1.0\textwidth]{figs/Selection/Veto_Comparison_B2DsKK_Ds2KKPi_m135.pdf}
      \end{subfigure}
      \begin{subfigure}[t]{0.32\textwidth}
         \includegraphics[width=1.0\textwidth]{figs/Selection/Veto_Comparison_B2DsKK_Ds2KKPi_m234.pdf}
      \end{subfigure}
   \end{subfigure}

   \caption{Invariant mass distributions for subsets of decay products for \decay{\Bp}{(\decay{\Dsp}{\Kp\Km\pip})\Kp\Km} decays in data (black) and simulation (red). The green region show the regions removed by the vetoes listed in Sec~\ref{sec:kinematicvetos}.}
   \label{fig:invariantmassvetoes_DsKK}   
\end{figure}
%%%%%%%%%%%%%%%%%%%%%%%%%%%%%%%%%%%%%%%%%%%%%%%%%%%%%%%%%%

{\color{Red}
\begin{itemize}
\item Include details of D* to D vetos
\end{itemize}
}

\subsection{Normalisation mode veto}
\label{sec:normvetos}

In the search for \decay{\Bp}{\Dsp\Kp\Km} decays, the entire $m(\Kp\Km)$ phasespace is used. This ranges from the \Kp\Km mass threshold at around $990\mevcc$ to the kinematic limit at $m(\Bp) - m(\Dsp) = 3300\mevcc$. This range is wide enough to include the mass of the \Dzb meson, $m(\Dz) = 1864 \mevcc$. Consequently, when inspecting the $m(\Kp\Km)$ spectrum for selected signal candidates there is an excess of events at the \Dz mass. It is necessary to remove these from the signal samples as, unsurprisingly, they result in a peak at the \Bp mass in the $m(\Dsp\Kp\Km)$ spectrum and lead to an incorrect signal yield. As described in Section~\ref{sec:selectionrequirements}, these removed events are used as the normalisation channel for the measurement of \decay{\Bp}{\Dsp\Kp\Km} decays. 
The region affected by the veto $|m(\Kp\Km) - m(\Dzb)| > 25 \mevcc$ is shown in Fig.~\ref{fig:normalisationveto_KK}.

%%%%%%%%%%%%%%%%%%%%%%%%%%%%%%%%%%%%%%%%%%%%%%%%%%%%%%%%%%
\begin{figure}[!h]
   \centering
   \begin{subfigure}[t]{0.49\textwidth}
      \centering
      \includegraphics[width=1.0\textwidth]{figs/Selection/D0Veto_Comparison_B2DsKK_Ds2KKPi_Phi_M.pdf}
      \caption{\decay{\Dzb}{\Kp\Km} }
      \label{fig:normalisationveto_KK}
   \end{subfigure}
   \begin{subfigure}[t]{0.49\textwidth}
      \centering
      \includegraphics[width=1.0\textwidth]{figs/Selection/D0Veto_Comparison_B2DsKK_Ds2KKPi_Phi_KPi_M.pdf}
      \caption{\decay{\Dzb}{\Kp\pim} }
      \label{fig:normalisationveto_KPi}
   \end{subfigure}
   \caption{Normalisation channel vetoes.}
   \label{fig:normalisationveto}   
\end{figure}
%%%%%%%%%%%%%%%%%%%%%%%%%%%%%%%%%%%%%%%%%%%%%%%%%%%%%%%%%%


In addition to the correctly reconstructed normalisation channel, the presence of the incorrectly reconstructed \decay{\Bp}{\Dsp(\decay{\Dzb}{\Kp\pim})} decay is observable in Fig.~\ref{fig:normalisationveto_KK}. This appears as smeared out peak to the right of the \Dzb peak. Although the probability of the \pim meson being misidentified as a \Km meson is low, the branching fraction for \decay{\Dzb}{\Kp\pim} is larger, leading to the observed excess. It is possible, and necessary, to remove this contribution. The \Kp\Km candidates are reconstructed again, swapping the mass hypothesis of the \Km to \pim. The distribution of these candidates in the vicinity of the \Dzb mass is shown in Fig.~\ref{fig:normalisationveto_KPi}. A distinct peak is observed at the \Dzb mass. This contribution is removed by the requirement $|m(\Kp\pim) - m(\Dzb)| > 25 \mevcc$. The effect of this requirement on the $m(\Kp\Km)$ spectrum is shown by the red line in Fig.~\ref{fig:normalisationveto_KK}, which represents the sample with this veto applied. This structure is effectively removed. 

Another structure appears to be present to the left of the \Dzb peak in Fig.~\ref{fig:normalisationveto_KK}. This is likely to be due to \decay{\Dzb}{\Kp\Km X} decays in which one or more particles have not been reconstructed. This partially reconstructed background would not peak at the \Bp mass in the $m(\Dsp\Kp\Km)$ spectrum due to the missing particles, therefore no attempt is made to remove this contribution.


\subsection{Multivariate analysis}
{\color{Red}Add citation and mention DD analysis}

Multivariate Analyses (MVAs) are used to help discriminate between genuine \Dsp and \phiz meson decays and combinations of unrelated tracks. 
These MVAs are trained using large samples of candidates from other \B mesons decays with similar topologies reconstructed in data. 
This data-driven approach can benefit from an expanded set of variables that are not perfectly represented in simulation, including track quality and particle identification information, in addition to the widely used kinematic and geometric properties.
The sample of \Dsp mesons is obtained from the relatively abundant \decay{\Bsb}{\Dsp\pim} decay. Similarly, the sample of \phiz mesons is obtained from \decay{\Bs}{\jpsi\phiz} decays. Large, high purity samples are reconstructed using similar requirements to those applied in the selection of signal \Dsp and \phiz mesons.
A sample is selected for each of the \Dsp and \phiz meson decays uses in this analysis, as listed in Table~\ref{tab:mva_modes}. 
The training of separate MVAs for the different \Dsp modes allows the use of particle identification variables that separate kaons and pions to be fully exploited.
The MVA for \decay{\phiz}{\Kp\Km} decays is used to select both \phiz and \Dzb mesons. For the normalisation mode this may be suboptimal, however it ensures the selection of the signal and normalisation channels are almost identical such that the systematic uncertainty in the ratio of efficiencies is minimised.
The samples of \decay{\Bsb}{\Dsp\pim} and \decay{\Bs}{\jpsi\phiz} decays are randomly split into two subsamples. The first is used to train the MVAs and the second is used to determine the efficiency of the selection. To prevent the difference between \phiz and \Dzb decays from affecting the ratio of efficiencies, the normalisation channel MVA efficiency is instead determined from a dedicated sample of \decay{\Dzb}{\Kp\Km} decays as detailed in Table.~\ref{tab:mva_modes}. The \decay{\phiz}{\Kp\Km} MVA is also used to select the \Kp\Km pair in \decay{\Bp}{\Dsp\Kp\Km} decays.
A total of eight MVAs are trained, one for each of the \Dsp and \phiz modes, in both Run I and Run II. Changes to the particle identification variables between the two running periods necessitates separate trainings. 


\begin{table}[h]
\begin{center}
\begin{tabular}{lll}
   \hline
   Sample                    & Mode                       & Use \\ 
   \hline
   \decay{\Bsb}{\Dsp\pim}    & \decay{\Dsp}{\Kp\Km\pip}   & Training, Efficiency \\
   \decay{\Bsb}{\Dsp\pim}    & \decay{\Dsp}{\Kp\pim\pip}  & Training, Efficiency \\
   \decay{\Bsb}{\Dsp\pim}    & \decay{\Dsp}{\pip\pim\pip} & Training, Efficiency \\
   \decay{\Bs}{\jpsi\phiz}   & \decay{\phiz}{\Kp\Km}      & Training, Efficiency \\
   \hline
   \decay{\Bp}{\Dzb\pip}     & \decay{\Dzb}{\Kp\Km}       & Efficiency          \\
   \hline
\end{tabular}
\end{center}
\caption{The decay modes used to train and determine the efficiency of the various MVAs used in this analysis.}
\label{tab:mva_modes}
\end{table}

\subsubsection{Preselection}

Before the samples of \Dsp and \phiz mesons are used to train the MVAs, precautions are taken to ensure the samples are representative of the \decay{\Bp}{\Dsp\phiz} and \decay{\Bp}{\Dsp\Kp\Km} signal decays. The \decay{\Bsb}{\Dsp\pim} and \decay{\Bs}{\jpsi\phiz} decays are selected using a similar procedure to the signal modes. Firstly, \emph{Stripping Lines} reconstruct the candidates. The \decay{\Bsb}{\Dsp\pim} decays are built using the same software module as the signal and normalisation channel. Therefore, the selection requirements for \Dsp candidates in \decay{\Bsb}{\Dsp\pim} decays are identical to those listed for the signal in Tables~\ref{tab:strippinglinecuts} and~\ref{tab:strippinglinecuts_norm}.
The \decay{\Bs}{\jpsi\phiz} decays are reconstructed using \decay{\jpsi}{\mup\mun} decays, therefore they are built using a different software module as the final state is not fully hadronic. As such there are some differences in the selection requirements for the \phiz mesons from the two sources. In general the selection requirements for the \decay{\Bs}{\jpsi\phiz} channel are looser as the presence of the muons allows an efficient triggering without the need to make tight selections on the \Kp\Km pair. A direct comparison of the relevant quantities are listed in Table~\ref{tab:strippingrequirments_phi}.

% Phi_P > 7000 && 
% Phi_FDCHI2_OWNPV >15 && 
% Phi_K0_P > 2000 && 
% Phi_K1_P > 2000 && 
% Phi_K0_IPCHI2_OWNPV > 4 && 
% Phi_K1_IPCHI2_OWNPV > 4 &&
% Phi_DIRA_OWNPV>0 && 
% Phi_ENDVERTEX_CHI2<16 && 
% Phi_K0_TRACK_GhostProb<0.4&& 
% Phi_K1_TRACK_GhostProb<0.4 && 
% (Phi_K0_PT+Phi_K1_PT)>1000)

\begin{table}[h]
\begin{center}
\begin{tabular}{ l l l l }
\hline
Particle       & Quantity                       &  Signal                              & Control                                 \\  
\hline
\phiz          & Mass minimum                   &  $870\mevcc$                         & $980\mevcc$ \\ 
               & Mass maximum                   &  $1170\mevcc$                        & $1050\mevcc$ \\ 
               & Transverse Momentum            &  -                                   & $\pt > 500 \mevc$                       \\  
               & Products \pt scalar sum        &  $\sum{|\pt|} > 1000 \mevc$          &  -                                      \\  
               & $\text{DOCA}(\Kp,\Km)$         &  $ < 0.5\mm$     &  -                                      \\  
               & Direction angle                &  $\cos{\theta}>0$                    &  -                                      \\  
               & Vertex quality                 &  $\chi^{2}/N_{\text{DOF}} < 16$      & $\chi^{2}/N_{\text{DOF}} < 25$          \\   
               & Flight distance significance   &  $\chi^{2}_{\text{FD} }  > 16$       &  -                                      \\ 
\hline
\Kpm           & Track quality                  &  $\chi^{2}/N_{\text{DOF}}<4.0$       &  $\chi^{2}/N_{\text{DOF}}<5.0$          \\  
               & Transverse momentum            &  $\pt > 100 \mevc$                   &  -                                      \\  
               & Momentum                       &  $\ptot > 1000 \mevc$                &  -                                      \\  
               & Impact parameter significance  &  $\chi^{2}_{\text{IP}} > 4$          &  -                                      \\  
               & Ghost track probability        &  $P_{\text{Ghost}} < 0.4$            &  -                                      \\
               & Particle identification        &  $\text{PIDK}>-10$                   & $\text{PIDK}>0$                         \\ 
\hline
\end{tabular}
\end{center}
\caption{The \emph{Stripping Line} requirements for \decay{\phiz}{\Kp\Km} candidates in the signal and MVA training (control) mode selection. All requirements are looser for the control channel with the exception of the particle identification requirements.}
\label{tab:strippingrequirments_phi}
\end{table}

The difference in the two selections could be potentially biasing when calculating the efficiency of the MVA, therefore the requirements are tightened on the half of the \decay{\Bs}{\jpsi\phiz} sample used to calculate the efficiency. The looser set of requirements are still used when training the MVA methods to maximise the sample sizes to help prevent overtraining. As detailed in Section~\ref{sec:mvamethod}, when training MVA the \phiz meson invariant mass sidebands in the \decay{\Bs}{\jpsi\phiz} sample are used as the background sample. It was found that using the tighter set of requirements led to a negligible amount of background candidates, resulting in overtraining. 


The second step of preselection aims to apply the same sequence of requirements to the modes used to train the MVAs as are applied to the signal and normalisation before the MVAs are applied. These are made up of the same trigger, background veto and PID requirements.
For \Dsp candidates from \decay{\Bsb}{\Dsp\pim} decays, the \Bsb meson is required to either be \texttt{TOS} with respect to the \texttt{L0Hadron} trigger or \texttt{TIS} with respect to \texttt{L0Global} as with the signal. The \Bsb candidates are then required to be \texttt{TOS} with respect to the same \hltone and \hlttwo triggers as used with the signal modes.

For the \decay{\Bs}{\jpsi\phiz} candidates, a large fraction are triggered due to the muons from the \jpsi decay. To ensure the sample is reflective of the signal decays the \Bs meson is required to be \texttt{TOS} with respect to the \texttt{L0Hadron} trigger or \texttt{TIS} with respect to \texttt{L0Global}. This effectively excludes events that have been selected solely due to the muons firing \texttt{L0Muon}. This consequently results in a large decrease in the available statistics.

As shown in Table~\ref{tab:strippingrequirments_phi}, the MVA training mode has tighter particle identification requirements than the signal mode. Particle identification plays a crucial role in the efficacy of the MVA methods, therefore the signal PID requirements for the \phiz meson decay products are tightened accordingly as discussed in Section~\ref{sec:pidrequirements}.

Finally, the same misidentified \D and \Lc hadron vetoes as detailed in Section~\ref{sec:pidvetos} are applied to the \decay{\Bsb}{\Dsp\pim} sample. These help to ensure the samples of \Dsp candidates are free from contamination.


\subsubsection{Background subtraction}
\label{sec:MVAbackgroundsubtraction}
%% Describe fits, sWeights and signal/background samples
In order to use the \decay{\Bs}{\jpsi\phiz} and \decay{\Bsb}{\Dsp\pim} samples to train MVAs, background subtracted distributions must be obtained for the variables used to discriminate the signal from backgrounds.  Firstly, unbinned extended maximum likelihood fits are performed to the \Dsp and \phiz invariant mass distribution in order to determine the yield of \Dsp and \phiz candidates respectively. 
Maximum likelihood fits determine the values of the signal and background yields for which the given data set was most likely. The likelihood, $\mathcal{L}$, is constructed from the probability density functions (PDFs) for the fit model, $F(m,\vec{p})$, for each entry $i$ in the data set. The PDF for each entry is evaluated at the corresponding value of the invariant mass, $m = m_{i}$, such that $\mathcal{L}$ is just a function of the PDF parameters, $\vec{p}$,
\begin{equation}
\mathcal{L}(\vec{p}) = \prod_{i}^{N} F(m=m_{i},\vec{p}).
\end{equation}

The maximum value of $\mathcal{L}(\vec{p})$ is achieved for the set of PDF parameters for which the data was most likely. 
It is computationally beneficial to instead compute the negative log-likelihood (NLL) rather than the likelihood directly, as addition is less intensive than multiplication. The NLL,
\begin{equation}
-\log\mathcal{L}(\vec{p}) = -\sum_{i}^{N} \log F(m=m_{i},\vec{p}),
\end{equation}
is then minimised with respect to the PDF parameters.

This likelihood can be \emph{extended} to allow the total yield attributed to the PDF to be determined as a parameter as well. This is nessesary to determine the yields of both signal and background contributions ($n_{\text{S}}$ and $n_{\text{B}}$).
The likelihood is multiplied by the Poisson probability density for measuring $n = n_{\text{S}} + n_{\text{B}}$ events, 
\begin{equation}
\mathcal{L}(n,\vec{p}) = \frac{n^{N}e^{-n}}{N!}\prod_{i}^{N} F(m=m_{i},\vec{p}) = \frac{e^{-n}}{N!}\prod_{i}^{N} n F(m=m_{i},\vec{p}).
\end{equation}

The extended NLL becomes 
\begin{equation}
-\log\mathcal{L}(n,\vec{p}) = -\sum_{i}^{N} \log n F(m=m_{i},\vec{p}) + n + \log N!.
\end{equation}
The $\log N!$ term can be ignored as it is a constant.
The fit model is constructed from signal and background contributions 
\begin{equation}
n F(m| n_{\text{S}},n_{\text{B}},\vec{p'},\vec{p''}) = n_{\text{S}} f(m|\vec{p'}) + n_{\text{B}} g(m|\vec{p''}),
\end{equation}
where $n_{\text{S}}$ and $n_{\text{B}}$ are the signal and background yields, $f$ and $g$ are the PDFs for the signal and background respectively, and $\vec{p'}$ and $\vec{p''}$ represent the parameters controlling the PDF shapes. The 

After the extended maximum likelihood fits have been performed and the yields established, the subtraction of candidates attributed to the background component is performed using the \sPlot technique~\cite{Pivk:2004ty}.
The \sPlot method assigns weights for each species in the fit model, namely signal and background. These are assigned to each event in the data set and are only dependent on the value of the fitted variable, in this case \phiz or \Dsp mass. When plotting other, uncorrelated variables and weighting each event accordingly, the contribution from each component in the fit can be extracted. 

%% Fit model
For both \decay{\Bs}{\jpsi\phiz} and \decay{\Bsb}{\Dsp\pim} decays, simple fit models are found to be sufficient when performing the \sPlot technique. For each mode the signal components are modelled with a single Gaussian probability density function
\begin{equation}
f(m|\mu,\sigma) = \frac{1}{\sqrt{2\pi\sigma^{2}}} \times e^{-\frac{(m-\mu)^{2}}{2\sigma^{2}}}, 
\end{equation}
where $\mu$ and $\sigma$ are the mean and width of the Gaussian distribution and $m$ is the observable invariant mass. The parameters $\mu$ and $\sigma$ are allowed to vary freely.
The background contributions to the \decay{\phiz}{\Kp\Km} decays are modelled using a Chebychev polynomial with two degrees of freedom
\begin{equation}
g(m|a,b) = a\times(2m^{2}-1) + b\times m + 1,
\end{equation}
where $a$ and $b$ are parameters that can vary freely and $m$ is the observable invariant mass. 
The \phiz meson mass is close to the threshold for \Kp\Km pair production so this parametrisation has sufficient freedom to successfully  model the increasing background shape.
The background contribution to all three \Dsp decays are modelled using exponetial functions with a single degree of freedom controlling the effective slope
\begin{equation}
g(m|c) = e^{-m\times c},
\end{equation}
where $c$ is a freely varying parameter and $m$ is the observable invariant mass.
The fits to the \phiz and \Dsp meson masses in the \decay{\Bs}{\jpsi\phiz} and \decay{\Bsb}{\Dsp\pim} samples are performed separately for each year of data taking and polarity of the \lhcb dipole magnet. These distributions are shown in Fig.~\ref{fig:mvatrainingsamples} for each of the \phiz and \Dsp decay modes.
The yields of candidates in each of the subsamples are tabulated in Table~\ref{table:mva_training_yields}, along with the totals for each mode. 


%%%%%%%%%%%%%%%%%%%%%%%%%%%%%%%%%%%%%%%%%%%%%%%%%%%%%%%%%%
\begin{figure}[!h]
   \centering
   \begin{subfigure}[t]{0.4\textwidth}
      \centering
      \includegraphics[width=1.0\textwidth]{figs/Selection/Fit_Data_Bs2JpsiPhi_Jpsi2MuMu_Phi2KK_2016_MagDown.pdf}
      \caption{\decay{\phiz}{\Kp\Km} MagDown}
   \end{subfigure}
   \begin{subfigure}[t]{0.4\textwidth}
      \centering
      \includegraphics[width=1.0\textwidth]{figs/Selection/Fit_Data_Bs2JpsiPhi_Jpsi2MuMu_Phi2KK_2016_MagUp.pdf}
      \caption{\decay{\phiz}{\Kp\Km} MagUp}
   \end{subfigure}\\
   \begin{subfigure}[t]{0.4\textwidth}
      \centering
      \includegraphics[width=1.0\textwidth]{figs/Selection/Fit_Data_Bs02DsPi_Ds2KKPi_2016_MagDown_PreSel.pdf}
      \caption{\decay{\Dsp}{\Kp\Km\pip} MagDown}
   \end{subfigure}
   \begin{subfigure}[t]{0.4\textwidth}
      \centering
      \includegraphics[width=1.0\textwidth]{figs/Selection/Fit_Data_Bs02DsPi_Ds2KKPi_2016_MagUp_PreSel.pdf}
      \caption{\decay{\Dsp}{\Kp\Km\pip} MagUp}
   \end{subfigure}
   \begin{subfigure}[t]{0.4\textwidth}
      \centering
      \includegraphics[width=1.0\textwidth]{figs/Selection/Fit_Data_Bs02DsPi_Ds2PiPiPi_2016_MagDown_PreSel.pdf}
      \caption{\decay{\Dsp}{\pip\pim\pip} MagDown}
   \end{subfigure}
   \begin{subfigure}[t]{0.4\textwidth}
      \centering
      \includegraphics[width=1.0\textwidth]{figs/Selection/Fit_Data_Bs02DsPi_Ds2PiPiPi_2016_MagUp_PreSel.pdf}
      \caption{\decay{\Dsp}{\pip\pim\pip} MagUp}
   \end{subfigure}
   \begin{subfigure}[t]{0.4\textwidth}
      \centering
      \includegraphics[width=1.0\textwidth]{figs/Selection/Fit_Data_Bs02DsPi_Ds2KPiPi_2016_MagDown_PreSel.pdf}
      \caption{\decay{\Dsp}{\Kp\pim\pip} MagDown}
   \end{subfigure}
   \begin{subfigure}[t]{0.4\textwidth}
      \centering
      \includegraphics[width=1.0\textwidth]{figs/Selection/Fit_Data_Bs02DsPi_Ds2KPiPi_2016_MagUp_PreSel.pdf}
      \caption{\decay{\Dsp}{\Kp\pim\pip} MagUp}
   \end{subfigure}
   \caption{Unbinned maximum likelihood fit to \decay{\Bs}{\jpsi\phiz} (top) and \decay{\Bsb}{\Dsp\pim} (bottom three rows) candidates. These distributions correspond to just the 2016 data samples.}
   \label{fig:mvatrainingsamples}   
\end{figure}


%%%%%%%%%%%%%%%%%%%%%%%%%%%%%%%%%%%%%%%%%%%%%%%%%%%%%%%%%%
\begin{table}[!h]
   \begin{center}
      \begin{tabular}{llcc}
         \hline
         Mode                       & Year   & MagDown            & MagUp           \\ 
         \hline                                                
         \decay{\phiz}{\Kp\Km}      & 2011   & 3190 $\pm$ 170     &   2120 $\pm$ 140 \\
                                    & 2012   & 5800 $\pm$ 200     &   5540 $\pm$ 200 \\
                                    & 2015   & 2240 $\pm$ 110     &   1640 $\pm$ 100 \\
                                    & 2016   & 11200 $\pm$ 300    &   11100 $\pm$ 300 \\
                                    & Total  & \multicolumn{2}{c}{42830 $\pm$ 600}\\
         \hline                                                
         \decay{\Dsp}{\Kp\Km\pip}   & 2011   & 97100  $\pm$ 500   & 68600  $\pm$ 400 \\
                                    & 2012   & 216600 $\pm$ 800   & 212800 $\pm$ 800 \\
                                    & 2015   & 68600  $\pm$ 400   & 43400  $\pm$ 300 \\
                                    & 2016   & 321400 $\pm$ 900   & 310800 $\pm$ 900 \\
                                    & Total  & \multicolumn{2}{c}{1339300 $\pm$ 1900}\\
         \hline                                                
         \decay{\Dsp}{\pip\pim\pip} & 2011   & 24300 $\pm$ 500    &   17100 $\pm$ 400 \\
                                    & 2012   & 57700 $\pm$ 800    &   56500 $\pm$ 900 \\
                                    & 2015   & 20400 $\pm$ 500    &   13100 $\pm$ 400 \\
                                    & 2016   & 99200 $\pm$ 1200   &  92100 $\pm$ 1200 \\
                                    & Total  & \multicolumn{2}{c}{380400 $\pm$ 2300}\\
         \hline                                                
         \decay{\Dsp}{\Kp\pim\pip}  & 2011   & 13200 $\pm$ 600    & 9500  $\pm$ 400 \\
                                    & 2012   & 29300 $\pm$ 900    & 29300 $\pm$ 900 \\
                                    & 2015   & 10000 $\pm$ 500    & 6200  $\pm$ 400 \\
                                    & 2016   & 47800 $\pm$ 1200   & 44200 $\pm$ 1100 \\
                                    & Total  & \multicolumn{2}{c}{189500 $\pm$ 2300}\\
         \hline
      \end{tabular}
      \caption{Yields of the samples used to train data-driven MVAs. The total yields are summed over all years and both magnet polarities.}
      \label{table:mva_training_yields}
   \end{center}
\end{table}
%%%%%%%%%%%%%%%%%%%%%%%%%%%%%%%%%%%%%%%%%%%%%%%%%%%%%%%%%%

\subsubsection{Input variables}


%% Discussion of input variables
The MVA method is trained using a large set of variables chosen to help discriminate between the signals of interest and combinations of unrelated tracks. These variables include the properties of the \Kpm or \pipm decay products as well as those of the \phiz or \Dsp candidate itself. Many of the quantities are the same as those previously defined in Section.~\ref{sec:selectionrequirements}. The additional parameters relate to the track quality and particle identification information and are defined as follows:

\begin{description}
\item \textbf{Track matching quality, $\chi^{2}_{\text{TRKMATCH}}$}: {\color{Red} chi2 from matching velo stubs to trackers stubs?}
\item \textbf{Velo track quality, $\chi^{2}_{\text{TRKVELO}}$}: {\color{Red} chi2 of fit to just hits in the velo?}
\item \textbf{Kaon, proton, pion and ghost probabilities, ProbNNx}: these variables are an alternative form of \emph{Particle Identification} variables in addition to those already discussed in Section~\ref{sec:pidrequirements}. Each ProbNNx variable acts as the probability for the reconstructed track to be of the species x. These variables are the responses of Artificial Neural Networks trained to differentiate between the various particle types using a large number of inputs from different subdetectors. The trainings use simulated samples of signal and background decays and have been retuned between Run~I and Run~II to reflect the difference in the detector configuration. Four of the available ProbNNx tuning are used in this analysis. Three of these identify kaons, protons and pions. The last aims to identify ghost tracks, similar to the $P_{\text{Ghost}}$ variable already discussed. 
\end{description}

The variables used in the \phiz meson MVA are listed in Table~\ref{tab:mvavars_phi}. Some of the variables are input into the MVA training algorithm as their logarithm, rather than the variable directly. These variables tend to be rapidly increasing in at certain values, so using the logarithm aids visualisation. A total of 24 variables are used in this MVA.

{\color{red}
\begin{itemize}
\item FDchi2 not included so it works on \Dzb and \phiz mesons.
\end{itemize}
} 

\begin{table}[h]
   \begin{center}
      \begin{tabular}{ l l l}

         \hline
         Particle       & Description                    & Quantity                          \\    
         \hline
         \phiz          & Momentum                       &  \ptot                            \\  
                        & Transverse Momentum            &  \pt                              \\  
                        & Vertex quality                 &  $\log_{10}(\chi^{2}_{\text{VXT}})$    \\  
                        & Impact parameter significance  &  $\log_{10}(\chi^{2}_{\text{IP}})$     \\    
         \hline
         \Kpm           & Momentum                       &  \ptot                            \\  
                        & Transverse momentum            &  \pt                              \\ 
                        & Longitudinal momentum          &  \pz                              \\
                        & Impact parameter significance  &  $\log_{10}(\chi^{2}_{\text{IP}})$     \\    
                        & Track quality                  &  $\chi^{2}_{\text{TRK}}$          \\    
                        & Track matching quality         &  $\chi^{2}_{\text{TRKMATCH}}$     \\    
                        & Kaon probability               &  ProbNNk                          \\    
                        & Proton probability             &  ProbNNp                          \\    
                        & Pion probability               &  ProbNNpi                         \\    
                        & Ghost probability              &  ProbNNghost                      \\    
         \hline
      \end{tabular}
   \end{center}
   \caption{Discriminating variables used to train the \phiz MVA.}
   \label{tab:mvavars_phi}
\end{table}
 
The variables used in the \Dsp MVA are listed in Table~\ref{tab:mvavars_ds}. These are largely very similar to those used for the \phiz meson MVA, however it also includes the \Dsp meson's flight distance significance as, unlike the \phiz meson, the \Dsp meson decays displaced from its production vertex. A total of 35 variables are used in these MVAs.



\begin{table}[h]
   \begin{center}
      \begin{tabular}{ l l l}

         \hline
         Particle       & Description                    & Quantity                          \\    
         \hline
         \Dsp           & Momentum                       &  \ptot                            \\  
                        & Transverse Momentum            &  \pt                              \\  
                        & Vertex quality                 &  $\log_{10}(\chi^{2}_{\text{VXT}})$    \\  
                        & Impact parameter significance  &  $\log_{10}(\chi^{2}_{\text{IP}})$     \\    
                        & Flight distance significance   &  $\log_{10}(\chi^{2}_{\text{FD}})$     \\    
         \hline
         \Kpm, \pipm    & Momentum                       &  \ptot                            \\  
                        & Transverse momentum            &  \pt                              \\
                        & Impact parameter significance  &  $\log_{10}(\chi^{2}_{\text{IP}})$     \\    
                        & Track quality                  &  $\chi^{2}_{\text{TRK}}$          \\    
                        & Velo track quality             &  $\chi^{2}_{\text{TRKVELO}}$      \\    
                        & Track matching quality         &  $\chi^{2}_{\text{TRKMATCH}}$     \\    
                        & Kaon probability               &  ProbNNk                          \\    
                        & Proton probability             &  ProbNNp                          \\    
                        & Pion probability               &  ProbNNpi                         \\    
                        & Ghost probability              &  ProbNNghost                      \\    
         \hline
      \end{tabular}
   \end{center}
   \caption{Discriminating variables used to train the \Dsp MVAs.}
   \label{tab:mvavars_ds}
\end{table}


When training the MVA methods, the variables are ranked according to their discrimination power. Examples of these ranking are shown for the Run~II \decay{\phiz}{\Kp\Km} MVA and Run~II \decay{\Dsp}{\Kp\Km\pip} MVA in Table~\ref{tab:mvarank_dsandphi}. It can be seen that for both the \Dsp and \phiz meson MVAs that the kaon particle identification variable and various $\log_{10}(\chi^{2}_{\text{IP}})$ variables are the mode discriminating.


% \begin{table}[h]
%    \begin{center}
%       \begin{tabular}{ c l c}

%          \hline 
%          Rank & Variable                           & Importance   \\
%          \hline
%           1 & \Kp ProbNNk                          & $6.001\times 10^{-2}$ \\
%           2 & \phiz $\log_{10}(\chi^{2}_{\text{IP}})$   & $6.001\times 10^{-2}$ \\
%           3 & \Km $\log_{10}(\chi^{2}_{\text{IP}})$     & $5.997\times 10^{-2}$ \\
%           4 & \phiz $\log_{10}(\chi^{2}_{\text{VTX}})$  & $5.827\times 10^{-2}$ \\
%           5 & \Km ProbNNpi                         & $5.826\times 10^{-2}$ \\
%           6 & \Kp $\log_{10}(\chi^{2}_{\text{IP}})$     & $5.761\times 10^{-2}$ \\
%           7 & \Kp ProbNNghost                      & $5.427\times 10^{-2}$ \\
%           8 & \Km ProbNNk                          & $5.404\times 10^{-2}$ \\
%           9 & \Kp $\chi^{2}_{\text{TRK}}$          & $5.359\times 10^{-2}$ \\
%          10 & \Km $\chi^{2}_{\text{TRK}}$          & $5.201\times 10^{-2}$ \\
%          11 & \Kp ProbNNpi                         & $5.162\times 10^{-2}$ \\
%          12 & \Km ProbNNghost                      & $4.985\times 10^{-2}$ \\
%          13 & \Kp ProbNNp                          & $4.486\times 10^{-2}$ \\
%          14 & \Km ProbNNp                          & $4.410\times 10^{-2}$ \\
%          15 & \Kp $\chi^{2}_{\text{TRKMATCH}}$     & $4.291\times 10^{-2}$ \\
%          16 & \Km $\chi^{2}_{\text{TRKMATCH}}$     & $3.977\times 10^{-2}$ \\
%          17 & \Km \pt                              & $3.437\times 10^{-2}$ \\
%          18 & \Kp \pt                              & $3.012\times 10^{-2}$ \\
%          19 & \phiz PT                             & $2.400\times 10^{-2}$ \\
%          20 & \phiz P                              & $1.792\times 10^{-2}$ \\
%          21 & \Km \pz                              & $1.397\times 10^{-2}$ \\
%          22 & \Km \ptot                            & $1.387\times 10^{-2}$ \\
%          23 & \Kp \pz                              & $1.245\times 10^{-2}$ \\
%          24 & \Kp \ptot                            & $1.218\times 10^{-2}$ \\
%          \hline
%       \end{tabular}
%    \end{center}
%    \caption{Ranking of variables used to train the \decay{\phiz}{\Kp\Km} Run~II MVA.}
%    \label{tab:mvarank_ds}
% \end{table}

% \begin{table}[h]
%    \begin{center}
%       \begin{tabular}{ c l c}

%          \hline 
%          Rank & Variable                         & Importance           \\
%          \hline
%           1 & \Kp ProbNNk                        & $7.639\times10^{-2}$ \\
%           2 & \Km ProbNNk                        & $6.043\times10^{-2}$ \\
%           3 & \Kp $\log_{10}(\chi^{2}_{\text{IP}})$   & $5.469\times10^{-2}$ \\
%           4 & \pip $\log_{10}(\chi^{2}_{\text{IP}})$  & $4.984\times10^{-2}$ \\
%           5 & \Kp \ptot                          & $4.348\times10^{-2}$ \\
%           6 & \Dsp $\log_{10}(\chi^{2}_{\text{IP}})$  & $4.100\times10^{-2}$ \\
%           7 & \pip \pt                           & $3.953\times10^{-2}$ \\
%           8 & \Kp \pt                            & $3.937\times10^{-2}$ \\
%           9 & \pip ProbNNpi                      & $3.881\times10^{-2}$ \\
%          10 & \Dsp $\log_{10}(\chi^{2}_{\text{FD}})$  & $3.788\times10^{-2}$ \\
%          11 & \Km $\log_{10}(\chi^{2}_{\text{IP}})$   & $3.735\times10^{-2}$ \\
%          12 & \Dsp $\log_{10}(\chi^{2}_{\text{VTX}})$ & $3.016\times10^{-2}$ \\
%          13 & \Km \ptot                          & $2.770\times10^{-2}$ \\
%          14 & \Kp ProbNNp                        & $2.758\times10^{-2}$ \\
%          15 & \Km ProbNNp                        & $2.692\times10^{-2}$ \\
%          16 & \pip ProbNNghost                   & $2.623\times10^{-2}$ \\
%          17 & \Kp ProbNNghost                    & $2.593\times10^{-2}$ \\
%          18 & \Dsp \pt                           & $2.543\times10^{-2}$ \\
%          19 & \Kp ProbNNpi                       & $2.525\times10^{-2}$ \\
%          20 & \Km \pt                            & $2.355\times10^{-2}$ \\
%          21 & \Dsp \ptot                         & $2.327\times10^{-2}$ \\
%          22 & \Km ProbNNpi                       & $2.226\times10^{-2}$ \\
%          23 & \pip \ptot                         & $2.083\times10^{-2}$ \\
%          24 & \pip ProbNNp                       & $1.849\times10^{-2}$ \\
%          25 & \pip ProbNNk                       & $1.793\times10^{-2}$ \\
%          26 & \Km ProbNNghost                    & $1.607\times10^{-2}$ \\
%          27 & \pip $\chi^{2}_{\text{TRKVELO}}$   & $1.589\times10^{-2}$ \\
%          28 & \Km $\chi^{2}_{\text{TRK}}$        & $1.588\times10^{-2}$ \\
%          29 & \pip $\chi^{2}_{\text{TRKMATCH}}$  & $1.548\times10^{-2}$ \\
%          30 & \Kp $\chi^{2}_{\text{TRK}}$        & $1.522\times10^{-2}$ \\
%          31 & \Km $\chi^{2}_{\text{TRKMATCH}}$   & $1.442\times10^{-2}$ \\
%          32 & \pip $\chi^{2}_{\text{TRK}}$       & $1.433\times10^{-2}$ \\
%          33 & \Kp $\chi^{2}_{\text{TRKVELO}}$    & $1.184\times10^{-2}$ \\
%          34 & \Kp $\chi^{2}_{\text{TRKMATCH}}$   & $1.176\times10^{-2}$ \\
%          35 & \Km $\chi^{2}_{\text{TRKVELO}}$    & $8.818\times10^{-3}$ \\
%          \hline
%       \end{tabular}
%    \end{center}
%    \caption{Ranking of variables used to train the \decay{\Dsp}{\Kp\Km\pip} Run~II MVA.}
%    \label{tab:mvarank_ds}
% \end{table}

\begin{table}[h]
\centering
\scalebox{1.0}{
\begin{tabular}{ c l c | c l c}

\hline 

%\multicolumn{3}{c|}{\decay{\Dsp}{\Kp\Km\pip}}                   & \multicolumn{3}{c}{\decay{\phiz}{\Kp\Km}}                        \\
Rank & Variable                         & Importance           & Rank & Variable                           & Importance           \\
\hline
 1 & \Kp ProbNNk                        & $7.639\times10^{-2}$ &  1 & \Kp ProbNNk                          & $6.001\times 10^{-2}$\\
 2 & \Km ProbNNk                        & $6.043\times10^{-2}$ &  2 & \phiz $\log_{10}(\chi^{2}_{\text{IP}})$   & $6.001\times 10^{-2}$\\
 3 & \Kp $\log_{10}(\chi^{2}_{\text{IP}})$   & $5.469\times10^{-2}$ &  3 & \Km $\log_{10}(\chi^{2}_{\text{IP}})$     & $5.997\times 10^{-2}$\\
 4 & \pip $\log_{10}(\chi^{2}_{\text{IP}})$  & $4.984\times10^{-2}$ &  4 & \phiz $\log_{10}(\chi^{2}_{\text{VTX}})$  & $5.827\times 10^{-2}$\\
 5 & \Kp \ptot                          & $4.348\times10^{-2}$ &  5 & \Km ProbNNpi                         & $5.826\times 10^{-2}$\\
 6 & \Dsp $\log_{10}(\chi^{2}_{\text{IP}})$  & $4.100\times10^{-2}$ &  6 & \Kp $\log_{10}(\chi^{2}_{\text{IP}})$     & $5.761\times 10^{-2}$\\
 7 & \pip \pt                           & $3.953\times10^{-2}$ &  7 & \Kp ProbNNghost                      & $5.427\times 10^{-2}$\\
 8 & \Kp \pt                            & $3.937\times10^{-2}$ &  8 & \Km ProbNNk                          & $5.404\times 10^{-2}$\\
 9 & \pip ProbNNpi                      & $3.881\times10^{-2}$ &  9 & \Kp $\chi^{2}_{\text{TRK}}$          & $5.359\times 10^{-2}$\\
10 & \Dsp $\log_{10}(\chi^{2}_{\text{FD}})$  & $3.788\times10^{-2}$ & 10 & \Km $\chi^{2}_{\text{TRK}}$          & $5.201\times 10^{-2}$\\
11 & \Km $\log_{10}(\chi^{2}_{\text{IP}})$   & $3.735\times10^{-2}$ & 11 & \Kp ProbNNpi                         & $5.162\times 10^{-2}$\\
12 & \Dsp $\log_{10}(\chi^{2}_{\text{VTX}})$ & $3.016\times10^{-2}$ & 12 & \Km ProbNNghost                      & $4.985\times 10^{-2}$\\
13 & \Km \ptot                          & $2.770\times10^{-2}$ & 13 & \Kp ProbNNp                          & $4.486\times 10^{-2}$\\
14 & \Kp ProbNNp                        & $2.758\times10^{-2}$ & 14 & \Km ProbNNp                          & $4.410\times 10^{-2}$\\
15 & \Km ProbNNp                        & $2.692\times10^{-2}$ & 15 & \Kp $\chi^{2}_{\text{TRKMATCH}}$     & $4.291\times 10^{-2}$\\
% 16 & \pip ProbNNghost                   & $2.623\times10^{-2}$ & 16 & \Km $\chi^{2}_{\text{TRKMATCH}}$     & $3.977\times 10^{-2}$\\
% 17 & \Kp ProbNNghost                    & $2.593\times10^{-2}$ & 17 & \Km \pt                              & $3.437\times 10^{-2}$\\
% 18 & \Dsp \pt                           & $2.543\times10^{-2}$ & 18 & \Kp \pt                              & $3.012\times 10^{-2}$\\
% 19 & \Kp ProbNNpi                       & $2.525\times10^{-2}$ & 19 & \phiz PT                             & $2.400\times 10^{-2}$\\
% 20 & \Km \pt                            & $2.355\times10^{-2}$ & 20 & \phiz P                              & $1.792\times 10^{-2}$\\
% 21 & \Dsp \ptot                         & $2.327\times10^{-2}$ & 21 & \Km \pz                              & $1.397\times 10^{-2}$\\
% 22 & \Km ProbNNpi                       & $2.226\times10^{-2}$ & 22 & \Km \ptot                            & $1.387\times 10^{-2}$\\
% 23 & \pip \ptot                         & $2.083\times10^{-2}$ & 23 & \Kp \pz                              & $1.245\times 10^{-2}$\\
% 24 & \pip ProbNNp                       & $1.849\times10^{-2}$ & 24 & \Kp \ptot                            & $1.218\times 10^{-2}$\\
% 25 & \pip ProbNNk                       & $1.793\times10^{-2}$ \\
% 26 & \Km ProbNNghost                    & $1.607\times10^{-2}$ \\
% 27 & \pip $\chi^{2}_{\text{TRKVELO}}$   & $1.589\times10^{-2}$ \\
% 28 & \Km $\chi^{2}_{\text{TRK}}$        & $1.588\times10^{-2}$ \\
% 29 & \pip $\chi^{2}_{\text{TRKMATCH}}$  & $1.548\times10^{-2}$ \\
% 30 & \Kp $\chi^{2}_{\text{TRK}}$        & $1.522\times10^{-2}$ \\
% 31 & \Km $\chi^{2}_{\text{TRKMATCH}}$   & $1.442\times10^{-2}$ \\
% 32 & \pip $\chi^{2}_{\text{TRK}}$       & $1.433\times10^{-2}$ \\
% 33 & \Kp $\chi^{2}_{\text{TRKVELO}}$    & $1.184\times10^{-2}$ \\
% 34 & \Kp $\chi^{2}_{\text{TRKMATCH}}$   & $1.176\times10^{-2}$ \\
% 35 & \Km $\chi^{2}_{\text{TRKVELO}}$    & $8.818\times10^{-3}$ \\
\hline
\end{tabular}
}
\caption{Ranking of variables for the 15 highest ranked variables used to train the \decay{\Dsp}{\Kp\Km\pip} (left) and \decay{\phiz}{\Kp\Km} (right) Run~II MVAs.}
\label{tab:mvarank_dsandphi}
\end{table}


%%%%%%%%%%%%%%%%%%%%%%%%%%%%%%%%%%%%%%%%%%%%%%%%%%%%%%%%%%
\begin{figure}[!h]
   \centering
   \begin{subfigure}[t]{0.22\textwidth}
      \centering
      \includegraphics[width=1.0\textwidth]{figs/Selection/Phi_BDT_Var_Ds2KKPi_Phi_P.pdf}
   \end{subfigure}
   \begin{subfigure}[t]{0.22\textwidth}
      \centering
      \includegraphics[width=1.0\textwidth]{figs/Selection/Phi_BDT_Var_Ds2KKPi_Phi_PT.pdf}
   \end{subfigure}
   \begin{subfigure}[t]{0.22\textwidth}
      \centering
      \includegraphics[width=1.0\textwidth]{figs/Selection/Phi_BDT_Var_Ds2KKPi_log10_Phi_ENDVERTEX_CHI2.pdf}
   \end{subfigure}
   \begin{subfigure}[t]{0.22\textwidth}
      \centering
      \includegraphics[width=1.0\textwidth]{figs/Selection/Phi_BDT_Var_Ds2KKPi_log10_Phi_IPCHI2_OWNPV.pdf}
   \end{subfigure}
   \begin{subfigure}[t]{0.22\textwidth}
      \centering
      \includegraphics[width=1.0\textwidth]{figs/Selection/Phi_BDT_Var_Ds2KKPi_Phi_K0_P.pdf}
   \end{subfigure}
   \begin{subfigure}[t]{0.22\textwidth}
      \centering
      \includegraphics[width=1.0\textwidth]{figs/Selection/Phi_BDT_Var_Ds2KKPi_Phi_K1_P.pdf}
   \end{subfigure}
   \begin{subfigure}[t]{0.22\textwidth}
      \centering
      \includegraphics[width=1.0\textwidth]{figs/Selection/Phi_BDT_Var_Ds2KKPi_Phi_K0_PT.pdf}
   \end{subfigure}
   \begin{subfigure}[t]{0.22\textwidth}
      \centering
      \includegraphics[width=1.0\textwidth]{figs/Selection/Phi_BDT_Var_Ds2KKPi_Phi_K1_PT.pdf}
   \end{subfigure}
   \begin{subfigure}[t]{0.22\textwidth}
      \centering
      \includegraphics[width=1.0\textwidth]{figs/Selection/Phi_BDT_Var_Ds2KKPi_Phi_K0_PZ.pdf}
   \end{subfigure}
   \begin{subfigure}[t]{0.22\textwidth}
      \centering
      \includegraphics[width=1.0\textwidth]{figs/Selection/Phi_BDT_Var_Ds2KKPi_Phi_K1_PZ.pdf}
   \end{subfigure}
   \begin{subfigure}[t]{0.22\textwidth}
      \centering
      \includegraphics[width=1.0\textwidth]{figs/Selection/Phi_BDT_Var_Ds2KKPi_log10_Phi_K0_IPCHI2_OWNPV.pdf}
   \end{subfigure}
   \begin{subfigure}[t]{0.22\textwidth}
      \centering
      \includegraphics[width=1.0\textwidth]{figs/Selection/Phi_BDT_Var_Ds2KKPi_log10_Phi_K1_IPCHI2_OWNPV.pdf}
   \end{subfigure}
   \begin{subfigure}[t]{0.22\textwidth}
      \centering
      \includegraphics[width=1.0\textwidth]{figs/Selection/Phi_BDT_Var_Ds2KKPi_Phi_K0_MC15TuneV1_ProbNNk.pdf}
   \end{subfigure}
   \begin{subfigure}[t]{0.22\textwidth}
      \centering
      \includegraphics[width=1.0\textwidth]{figs/Selection/Phi_BDT_Var_Ds2KKPi_Phi_K1_MC15TuneV1_ProbNNk.pdf}
   \end{subfigure}
   \begin{subfigure}[t]{0.22\textwidth}
      \centering
      \includegraphics[width=1.0\textwidth]{figs/Selection/Phi_BDT_Var_Ds2KKPi_Phi_K0_MC15TuneV1_ProbNNp.pdf}
   \end{subfigure}
   \begin{subfigure}[t]{0.22\textwidth}
      \centering
      \includegraphics[width=1.0\textwidth]{figs/Selection/Phi_BDT_Var_Ds2KKPi_Phi_K1_MC15TuneV1_ProbNNp.pdf}
   \end{subfigure}
   \begin{subfigure}[t]{0.22\textwidth}
      \centering
      \includegraphics[width=1.0\textwidth]{figs/Selection/Phi_BDT_Var_Ds2KKPi_Phi_K0_MC15TuneV1_ProbNNpi.pdf}
   \end{subfigure}
   \begin{subfigure}[t]{0.22\textwidth}
      \centering
      \includegraphics[width=1.0\textwidth]{figs/Selection/Phi_BDT_Var_Ds2KKPi_Phi_K1_MC15TuneV1_ProbNNpi.pdf}
   \end{subfigure}
   \begin{subfigure}[t]{0.22\textwidth}
      \centering
      \includegraphics[width=1.0\textwidth]{figs/Selection/Phi_BDT_Var_Ds2KKPi_Phi_K0_MC15TuneV1_ProbNNghost.pdf}
   \end{subfigure}
   \begin{subfigure}[t]{0.22\textwidth}
      \centering
      \includegraphics[width=1.0\textwidth]{figs/Selection/Phi_BDT_Var_Ds2KKPi_Phi_K1_MC15TuneV1_ProbNNghost.pdf}
   \end{subfigure}
   \begin{subfigure}[t]{0.22\textwidth}
      \centering
      \includegraphics[width=1.0\textwidth]{figs/Selection/Phi_BDT_Var_Ds2KKPi_Phi_K0_TRACK_CHI2NDOF.pdf}
   \end{subfigure}
   \begin{subfigure}[t]{0.22\textwidth}
      \centering
      \includegraphics[width=1.0\textwidth]{figs/Selection/Phi_BDT_Var_Ds2KKPi_Phi_K1_TRACK_CHI2NDOF.pdf}
   \end{subfigure}
   \begin{subfigure}[t]{0.22\textwidth}
      \centering
      \includegraphics[width=1.0\textwidth]{figs/Selection/Phi_BDT_Var_Ds2KKPi_Phi_K0_TRACK_MatchCHI2.pdf}
   \end{subfigure}
   \begin{subfigure}[t]{0.22\textwidth}
      \centering
      \includegraphics[width=1.0\textwidth]{figs/Selection/Phi_BDT_Var_Ds2KKPi_Phi_K1_TRACK_MatchCHI2.pdf}
   \end{subfigure}
   \caption{Distributions of MVA training variables in the signal (blue) and background (red) samples for the Run~II \decay{\phiz}{\Kp\Km} MVA.}
   \label{fig:mvatrainingvariables_phi}   
\end{figure}
%%%%%%%%%%%%%%%%%%%%%%%%%%%%%%%%%%%%%%%%%%%%%%%%%%%%%%%%%%



%%%%%%%%%%%%%%%%%%%%%%%%%%%%%%%%%%%%%%%%%%%%%%%%%%%%%%%%%%
\begin{figure}[!h]
   \centering
   \begin{subfigure}[t]{0.22\textwidth}
      \centering
      \includegraphics[width=1.0\textwidth]{figs/Selection/Ds_BDT_Var_Ds2KKPi_D_P.pdf}
   \end{subfigure}
   \begin{subfigure}[t]{0.22\textwidth}
      \centering
      \includegraphics[width=1.0\textwidth]{figs/Selection/Ds_BDT_Var_Ds2KKPi_D_PT.pdf}
   \end{subfigure}
   \begin{subfigure}[t]{0.22\textwidth}
      \centering
      \includegraphics[width=1.0\textwidth]{figs/Selection/Ds_BDT_Var_Ds2KKPi_log10_D_ENDVERTEX_CHI2.pdf}
   \end{subfigure}
   \begin{subfigure}[t]{0.22\textwidth}
      \centering
      \includegraphics[width=1.0\textwidth]{figs/Selection/Ds_BDT_Var_Ds2KKPi_log10_D_IPCHI2_OWNPV.pdf}
   \end{subfigure}
   \begin{subfigure}[t]{0.22\textwidth}
      \centering
      \includegraphics[width=1.0\textwidth]{figs/Selection/Ds_BDT_Var_Ds2KKPi_log10_D_FDCHI2_OWNPV.pdf}
   \end{subfigure}
   \begin{subfigure}[t]{0.22\textwidth}
      \centering
      \includegraphics[width=1.0\textwidth]{figs/Selection/Ds_BDT_Var_Ds2KKPi_D_K0_P.pdf}
   \end{subfigure}
   \begin{subfigure}[t]{0.22\textwidth}
      \centering
      \includegraphics[width=1.0\textwidth]{figs/Selection/Ds_BDT_Var_Ds2KKPi_D_K1_P.pdf}
   \end{subfigure}
   \begin{subfigure}[t]{0.22\textwidth}
      \centering
      \includegraphics[width=1.0\textwidth]{figs/Selection/Ds_BDT_Var_Ds2KKPi_D_P_P.pdf}
   \end{subfigure}
   \begin{subfigure}[t]{0.22\textwidth}
      \centering
      \includegraphics[width=1.0\textwidth]{figs/Selection/Ds_BDT_Var_Ds2KKPi_D_K0_PT.pdf}
   \end{subfigure}
   \begin{subfigure}[t]{0.22\textwidth}
      \centering
      \includegraphics[width=1.0\textwidth]{figs/Selection/Ds_BDT_Var_Ds2KKPi_D_K1_PT.pdf}
   \end{subfigure}
   \begin{subfigure}[t]{0.22\textwidth}
      \centering
      \includegraphics[width=1.0\textwidth]{figs/Selection/Ds_BDT_Var_Ds2KKPi_D_P_PT.pdf}
   \end{subfigure}
   \begin{subfigure}[t]{0.22\textwidth}
      \centering
      \includegraphics[width=1.0\textwidth]{figs/Selection/Ds_BDT_Var_Ds2KKPi_log10_D_K0_IPCHI2_OWNPV.pdf}
   \end{subfigure}
   \begin{subfigure}[t]{0.22\textwidth}
      \centering
      \includegraphics[width=1.0\textwidth]{figs/Selection/Ds_BDT_Var_Ds2KKPi_log10_D_K1_IPCHI2_OWNPV.pdf}
   \end{subfigure}
   \begin{subfigure}[t]{0.22\textwidth}
      \centering
      \includegraphics[width=1.0\textwidth]{figs/Selection/Ds_BDT_Var_Ds2KKPi_log10_D_P_IPCHI2_OWNPV.pdf}
   \end{subfigure}
   \begin{subfigure}[t]{0.22\textwidth}
      \centering
      \includegraphics[width=1.0\textwidth]{figs/Selection/Ds_BDT_Var_Ds2KKPi_D_K0_MC15TuneV1_ProbNNk.pdf}
   \end{subfigure}
   \begin{subfigure}[t]{0.22\textwidth}
      \centering
      \includegraphics[width=1.0\textwidth]{figs/Selection/Ds_BDT_Var_Ds2KKPi_D_K1_MC15TuneV1_ProbNNk.pdf}
   \end{subfigure}
   \begin{subfigure}[t]{0.22\textwidth}
      \centering
      \includegraphics[width=1.0\textwidth]{figs/Selection/Ds_BDT_Var_Ds2KKPi_D_P_MC15TuneV1_ProbNNk.pdf}
   \end{subfigure}
   \begin{subfigure}[t]{0.22\textwidth}
      \centering
      \includegraphics[width=1.0\textwidth]{figs/Selection/Ds_BDT_Var_Ds2KKPi_D_K0_MC15TuneV1_ProbNNp.pdf}
   \end{subfigure}
   \begin{subfigure}[t]{0.22\textwidth}
      \centering
      \includegraphics[width=1.0\textwidth]{figs/Selection/Ds_BDT_Var_Ds2KKPi_D_K1_MC15TuneV1_ProbNNp.pdf}
   \end{subfigure}
   \begin{subfigure}[t]{0.22\textwidth}
      \centering
      \includegraphics[width=1.0\textwidth]{figs/Selection/Ds_BDT_Var_Ds2KKPi_D_P_MC15TuneV1_ProbNNp.pdf}
   \end{subfigure}
   \begin{subfigure}[t]{0.22\textwidth}
      \centering
      \includegraphics[width=1.0\textwidth]{figs/Selection/Ds_BDT_Var_Ds2KKPi_D_K0_MC15TuneV1_ProbNNpi.pdf}
   \end{subfigure}
   \begin{subfigure}[t]{0.22\textwidth}
      \centering
      \includegraphics[width=1.0\textwidth]{figs/Selection/Ds_BDT_Var_Ds2KKPi_D_K1_MC15TuneV1_ProbNNpi.pdf}
   \end{subfigure}
   \begin{subfigure}[t]{0.22\textwidth}
      \centering
      \includegraphics[width=1.0\textwidth]{figs/Selection/Ds_BDT_Var_Ds2KKPi_D_P_MC15TuneV1_ProbNNpi.pdf}
   \end{subfigure}
   \begin{subfigure}[t]{0.22\textwidth}
      \centering
      \includegraphics[width=1.0\textwidth]{figs/Selection/Ds_BDT_Var_Ds2KKPi_D_K0_MC15TuneV1_ProbNNghost.pdf}
   \end{subfigure}
   \begin{subfigure}[t]{0.22\textwidth}
      \centering
      \includegraphics[width=1.0\textwidth]{figs/Selection/Ds_BDT_Var_Ds2KKPi_D_K1_MC15TuneV1_ProbNNghost.pdf}
   \end{subfigure}
   \begin{subfigure}[t]{0.22\textwidth}
      \centering
      \includegraphics[width=1.0\textwidth]{figs/Selection/Ds_BDT_Var_Ds2KKPi_D_P_MC15TuneV1_ProbNNghost.pdf}
   \end{subfigure}
   \begin{subfigure}[t]{0.22\textwidth}
      \centering
      \includegraphics[width=1.0\textwidth]{figs/Selection/Ds_BDT_Var_Ds2KKPi_D_K0_BDT_TRACK_VeloCHI2NDOF.pdf}
   \end{subfigure}
   \begin{subfigure}[t]{0.22\textwidth}
      \centering
      \includegraphics[width=1.0\textwidth]{figs/Selection/Ds_BDT_Var_Ds2KKPi_D_K1_BDT_TRACK_VeloCHI2NDOF.pdf}
   \end{subfigure}
   \begin{subfigure}[t]{0.22\textwidth}
      \centering
      \includegraphics[width=1.0\textwidth]{figs/Selection/Ds_BDT_Var_Ds2KKPi_D_P_BDT_TRACK_VeloCHI2NDOF.pdf}
   \end{subfigure}
   \begin{subfigure}[t]{0.22\textwidth}
      \centering
      \includegraphics[width=1.0\textwidth]{figs/Selection/Ds_BDT_Var_Ds2KKPi_D_K0_TRACK_CHI2NDOF.pdf}
   \end{subfigure}
   \begin{subfigure}[t]{0.22\textwidth}
      \centering
      \includegraphics[width=1.0\textwidth]{figs/Selection/Ds_BDT_Var_Ds2KKPi_D_K1_TRACK_CHI2NDOF.pdf}
   \end{subfigure}
   \begin{subfigure}[t]{0.22\textwidth}
      \centering
      \includegraphics[width=1.0\textwidth]{figs/Selection/Ds_BDT_Var_Ds2KKPi_D_P_TRACK_CHI2NDOF.pdf}
   \end{subfigure}
   \begin{subfigure}[t]{0.22\textwidth}
      \centering
      \includegraphics[width=1.0\textwidth]{figs/Selection/Ds_BDT_Var_Ds2KKPi_D_K0_TRACK_MatchCHI2.pdf}
   \end{subfigure}
   \begin{subfigure}[t]{0.22\textwidth}
      \centering
      \includegraphics[width=1.0\textwidth]{figs/Selection/Ds_BDT_Var_Ds2KKPi_D_K1_TRACK_MatchCHI2.pdf}
   \end{subfigure}
   \begin{subfigure}[t]{0.22\textwidth}
      \centering
      \includegraphics[width=1.0\textwidth]{figs/Selection/Ds_BDT_Var_Ds2KKPi_D_P_TRACK_MatchCHI2.pdf}
   \end{subfigure}
   \caption{Distributions of MVA training variables in the signal (blue) and background (red) samples for the Run~II \decay{\Dsp}{\Kp\Km\pip} MVA.}
   \label{fig:mvatrainingvariables_Ds}   
\end{figure}
%%%%%%%%%%%%%%%%%%%%%%%%%%%%%%%%%%%%%%%%%%%%%%%%%%%%%%%%%%


\subsubsection{MVA method}
\label{sec:mvamethod}

The MVA training and validation is implemented using the \tmva package as part of the \root framework.
During training, each method uses samples that have been defined to be `signal' and `background' in order to characterise the differences between the two categories. The multidimensional space defined by the input variables is effectively condensed onto a single axis given by the classifier's response. Typically, high values of the response correspond to `signal'-like events and low values to `background'-like events. This training produces a \emph{weights file} allowing the response to be calculated for other samples. 

A second set of `signal'-like and `background'-like samples are used to blindly validate the training. The classifier response is compared in the training and validation samples to demonstrate if the expected separation is reproducible. 

{\color{Red}
\begin{itemize}
\item Explain what overtraining is
\end{itemize}
}

The `signal' and `background' samples are both taken from the relevant \decay{\Bsb}{\Dsp\pim} or \decay{\Bs}{\jpsi\phiz} decay samples.
The signal samples include all events within the fit ranges shown in Fig.~\ref{fig:mvatrainingsamples}. Each candidate is weighted with the appropriate Gaussian fit component weight as determined by the \sPlot technique. This weight is passed to the \tmva framework when training the methods.
The background samples are comprised of the sideband regions of the relevant ranges in Fig.~\ref{fig:mvatrainingsamples}. These ranges are defined by $|m(\Kp\Km)-m(\phiz)| > 10 \mevcc$ and $|m(h^{+}h^{-}h^{+})-m(\Dsp)| > 30\mevcc$ for the \phiz and \Dsp MVAs respectively. No weights are used for the background samples as the regions are known to contain no signal events.


The efficacy of three MVA methods are compared and the most effective selected. The three methods are derivatives of Boosted Decision Trees (BDTs)~\cite{Breiman}.

{\color{Red}
\begin{itemize}
\item Detail of what BDT is?
\end{itemize}
}
% \begin{description}
% \item \textbf{BDT: (Adaptive Boost)} 
% \item \textbf{BDTG (Gradient Boosted Decision Tree):}
% \item \textbf{BDTB (bagged Boosted Decision Tree):}
% \end{description}

% {\color{Red}
% \begin{itemize}
% \item Not sure where the ROC curve comparison is at the mo...
% \end{itemize}
% }

The BDTG method is found to separate the signal and background most effectively and therefore is used in training all MVAs.
The exact configuration of the BDTG method is listed in Table~\ref{tab:mvaconfiguration}. 

{\color{Red}
\begin{itemize}
\item Explain configuration
\end{itemize}
}
\begin{table}[h]
\begin{center}
\begin{tabular}{ l l }
\hline
Parameter & Value \\
\hline
Number of trees         & 1000      \\
Minimum node size       & 2.5\%     \\
Boost type              & Gradient  \\ 
Shrinkage               & 0.10      \\ 
Bagged Boost            & True      \\ 
Bagged sample fraction  & 0.5       \\  
Number of cuts          & 20        \\  
Maximum depth           & 2         \\  
\hline
\end{tabular}
\end{center}
\caption{MVA configuration}
\label{tab:mvaconfiguration}
\end{table}  

%%%%%%%%%%%%%%%%%%%%%%%%%%%%%%%%%%%%%%%%%%%%%%%%%%%%%%%%%%
\begin{figure}[!h]
   \centering
   \begin{subfigure}[t]{0.32\textwidth}
      \centering
      \includegraphics[width=1.0\textwidth]{figs/Selection/Phi_BDT_classifier_Run1.pdf}
      \caption{Run I \decay{\phiz}{\Kp\Km}}
   \end{subfigure}
   \begin{subfigure}[t]{0.32\textwidth}
      \centering
      \includegraphics[width=1.0\textwidth]{figs/Selection/Phi_BDT_classifier_Run2.pdf}
      \caption{Run II \decay{\phiz}{\Kp\Km}}
   \end{subfigure}\\
   \begin{subfigure}[t]{0.32\textwidth}
      \centering
      \includegraphics[width=1.0\textwidth]{figs/Selection/Ds_BDT_classifier_Ds2KKPi_Run1.pdf}
      \caption{Run I \decay{\Dsp}{\Kp\Km\pip}}
   \end{subfigure}
   \begin{subfigure}[t]{0.32\textwidth}
      \centering
      \includegraphics[width=1.0\textwidth]{figs/Selection/Ds_BDT_classifier_Ds2KKPi_Run2.pdf}
      \caption{Run II \decay{\Dsp}{\Kp\Km\pip}}
   \end{subfigure}\\
   \begin{subfigure}[t]{0.32\textwidth}
      \centering
      \includegraphics[width=1.0\textwidth]{figs/Selection/Ds_BDT_classifier_Ds2PiPiPi_Run1.pdf}
      \caption{Run I \decay{\Dsp}{\pip\pim\pip}}
   \end{subfigure}
   \begin{subfigure}[t]{0.32\textwidth}
      \centering
      \includegraphics[width=1.0\textwidth]{figs/Selection/Ds_BDT_classifier_Ds2PiPiPi_Run2.pdf}
      \caption{Run II \decay{\Dsp}{\pip\pim\pip}}
   \end{subfigure}\\
   \begin{subfigure}[t]{0.32\textwidth}
      \centering
      \includegraphics[width=1.0\textwidth]{figs/Selection/Ds_BDT_classifier_Ds2KPiPi_Run1.pdf}
      \caption{Run I \decay{\Dsp}{\Kp\pim\pip}}
   \end{subfigure}
   \begin{subfigure}[t]{0.32\textwidth}
      \centering
      \includegraphics[width=1.0\textwidth]{figs/Selection/Ds_BDT_classifier_Ds2KPiPi_Run2.pdf}
      \caption{Run II \decay{\Dsp}{\Kp\pim\pip}}
   \end{subfigure}\\
   \caption{MVA classifier response for the eight MVAs trained for both signal (blue) and background (red). The training samples are represented by markers and the validation samples by filled histograms.}
   \label{fig:mvaclassifierresponses}   
\end{figure}
%%%%%%%%%%%%%%%%%%%%%%%%%%%%%%%%%%%%%%%%%%%%%%%%%%%%%%%%%%

The MVA classifier distributions for each of the eight MVAs trained are shown in Fig.~\ref{fig:mvaclassifierresponses}. This figure shows both the signal and background samples, as well as the training and validation subsamples. Each mode shows a good separation between the signal and background samples. For the majority of the distributions the training and testing samples have almost identical shapes, implying response is reproducible and the method has not overtrained. There are noticeable exceptions, however. The signal training and validation samples tend to show different distributions in the regions where the background distributions are maximal. This effect is likely to be a result of the use of weights in the MVA training. Ideally, the MVA classifier should remain positive across the whole range. The negative values implies that in that range the MVA classifier is correlated to the parameter used to generate the weights: the \phiz or \Dsp mass. As this only affects the range of MVA classifier values that are dominantly `background'-like, it likely this discrepancy is due specifically to presence of weighted background events in the signal sample passed to \tmva.
This effect is further studied to determine if it results in any systematic uncertainty in the MVA efficiencies. In all cases the discrepancies are far below the range of values that selection requirements are placed. 


{\color{Red}
\begin{itemize}
\item Comparison to alternative? 
\end{itemize}
}

\subsubsection{MVA efficiency}
\label{sec:selection_MVA_eff}


The efficiencies of the MVAs on \Dsp and \phiz meson in the signal decays can be obtained from the validation samples of \decay{\Bs}{\jpsi\phiz} and \decay{\Bsb}{\Dsp\pim} decays. Additionally, a sample of \decay{\Bp}{\Dzb\pip} decays is used to calculate the efficiency of \decay{\Dzb}{\Kp\Km} decays in the normalisation channel. The efficiency calculation takes into account the kinematic differences between the training and signal samples, as well as any possible correlations between the \Dsp and \phiz kinematics, by using input from simulation samples. 
The signal MVA response is extracted from the validation samples in bins of \pt and $\chi^2_{\text{FD}}$. {\color{Red} How many?}. Simulation samples for \decay{\Bp}{\Dsp\phiz} or \decay{\Bp}{\Dsp\Dzb} decays are iterated though in turn, selecting the corresponding MVA response from the appropriate kinematic bins. The per-candidate efficiency is determined by integrating the validation sample \Dsp and \phiz MVA responses above the required cut values
\begin{equation}
\varepsilon_{\text{event}} = \int\limits_{\text{cut}_{\phiz}}^{1} f(x) dx \times  \int\limits_{\text{cut}_{\Dsp}}^{1} g(y) dy
\label{eq:mva_eff}
\end{equation}
where $f(x)$ and $g(y)$ represent the weighted MVA classifier output for the \phiz and \Dsp MVAs respectively, and $\text{cut}_{\phiz}$ and $\text{cut}_{\Dsp}$ represent the chosen MVA cut value.
The total efficiency for each mode is given by the sum of the per-candidate efficiencies within the relevant simulation sample.


The \decay{\phiz}{\Kp\Km} MVA is also used to select the \Kp\Km pair in \decay{\Bp}{\Dsp\Kp\Km} decays. Due to the similar topologies of \decay{\Bp}{\Dsp\Kp\Km} and \decay{\Bp}{\Dsp\phiz} decays, the \Kp\Km pair share lots of similarities to the \decay{\phiz}{\Kp\Km} counterparts. One obvious difference is the invariant mass of the pair, as the \decay{\Bp}{\Dsp\Kp\Km} decays includes a much larger phase-space. In training the MVA this could lead to the selection being non-optimal for \decay{\Bp}{\Dsp\Kp\Km} decays at high $m(\Kp\Km)$. However, when determining the efficiency of the selection using these same \decay{\Bs}{\jpsi\phi} decays these differences could lead to bias. Instead, calibration samples are used to correct for the imperfect modelling of the particle identification in the \decay{\Bp}{\Dsp\Kp\Km} simulation samples. These corrected simulations are then used to obtain the variations in the MVA efficiencies as a function of the phase-space position, in particular of the $m(\Kp\Km)$ invariant mass as discussed in Section~\ref{sec:B2DsKK_effcorrection}.

{\color{Red}
\begin{itemize}
\item Crosschecks for eff?
\end{itemize}
}

\subsubsection{MVA optimisation}

The selection criteria for each of the BDTG classifiers are determined by optimising the Punzi figure of merit~\cite{Punzi:2003bu}, 
\begin{equation}
\frac{\epsilon_{s}}{(\frac{a}{2} + \sqrt{N_{\text{BKG}}})}
\end{equation}
with $a=5$, where $\epsilon_{s}$ is the signal efficiency and $N_{\text{BKG}}$ is the number of background candidates determined from fits to data, calculated in the signal region. 
The signal efficiency is calculated for the relevant mode and MVA requirement values using the data-driven technique previously described. 
The optimisation is performed in two dimensions, one for each of the \phiz and \Dsp MVAs. This is performed separately for the three different \Dsp decay modes. The results of the optimisation for each \Dsp decay mode and running period are shown in Fig.~\ref{fig:mvaoptmisation}. 

%%%%%%%%%%%%%%%%%%%%%%%%%%%%%%%%%%%%%%%%%%%%%%%%%%%%%%%%%%
\begin{figure}[!h]
   \centering
   \begin{subfigure}[t]{0.4\textwidth}
      \centering
      \includegraphics[width=1.0\textwidth]{figs/Selection/Ds2KKPi_BDTG_punzi_Run1_cont.pdf}
      \caption{Run I \decay{\Dsp}{\Kp\Km\pip}}
   \end{subfigure}
   \begin{subfigure}[t]{0.4\textwidth}
      \centering
      \includegraphics[width=1.0\textwidth]{figs/Selection/Ds2KKPi_BDTG_punzi_Run2_cont.pdf}
      \caption{Run II \decay{\Dsp}{\Kp\Km\pip}}
   \end{subfigure}\\
   \begin{subfigure}[t]{0.4\textwidth}
      \centering
      \includegraphics[width=1.0\textwidth]{figs/Selection/Ds2PiPiPi_BDTG_punzi_Run1_cont.pdf}
      \caption{Run I \decay{\Dsp}{\pip\pim\pip}}
   \end{subfigure}
   \begin{subfigure}[t]{0.4\textwidth}
      \centering
      \includegraphics[width=1.0\textwidth]{figs/Selection/Ds2PiPiPi_BDTG_punzi_Run2_cont.pdf}
      \caption{Run II \decay{\Dsp}{\pip\pim\pip}}
   \end{subfigure}\\
   \begin{subfigure}[t]{0.4\textwidth}
      \centering
      \includegraphics[width=1.0\textwidth]{figs/Selection/Ds2KPiPi_BDTG_punzi_Run1_cont.pdf}
      \caption{Run I \decay{\Dsp}{\Kp\pim\pip}}
   \end{subfigure}
   \begin{subfigure}[t]{0.4\textwidth}
      \centering
      \includegraphics[width=1.0\textwidth]{figs/Selection/Ds2KPiPi_BDTG_punzi_Run2_cont.pdf}
      \caption{Run II \decay{\Dsp}{\Kp\pim\pip}}
   \end{subfigure}\\
   \caption{MVA optimsation}
   \label{fig:mvaoptmisation}   
\end{figure}
%%%%%%%%%%%%%%%%%%%%%%%%%%%%%%%%%%%%%%%%%%%%%%%%%%%%%%%%%%

The optimal values are selected separately for each of the modes and tabulated in Table~\ref{table:mvarequirementvalues}.

\begin{table}[!h]
\begin{center}
\begin{tabular}{ l l  c  c  }

\hline
Period   & \Dsp decay mode               & \Dsp requirement & \phiz/\Dzb requirement \\ 
\hline
Run I    & \decay{\Dsp}{\Kp\Km\pip}      & 0.5 & 0.8 \\
         & \decay{\Dsp}{\pip\pim\pip}    & 0.3 & 0.8 \\
         & \decay{\Dsp}{\Kp\pim\pip}     & 0.2 & 0.8 \\ 
\hline
Run II   & \decay{\Dsp}{\Kp\Km\pip}      & 0.5 & 0.8 \\
         & \decay{\Dsp}{\pip\pim\pip}    & 0.2 & 0.8 \\
         & \decay{\Dsp}{\Kp\pim\pip}     & 0.2 & 0.8 \\                                      
\hline
\end{tabular}
\caption{Optimised MVA cuts for the different \Dsp decay modes and datasets. }
\label{table:mvarequirementvalues}
\end{center}
\end{table}

\subsection{Impact parameter requirements}
\label{sec:selection_IPCHI2}
The data-driven MVAs target the selection of \phiz and \Dsp candidates, removing a significant background contribution. However it is possible to further purify the candidates by making requirements on the \Bp candidates formed from the combination of the two. This helps to remove possible backgrounds coming from the incorrect combination of two unrelated mesons. Requirements are places on the impact parameter significance $\chi^{2}_{\text{IP}}$ to help ensure that the different components of the decay chain have a topology consistent with those desired.

The \Bp meson candidate is required to be consistent with originating at the PV by requiring $\chi^2_{\text{IP}} < 10$. This helps to remove background as unrelated combinations of \Dsp and \phiz mesons will not necessarily point back towards the PV.
Additionally, the opposite requirement is made of the \Dsp meson, $\chi^2_{\text{IP}} > 10$. It is possible for \Dsp mesons to be produced promptly at the PV rather than in a \Bp meson decay and this requirement helps to remove those candidates.



%%%%%%%%%%%%%%%%%%%%%%%%%%%%%%%%%%%%%%%%%%%%%%%%%%%%%%%%%%
\begin{figure}[!h]
   \centering
   \begin{subfigure}[t]{0.32\textwidth}
      \centering
      \includegraphics[width=1.0\textwidth]{figs/Selection/Data_MC_Comparison_Var_1_B2DsPhi_Ds2KKPi.pdf}
      \caption{\decay{\Dsp}{\Kp\Km\pip}}
   \end{subfigure}
   \begin{subfigure}[t]{0.32\textwidth}
      \centering
      \includegraphics[width=1.0\textwidth]{figs/Selection/Data_MC_Comparison_Var_1_B2DsPhi_Ds2PiPiPi.pdf}
      \caption{\decay{\Dsp}{\pip\pim\pip}}
   \end{subfigure}
   \begin{subfigure}[t]{0.32\textwidth}
      \centering
      \includegraphics[width=1.0\textwidth]{figs/Selection/Data_MC_Comparison_Var_1_B2DsPhi_Ds2KPiPi.pdf}
      \caption{\decay{\Dsp}{\Kp\pim\pip}}
   \end{subfigure}\\
   \begin{subfigure}[t]{0.32\textwidth}
      \centering
      \includegraphics[width=1.0\textwidth]{figs/Selection/Data_MC_Comparison_Var_2_B2DsPhi_Ds2KKPi.pdf}
      \caption{\decay{\Dsp}{\Kp\Km\pip}}
   \end{subfigure}
   \begin{subfigure}[t]{0.32\textwidth}
      \centering
      \includegraphics[width=1.0\textwidth]{figs/Selection/Data_MC_Comparison_Var_2_B2DsPhi_Ds2PiPiPi.pdf}
      \caption{\decay{\Dsp}{\pip\pim\pip}}
   \end{subfigure}
   \begin{subfigure}[t]{0.32\textwidth}
      \centering
      \includegraphics[width=1.0\textwidth]{figs/Selection/Data_MC_Comparison_Var_2_B2DsPhi_Ds2KPiPi.pdf}
      \caption{\decay{\Dsp}{\Kp\pim\pip}}
   \end{subfigure}\\
   \caption{The \Bp and \Dsp meson $\chi^{2}_{\text{IP}}$ distributions for \decay{\Bp}{\Dsp\phiz} candidates in data (black) and simulation (red). Candidates within the range $4900 < m(\Dsp\phiz) < 5900 \mevcc$ are included in these figures.}
   \label{fig:ipchi2dist_signal}   
\end{figure}
%%%%%%%%%%%%%%%%%%%%%%%%%%%%%%%%%%%%%%%%%%%%%%%%%%%%%%%%%%

The distributions for the \Bp and \Dsp meson impact parameter significance is shown in Fig.~\ref{fig:ipchi2dist_signal} for \decay{\Bp}{\Dsp\phiz} candidates. The requirement values are represented by vertical lines. Small peaks can be observed at low \Dsp meson impact parameter significance, corresponding to \Dsp mesons that originated at the PV. 
As a cross-check, the same distributions are produced for the normalisation channel, \decay{\Bp}{\Dsp\Dzb}. However, instead of including all candidates in a wide $m(\Dsp\phiz)$ mass range, only those within $20\mevcc$ of the \Bp mass are included. As the normalisation sample has a very high purity this effectively isolates these decays in data. This comparison of data and simulation distributions is shown in Fig.~\ref{fig:ipchi2dist_normalisation}.

%%%%%%%%%%%%%%%%%%%%%%%%%%%%%%%%%%%%%%%%%%%%%%%%%%%%%%%%%%
\begin{figure}[!h]
   \centering
   \begin{subfigure}[t]{0.32\textwidth}
      \centering
      \includegraphics[width=1.0\textwidth]{figs/Selection/Data_MC_Comparison_Var_1_B2DsD0_Ds2KKPi.pdf}
      \caption{\decay{\Dsp}{\Kp\Km\pip}}
   \end{subfigure}
   \begin{subfigure}[t]{0.32\textwidth}
      \centering
      \includegraphics[width=1.0\textwidth]{figs/Selection/Data_MC_Comparison_Var_1_B2DsD0_Ds2PiPiPi.pdf}
      \caption{\decay{\Dsp}{\pip\pim\pip}}
   \end{subfigure}
   \begin{subfigure}[t]{0.32\textwidth}
      \centering
      \includegraphics[width=1.0\textwidth]{figs/Selection/Data_MC_Comparison_Var_1_B2DsD0_Ds2KPiPi.pdf}
      \caption{\decay{\Dsp}{\Kp\pim\pip}}
   \end{subfigure}\\
   \begin{subfigure}[t]{0.32\textwidth}
      \centering
      \includegraphics[width=1.0\textwidth]{figs/Selection/Data_MC_Comparison_Var_2_B2DsD0_Ds2KKPi.pdf}
      \caption{\decay{\Dsp}{\Kp\Km\pip}}
   \end{subfigure}
   \begin{subfigure}[t]{0.32\textwidth}
      \centering
      \includegraphics[width=1.0\textwidth]{figs/Selection/Data_MC_Comparison_Var_2_B2DsD0_Ds2PiPiPi.pdf}
      \caption{\decay{\Dsp}{\pip\pim\pip}}
   \end{subfigure}
   \begin{subfigure}[t]{0.32\textwidth}
      \centering
      \includegraphics[width=1.0\textwidth]{figs/Selection/Data_MC_Comparison_Var_2_B2DsD0_Ds2KPiPi.pdf}
      \caption{\decay{\Dsp}{\Kp\pim\pip}}
   \end{subfigure}\\
   \caption{The \Bp and \Dsp meson $\chi^{2}_{\text{IP}}$ distributions for \decay{\Bp}{\Dsp\Dzb} candidates in data (black) and simulation (red). Candidates within the range $|m(\Dsp\phiz) - m(\Bp)| < 20 \mevcc$ are included in these figures, effectively isolating the normalisation events in data.}
   \label{fig:ipchi2dist_normalisation}   
\end{figure}
%%%%%%%%%%%%%%%%%%%%%%%%%%%%%%%%%%%%%%%%%%%%%%%%%%%%%%%%%%



{\color{Red}
\begin{itemize}
\item norm channel systematic 
\end{itemize}
}


\section{Multiple candidates}
\label{sec:multiplecandidates}

Signal and normalisation candidates are constructed by combining five of the large number of tracks in each event in different combinations. It is possible that after the many steps of selection have been performed, that there are still multiple candidates reconstructed in a single event. It is, of course, possible that there are more than one \B meson decays in a single event. Indeed production via \bquark\bquarkbar pairs would lead to two \B hadrons in the same event. However both the singal and normalisation channels searched for here have relatively small branching fractions, therefore the probability for both \Bp mesons to decay the same final state is low. 
The rate of multiple candidates is studied for each of the samples after passing through the full selection. The run and event number of each candidate is compared, and the number of events that contain more than one candidate is divided by the total number of candidates to produce a rate. These are tabulated in Table~\ref{table:multiplecandidates}.
\begin{table}[!h]
\begin{center}
\begin{tabular}{ l l c }

\hline
\Bp decay mode          & \Dsp decay mode                & Multiple candidate rate     \\ 
\hline
\decay{\Bp}{\Dsp\phiz}  & \decay{\Dsp}{\Kp\Km\pip}       & 0.33\%                      \\
                        & \decay{\Dsp}{\pip\pim\pip}     & 0.46\%                      \\
                        & \decay{\Dsp}{\Kp\pim\pip}      & 0.00\%                      \\
\hline
\decay{\Bp}{\Dsp\Dzb}   & \decay{\Dsp}{\Kp\Km\pip}       & 0.27\%                      \\
                        & \decay{\Dsp}{\pip\pim\pip}     & 0.11\%                      \\
                        & \decay{\Dsp}{\Kp\pim\pip}      & 0.17\%                      \\
\hline
\decay{\Bp}{\Dsp\Kp\Km} & \decay{\Dsp}{\Kp\Km\pip}       & 0.25\%                      \\
\hline
\decay{\Bp}{\Dsp\Dzb}   & \decay{\Dsp}{\Kp\Km\pip}       & 0.21\%                      \\
\hline
\end{tabular}
\caption{Multiple candidate rates for }
\label{table:multiplecandidates}
\end{center}
\end{table}
The rates of multiple candidates are sufficiently low that no attempt is made to remove them.

\section{Refitting the decay chain}
\label{sec:decaytreefitter}

The yields of \Bp meson candidates for the signal and normalisation channels are determined from fits to the \Bp meson invariant mass, the exact details of which will be in Chapters~\ref{ch:B2DsKK} and \ref{ch:B2DsPhi}. 
The \emph{raw} \Bp meson invariant mass is calculated in the vertex fits used to constructed the \Bp candidates from combinations of \Dsp and \phiz or \Dzb candidates. The 4-momentum vectors for the decay products trajectories are extrapolated to the vertex location and the invariant mass calculated from the sum of the decay products 4-vectors. The width of the \Bp meson distribution is dominated by the resolution of the \lhcb detector, as the natural width of the meson itself is $~0.4\times 10^{-3}\ev$.

It is possible to reduce the effect of the resolution by adding more constraints in the fit to the decay chain. The mass of the \Dsp meson (and \Dzb meson for the normalisation channel) are constrained to their known values. Additionally, the momentum vector of the \Bp meson is constrained to be parallel to the line between the primary interaction and the \Bp meson decay vertex. 
This is implemented with the Decay Tree Fitter algorithm \cite{Hulsbergen:2005pu}.
With these constraints applied, the whole decay chain is refitted, and the \Bp meson mass recalculated. For genuine signal and normalisation decays the resolution is improved; the smearing effects that might lead the track momentum or position to be incorrectly measured can be circumvented by including the extra information about indeterminate meson masses, or \Bp flight direction. Background candidates that have different topologies to the constrained decay chain will be randomly shifted up or down in invariant mass, leading to no overall difference in the combinatorial distribution.    


It is possible to calculate all kinematic distributions for the components of the decay chain with these constraints applied, for example the momentum of the decay products. However, these variables are not used when training the MVA methods even though they may result in a greater discrimination between signal and background. The modes used to train the MVA methods have slightly different topologies, therefore the improvements in the resolution achieved in these modes may not be the same as that seen for the signal and normalisation channel. 


{\color{Red}
\begin{itemize}
\item plot of \Bp meson mass with and without constraint?
\item DTF for helicity angle
\item DTF for dalitz plot
\end{itemize}
}


\section{Invariant mass distributions}
\label{sec:invarinantmassdistros}


The \Dsp and \phiz meson invariant mass distributions for signal \decay{\Bp}{\Dsp\phiz} candidates passing the full selection requirements are shown in Fig.~\ref{fig:d_phi_mass_signal}. The same distributions for the normalisation channel \decay{\Bp}{\Dsp\Dzb} are shown in Fig.~\ref{fig:d_phi_mass_normlaisation}. The purity of the \decay{\Dsp}{\pip\pim\pip} and \decay{\Dsp}{\Kp\pim\pip} decays are significantly lower than the \decay{\Dsp}{\Kp\Km\pip} mode as a result of the smaller branching fractions for these modes. 

Requirements are placed in the invariant masses of the \Dsp, \phiz and \Dzb candidates to isolate the signal decays. The requirements are listed in Table~\ref{table:masscuts} and represented in Figs.~\ref{fig:d_phi_mass_signal} and~\ref{fig:d_phi_mass_normlaisation} by red vertical lines. The \phiz meson requirement is wider than those for the \Dsp and \Dzb mesons even though the width is clearly narrower. The \decay{\Bp}{\Dsp\phiz} candidates are further split into two categories as detailed in Chapter~\ref{ch:B2DsPhi}, therefore the sidebands either side of the \phiz meson are retained.

\begin{table}[!h]
\begin{center}
\begin{tabular}{ c c }

\hline
Particle          & Requirement                     \\
\hline
\Dsp              & $|m(h^{+}h^{-}h^{+})-m(\Dsp)| < 25 \mevcc$ \\
\Dzb              & $|m(\Kp\Km)-m(\Dzb)| < 25 \mevcc$          \\
\phiz             & $|m(\Kp\Km)-m(\phiz)| < 40 \mevcc$ \\
\hline
\end{tabular}
\caption{Invariant mass requirements applied to all candidates.}
\label{table:masscuts}
\end{center}
\end{table}

%%%%%%%%%%%%%%%%%%%%%%%%%%%%%%%%%%%%%%%%%%%%%%%%%%%%%%%%%%
\begin{figure}[!h]
   \centering
   \begin{subfigure}[t]{1.0\textwidth}
   \centering
     \begin{subfigure}[t]{0.35\textwidth}
        \centering
        \includegraphics[width=1.0\textwidth]{figs/Selection/Phimass_KKPi_B2DsPhi.pdf}
     \end{subfigure}
     \begin{subfigure}[t]{0.35\textwidth}
        \centering
        \includegraphics[width=1.0\textwidth]{figs/Selection/Dmass_KKPi_B2DsPhi.pdf}
     \end{subfigure}
     \caption{\decay{\Dsp}{\Kp\Km\pip}}
   \end{subfigure}   
   \begin{subfigure}[t]{1.0\textwidth}
   \centering
     \begin{subfigure}[t]{0.35\textwidth}
        \centering
        \includegraphics[width=1.0\textwidth]{figs/Selection/Phimass_PiPiPi_B2DsPhi.pdf}
     \end{subfigure}
     \begin{subfigure}[t]{0.35\textwidth}
        \centering
        \includegraphics[width=1.0\textwidth]{figs/Selection/Dmass_PiPiPi_B2DsPhi.pdf}
     \end{subfigure}
     \caption{\decay{\Dsp}{\pip\pim\pip}}
   \end{subfigure}   
   \begin{subfigure}[t]{1.0\textwidth}
   \centering
     \begin{subfigure}[t]{0.35\textwidth}
        \centering
        \includegraphics[width=1.0\textwidth]{figs/Selection/Phimass_KPiPi_B2DsPhi.pdf}
     \end{subfigure}
     \begin{subfigure}[t]{0.35\textwidth}
        \centering
        \includegraphics[width=1.0\textwidth]{figs/Selection/Dmass_KPiPi_B2DsPhi.pdf}
     \end{subfigure}
     \caption{\decay{\Dsp}{\Kp\pim\pip}}
   \end{subfigure}
   \caption{The \Dsp and \phiz invariant mass distributions for signal \decay{\Bp}{\Dsp\phiz} candidates passing the selection requirements. The inner and outer $m(\Kp\Km)$ ranges, as defined in Chapter~\ref{ch:B2DsPhi}, are represented by the blue and red vertical lines respectively.}
   \label{fig:d_phi_mass_signal}   
\end{figure}
%%%%%%%%%%%%%%%%%%%%%%%%%%%%%%%%%%%%%%%%%%%%%%%%%%%%%%%%%%
%%%%%%%%%%%%%%%%%%%%%%%%%%%%%%%%%%%%%%%%%%%%%%%%%%%%%%%%%%
\begin{figure}[!h]
   \centering
   \begin{subfigure}[t]{1.0\textwidth}
   \centering
     \begin{subfigure}[t]{0.35\textwidth}
        \centering
        \includegraphics[width=1.0\textwidth]{figs/Selection/Phimass_KKPi_B2DsD0.pdf}
     \end{subfigure}
     \begin{subfigure}[t]{0.35\textwidth}
        \centering
        \includegraphics[width=1.0\textwidth]{figs/Selection/Dmass_KKPi_B2DsD0.pdf}
     \end{subfigure}
     \caption{\decay{\Dsp}{\Kp\Km\pip}}
   \end{subfigure}   
   \begin{subfigure}[t]{1.0\textwidth}
   \centering
     \begin{subfigure}[t]{0.35\textwidth}
        \centering
        \includegraphics[width=1.0\textwidth]{figs/Selection/Phimass_PiPiPi_B2DsD0.pdf}
     \end{subfigure}
     \begin{subfigure}[t]{0.35\textwidth}
        \centering
        \includegraphics[width=1.0\textwidth]{figs/Selection/Dmass_PiPiPi_B2DsD0.pdf}
     \end{subfigure}
     \caption{\decay{\Dsp}{\pip\pim\pip}}
   \end{subfigure}   
   \begin{subfigure}[t]{1.0\textwidth}
   \centering
     \begin{subfigure}[t]{0.35\textwidth}
        \centering
        \includegraphics[width=1.0\textwidth]{figs/Selection/Phimass_KPiPi_B2DsD0.pdf}
     \end{subfigure}
     \begin{subfigure}[t]{0.35\textwidth}
        \centering
        \includegraphics[width=1.0\textwidth]{figs/Selection/Dmass_KPiPi_B2DsD0.pdf}
     \end{subfigure}
     \caption{\decay{\Dsp}{\Kp\pim\pip}}
   \end{subfigure}
   \caption{The \Dsp and \Dzb invariant mass distributions for normalisation \decay{\Bp}{\Dsp\Dzb} candidates passing the selection requirements.}
   \label{fig:d_phi_mass_normlaisation}   
\end{figure}
%%%%%%%%%%%%%%%%%%%%%%%%%%%%%%%%%%%%%%%%%%%%%%%%%%%%%%%%%%



The distribution of \Dsp and \Kp\Km invariant masses for \decay{\Bp}{\Dsp\Kp\Km} and the corresponding \decay{\Bp}{\Dsp\Dzb} candidates are shown in Fig.~\ref{fig:d_KK_mass}.

%%%%%%%%%%%%%%%%%%%%%%%%%%%%%%%%%%%%%%%%%%%%%%%%%%%%%%%%%%
\begin{figure}[!h]
   \centering
   \begin{subfigure}[t]{1.0\textwidth}
      \centering
      \begin{subfigure}[t]{0.40\textwidth}
         \centering
         \includegraphics[width=1.0\textwidth]{figs/Selection/Phimass_KKPi_B2DsKK.pdf}
      \end{subfigure}
      \begin{subfigure}[t]{0.40\textwidth}
         \centering
         \includegraphics[width=1.0\textwidth]{figs/Selection/Dmass_KKPi_B2DsKK.pdf}
      \end{subfigure}
         \caption{Signal \decay{\Bp}{\Dsp\Kp\Km} decays}
   \end{subfigure} 
   \begin{subfigure}[t]{1.0\textwidth}
      \centering
      \begin{subfigure}[t]{0.40\textwidth}
         \centering
         \includegraphics[width=1.0\textwidth]{figs/Selection/Phimass_KKPi_B2DsD0.pdf}
         \caption{\decay{\Dsp}{\Kp\Km\pip}}
      \end{subfigure}
      \begin{subfigure}[t]{0.40\textwidth}
         \centering
         \includegraphics[width=1.0\textwidth]{figs/Selection/Dmass_KKPi_B2DsD0.pdf}
      \end{subfigure}
      \caption{Normalisation \decay{\Bp}{\Dsp\Dzb} decays}
   \end{subfigure}
   \caption{The \Dsp and \Kp\Km invariant mass distributions for \decay{\Bp}{\Dsp\Kp\Km} candidates passing the selection requirements.}
   \label{fig:d_KK_mass}   
\end{figure}
%%%%%%%%%%%%%%%%%%%%%%%%%%%%%%%%%%%%%%%%%%%%%%%%%%%%%%%%%%

%%%%%%%%%%%%%%%%%%%%%%%%%%%%%%%%%%%%%%%%%%%%%%%%%%%%%%%%%%
\begin{figure}[!h]
    \centering
        \includegraphics[width=0.8\textwidth]{figs/Selection/Normalisation_with_sel_B2DsD0.pdf}
    \caption{Various selection steps applied to the normalisation channel.}
    \label{fig:norm_selection}   
\end{figure}
%%%%%%%%%%%%%%%%%%%%%%%%%%%%%%%%%%%%%%%%%%%%%%%%%%%%%%%%%%


