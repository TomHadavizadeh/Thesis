\documentclass[12pt]{article}
%\documentclass[12pt]{../ociamthesis}
\usepackage{color}
%\usepackage{hyperref}
\usepackage[margin=1.0in]{geometry}
\usepackage{amsmath}
\title{Minor corrections\\
Thesis: Rare hadronic decays of B mesons at LHCb}
\author{Tom Hadavizadeh}

\begin{document}
\maketitle

This document lists the changes that have been made to the thesis as minor corrections. Changes are highlighted in red. Page numbers refer to the version of the thesis used in the Viva. 

%%%%%%%%%%%%%%%%%%%%%%%%%%%%%%%%%%%%%%%%%%%%%%%%%%%%%
\section{Acknowledgements}

\begin{itemize}
\item {\color{blue} Add acknowledgements section}
\item {\color{blue} Add quote}

\end{itemize}
\section{Introduction}



\begin{itemize}
\item Page 4: ... have {\color{red}an} electromagnetic charge...
\item Page 4: ... there exists {\color{red}an} antimatter...
\item Page 6: ... described {\color{red}as} a non-Abelian...
\item Page 13: ... production {\color{red}of} Gravitons...
\end{itemize}

\section{Theory}
\begin{itemize}
\item Page 15: ... Ground state hadrons {\color{red}containing a single b-quark} can only...
\item Page 15: Added citation for Fermi coupling.
\item Page 16: ... decay {\color{red}was} measured ... 
\item Page 17: {\color{blue}cite ratio}
\item Page 18: Added the definitions of $\lambda$, $A$, $\rho$, $\eta$ in terms of $V_{qq}$
\item Page 18: Double checked sizes of diagonal CKM elements
\item Page 18: {\color{blue} add description of unitarity triangle}
\item Page 19: Added details about how $K^{0}$ and $\bar{K}^{0}$ mix into $K_{S}^{0}$ and $K_{L}^{0}$
\item Page 19: Added details of the different types of CP violation and stated that direct CP violation is the only relevant method.
\item Page 20: Added explaination to figre 2.1 caption describing what colour suppressed or favoured means.
\item Page 21: ...expensive {\color{red}and} requires the...
\item Page 22: ...mass {\color{red}of the} charged lepton...

\item Page 22: Clarified introduction to pure annihilation decays
%\item Page 22: {\color{blue} what do you mean by mutually exclusive?}
\item Page 23: Added minus sign to limit order of magnitude $6.1\times10^{{\color{red}-}7}$
\item Page 23: .. this can {\color{red}potentially} limit the sensitivity {\color{red}to} the...
\item Page 23: Added chapter reference when mentioning $B^+ \rightarrow D^+_s \phi$
\item Page 24: removed {\color{red}dominantly}
\item Page 27: ... SM values given {\color{red}in Table 2.2} ...
\end{itemize}
\section{The LHCb experiment}

\begin{itemize}
\item Page 30: ... at the {\color{red}shallowest and deepest} points...
\item Page 30: Added LEP reference
\item Page 34: Added LHCb detector reference
\item Page 35: Added description of sub-detectors to figure 3.2 caption.
\item Page 36: Added reference for bb production plot figure
\item Page 39: Defined FPGA
\item Page 42: Added to figure caption explaining which of the planes (X1, U, V, or X2) is shown in Figure 3.9
\item Page 46: ...in the {\color{red}maximum} value...
\item Page 48: Defined calorimeter acronyms
\item Page 51: ...to determine {\color{red}if} the innermost...
\item Page 53: ...trigger is composed {\color{red}of} two parts...
\item Page 54: A table of the hardware trigger thresholds in 2012 has been added for completeness.
\item Page 54: ...pattern recognition {\color{red}to} identify track candidates...
\item Page 55: Added figure and reference showing corrected mass variable for B decays
\end{itemize}

\section{Event selection}

\begin{itemize}
\item Page 64: ...normalisation channel $B^+ \rightarrow D_s^+ \overline{D}^0$ {\color{red}with $\overline{D}^0 \rightarrow K^+K^-$.}
\item Page 65: {\color{red}All signal and normalisation simulation samples} are generated...
\item Page 65: ...Samples are {\color{red}generated} assuming...
\item Page 65: ...requiring the {\color{red}$\overline{D}^0$ mesons to decay to the $K^+K^-$ final state and the $D_s^+$ meson to one of $K^+K^-\pi^+$, $K^+\pi^-\pi^+$ or $\pi^+\pi^-\pi^+$.}
\item Page 66: Removed: {\color{red}The event selection aims to reduce overall rate of collisions, whilst maximising the signal efficiency.} 
\item Page 66: Added: {\color{red} However, all hardware triggers are implicitly used to select the signal candidates as it is possible for a unrelated interaction to have initiated the trigger. The reconstructed objects are then associated to the trigger decisions to determine if they were responsible.}
\item Page 66: Added: {\color{red}For a relatively simple trigger candidates, such as the combination of tracks considered in these hadronic decays, at least 70\% of online VELO and IT/OT hits must be in the total set of offline tracking hits. For more complicated composite candidates, for example decays containing electrons with both tracks and calorimeter deposits, the combination of the individual trigger candidates is compared to the set of offline candidates.} 
\item Page 67: ...if the {\color{red}deposit} matched to a reconstructed...
\item Page 68: ...hadronic {\color{red}hardware} trigger (\texttt{L0Hadron})...
\item Page 69: ...hardware sub-systems {\color{red}(Muon, DiMuon, Electron, Photon or Hadron)} is...
\item Page 70: Corrected totals in Table 4.4.
\item Page 72: ...products {\color{red}with respect to the proton beam axis}... 
\item Page 74: ...track $\chi^{2}/N_{\text{DOF}}$ is required to be below $4.0$ {\color{red}to remove poorly reconstructed tracks}...
\item Page 76: Added: {\color{red}These requirements have a combined efficiency of between 85-90\%, depending on $D_s^+$ decay mode.}
\item Page 79: Added: {\color{red} As illustrated in Fig. 4.9, the blue and red regions in the $D^0$ sidebands are wider to the left of the peak than the right. This helps to prevent misidentified $B^+\rightarrow D_s^+ (D^0 \rightarrow K^-\pi^+)$ decays from being included in the sideband sample, visible as an excess of points at high values of $m(K^+K^-)$.}
\item Page 80: Added to table 4.6 caption: {\color{red} The selection requirements are chosen such that the contribution from charmless and single charm decays are below 2\% of the normalisation yield. The large variation in residual yields therefore reflects reflects the variation in the expected normalisation yields.}
\item Page 81: {\color{blue}Explain shape in caption}
\item Page 82: Added: {\color{red} These result from various signal or background processes in which incorrect tracks have been assigned to the the final state particles.}
\item Page 84: Clarified introduction to MVA methods.
\item Page 88: Added: {\color{red}These variables improve the performance over the variables discussed in Section 4.3.2 by using input from all PID sub-detectors and exploiting any correlations.}
\item Page 90: ... {\color{red}to account for the difference in the kinematics and geometry of the validation and signal modes}...
\item Page 90: Add reference to pages where efficiencies are discussed.
\item Page 90: Added: {\color{red} This figure of merit is chosen as it depends only on the signal efficiency rather than the absolute number of expected signal events. The parameter $a$ defines the number of standard deviations corresponding to a one-sided Gaussian test, here chosen to be five to correspond to the significance required for an observation.}
\item Page 91: ...Requirements are {\color{red}placed} on the impact... 
\item Page 93: Caption: The {\color{red}$\phi$ (left) and $D_{s}^{+}$ (right)} invariant mass
\item Page 93: ...normalisation channel {\color{red}decreases as the selection is applied, but only reduces to around half of the original size}...
\item Page 94: Caption: The {\color{red}$D^{0}$ (left) and $D_{s}^{+}$ (right)} invariant mass
\item Page 95: Caption: The {\color{red}$K^+K^-$ (left) and $D_{s}^{+}$ (right)} invariant mass


\end{itemize}
\section{Mass fit to $B^+ \rightarrow D_s^+ K^+K^-$ candidates}

\begin{itemize}
\item Page 106: ...The parameters of interest {\color{red}in the two fits are}...
\item Page 106: Added: {\color{red} The error in the normalisation yield is found to be overestimated by the fit model. When propagated to the final branching fraction, this effect is reduced, therefore no attempt is made to correct it.}
\item Page 106: ...These figures {\color{red}show} the distribution...
\item Page 106: ...in the {\color{red}negative log likelihood} minimisation...
\item Page 108: ...Similarly the mass shifts vary between the two, {\color{red}but are consistent within the quoted uncertainties.}
\item Page 110: ...{\color{red}The} relative charmless efficiency...
\item Page 110: ...dependency {\color{red}on the position} in phase-space...
\item Page 111: ...dependence on the position {\color{red}in phase-space}...
\item Page 113: ... is found, {\color{red}however the possible resulting systematic uncertainty is quantified in Sec.~5.7.1.}

\end{itemize}
\section{Mass fit to $B^+ \rightarrow D_s^+ \phi$ candidates}

\begin{itemize}


\item Page 126: ...pair {\color{red}from} the $\phi$ meson, $m(K^+K^-)$, and the cosine of {\color{red}the} angle $\cos\theta_{K}$...
\item Page 126: ...The total extended {\color{red}negative log likelihood} (NLL) for...
\item Page 127: Added reference for MINOS.
\item Page 127: Added to fig 6.1 caption: {\color{red}The category labelled `$D_s^+\phi$ inner' refers to the invariant mass range $|m(K^+K^-)-m(\phi)|<10\,MeV/c^2$ and `$D_s^+\phi$  outer' to $10<|m(K^+K^-)-m(\phi)|<40\,MeV/c^2$.}
\item Page 130: ... {\color{red}For each contribution, the fractions of each decay} expected in the...
\item Page 130: ...The maximum differences {\color{red}between the fully simulated and LAURA++ fractions} are included...
\item Page 130: ...half the difference  {\color{red}between the $f_{0}^{0}(980)$ and $a_{0}^{0}(980)$ fractions}... 
\item Page 134: ...These PDFs are {\color{red}updated} to...
\item Page 139: ...so the {\color{red}parameter} would be expected...
\item Page 140: {\color{blue} change colour scheme to highlight signal component}
\item Page 151: ...The {\color{red}significance of} the measured branching fraction, $BF(B^+\rightarrow D_s^+\phi) = (1.2^{+1.6}_{-1.4} \pm 0.8  \pm 0.1)\times 10^{-7}$, is {\color{red}not large enough to constitute evidence} for the $B^+\rightarrow D_s^+\phi$ decay and {\color{red}the branching fraction is consistent with a value of zero}...
\item {\color{blue} improve summary of toys}

\end{itemize}
\section{Conclusions}

\begin{itemize}
\item {\color{blue}Add conclusions}
\item Page 119: {\color{blue} Phi could also be annihilation}

\end{itemize}
\section{Appendix A}

\begin{itemize}
\item Page 161: swapped figures A.3 and A.4
\item Page 168: {\color{blue} Check numbers in table... 5.6 out of order}

\end{itemize}
\section{Appendix C}

\begin{itemize}
\item Page 176: {\color{blue} add description about figures like in C.2}



\end{itemize}
\section{Appendix D}

\begin{itemize}
\item Page 180: ...the {\color{red}pull} is defined...  
\item Page 180: ...different $D_s^+$ decay modes are shown in {\color{red}Fig. D.3}...


\end{itemize}
\section{Appendix D}

\begin{itemize}
\item {\color{blue} Change fit colour scheme}

\end{itemize}
\section{References}


\begin{itemize}
\item Page 189: Ref [1] ...JHEP 01 (2018) {\color{red}131}...
\item Page 191: Ref [23] ...{\color{red}P. Dirac}, The quantum...
\item Page 199: Ref [199]  Add name to reference




\end{itemize}















\end{document}
