\documentclass[12pt]{article}
%\documentclass[12pt]{../ociamthesis}
\usepackage{color}
%\usepackage{hyperref}
\usepackage[margin=1.0in]{geometry}
\usepackage{amsmath}
\title{Minor corrections\\
Thesis: Rare hadronic decays of B mesons at LHCb}
\author{Tom Hadavizadeh}

\begin{document}
\maketitle

This document lists the changes that have been made to the thesis as minor corrections. Changes are highlighted in red. Page numbers refer to the version of the thesis used in the Viva. 

%%%%%%%%%%%%%%%%%%%%%%%%%%%%%%%%%%%%%%%%%%%%%%%%%%%%%
\section{Acknowledgements}

\begin{itemize}
\item {\color{blue} Add acknowledgements section}
\item {\color{blue} Add quote}

\end{itemize}
\section{Introduction}



\begin{itemize}
\item Page 4: ... have {\color{red}an} electromagnetic charge...
\item Page 4: ... there exists {\color{red}an} antimatter...
\item Page 6: ... described {\color{red}as} a non-Abelian...
\item Page 13: ... production {\color{red}of} Gravitons...
\end{itemize}

\section{Theory}
\begin{itemize}
\item Page 15: ... Ground state hadrons {\color{red}containing a single b-quark} can only...
\item Page 15: Added citation for Fermi coupling.
\item Page 16: ... decay {\color{red}was} measured ... 
\item Page 17: {\color{blue}cite ratio}
\item Page 18: Added the definitions of $\lambda$, $A$, $\rho$, $\eta$ in terms of $V_{qq}$
\item Page 18: Double checked sizes of diagonal CKM elements
\item Page 18: {\color{blue} add description of unitarity triangle}
\item Page 19: {\color{blue} How do $K^{0}$ and $\bar{K}^{0}$ mix into $K_{S}^{0}$ and $K_{L}^{0}$}
\item Page 19: Added details of the different types of CP violation and stated that direct CP violation is the only relevant method.
\item Page 20: {\color{blue} explain why colour suppressed or favoured}
\item Page 21: ...expensive {\color{red}and} requires the...
\item Page 22: ...mass {\color{red}of the} charged lepton...

\item Page 22: {\color{blue} what allows?}
\item Page 22: {\color{blue} what do you mean by mutually exclusive?}
\item Page 23: Added minus sign to limit order of magnitude $6.1\times10^{{\color{red}-}7}$
\item Page 23: .. this can {\color{red}potentially} limit the sensitivity {\color{red}to} the...
\item Page 23: {\color{blue}I think you should emphasis that this is in fact the final state you are looking for}
\item Page 24: removed {\color{red}dominantly}
\item Page 27: ... SM values given {\color{red}in Table 2.2} ...
\end{itemize}
\section{The LHCb experiment}

\begin{itemize}
\item Page 30: ... at the {\color{red}shallowest and deepest} points...
\item Page 30: {\color{blue} Add LEP reference}
\item Page 34: {\color{blue} Add LHCb detector reference}
\item Page 35: {\color{blue} Add description of T1, T2, T3 to figure caption}
\item Page 36: {\color{blue} Add reference for figure}
\item Page 39: {\color{blue} Define FPGA}
\item Page 42: {\color{blue} which of the planes X1, U, V, or X2 are shown in Figure 3.9?}
\item Page 46: ...in the {\color{red}maximum} value...
\item Page 48: Defined calorimeter acronyms
\item Page 51: ...to determine {\color{red}if} the innermost...
\item Page 53: ...trigger is composed {\color{red}of} two parts...
\item Page 54: {\color{blue} You could add a table with the thresholds for completeness}
\item Page 54: ...pattern recognition {\color{red}to} identify track candidates...
\item Page 55: {\color{blue}Can you show how this variable distinguishes between each b-hadron and background. Or add reference}
\end{itemize}

\section{Event selection}

\begin{itemize}
\item Page 64: {\color{blue} Which normalisation final states are you considering?}
\item Page 65: {\color{blue} Which samples? Clarify}
\item Page 65: ...Samples are {\color{red}generated} assuming...
\item Page 66: {\color{blue}What are the final states? Clarify}
\item Page 66: Removed {\color{red}The event selection aims to reduce overall rate of collisions, whilst maximising the signal efficiency.} 
\item Page 66: {\color{blue}Which triggers are you therefore using in your analysis? clarify.}
\item Page 66: {\color{blue}Clarify what is meant by hits in final paragraph.} 
\item Page 67: {\color{blue} Thresholds of what? Energy? Please clarify}
\item Page 67: ...if the {\color{red}deposit} matched to a reconstructed...
\item Page 68: ...hadronic {\color{red}hardware} trigger (\texttt{L0Hadron})...
\item Page 68: {\color{blue}Define L0Global using logic.}
\item Page 70: {\color{blue}Check why numbers don't add up}
\item Page 72: ...products {\color{red}with respect to the proton beam axis}... 
\item Page 74: {\color{blue}Add figure to show why track chi2 cut is 4}
\item Page 76: {\color{blue}What is the rough purity of the PIDK cut?}
\item Page 79: {\color{blue}Reword description of asymmetric sidebands.}
\item Page 80: {\color{blue}Clarify wht residual yields vary so much in caption}
\item Page 81: {\color{blue}Explain shape in caption}
\item Page 82: {\color{blue}Clarify the source of peaking structures of invariant masses}
\item Page 84: {\color{blue}Add introduction to give more context to MVA methods, i.e. more normal to train using MC, but this requires separate PID and MVA optimsation}
\item Page 88: {\color{blue} Add small amount of detail giving context and justification, i.e. DLL just specific detectors, ProbNN uses all}
\item Page 90: {\color{blue}Why are pT and $\chi^{2}_{FD}$ used to do efficiencies.}
\item Page 90: {\color{blue}Add reference to page where efficiencies are discussed.}
\item Page 91: {\color{blue}Add justification for $a=5$ from Punzi paper.}
\item Page 91: ...Requirements are {\color{red}placed} on the impact... 
\item Page 93: Caption: The {\color{red}$\phi$ (left) and $D_{s}^{+}$ (right)} invariant mass
\item Page 93: {\color{blue} Much slower rate: explain}
\item Page 94: Caption: The {\color{red}$D^{0}$ (left) and $D_{s}^{+}$ (right)} invariant mass
\item Page 95: Caption: The {\color{red}$K^+K^-$ (left) and $D_{s}^{+}$ (right)} invariant mass

\end{itemize}
\section{Mass fit to $B^+ \rightarrow D_s^+ K^+K^-$ candidates}

\begin{itemize}
\item Page 106: ...The parameters of interest {\color{red}in the two fits are}...
\item Page 106: ...These figures {\color{red}show} the distribution...
\item Page 106: ...in the {\color{red}negative log likelihood} minimisation...
\item Page 108: ...Similarly the mass shifts vary between the two, {\color{red}but are consistent within the quoted uncertainties.}
\item Page 110: ...{\color{red}The} relative charmless efficiency...
\item Page 110: ...dependency {\color{red}on the position} in phase-space...
\item Page 111: ...dependence on the position {\color{red}in phase-space}...
\item Page 112: {\color{blue} As a function??}
\item Page 113: {\color{blue} Can you show the agreement?}
\item Page 116: {\color{blue}Add the extent of this variation} 
\item Page 119: {\color{blue} Phi could also be annihilation}

\end{itemize}
\section{Mass fit to $B^+ \rightarrow D_s^+ \phi$ candidates}

\begin{itemize}


\item Page 126: ...pair {\color{red}from} the $\phi$ meson, $m(K^+K^-)$, and the cosine of {\color{red}the} angle $\cos\theta_{K}$...
\item Page 126: ...The total extended {\color{red}negative log likelihood} (NLL) for...
\item Page 127: {\color{blue} cite minos}
\item Page 127: {\color{blue} add description of categories in figure.}
\item Page 130: {\color{blue} Which fractions? Difference between which fractions? Clarify.}
\item Page 134: ...These PDFs are {\color{red}updated} to...
\item Page 139: ...so the {\color{red}parameter} would be expected...
\item Page 140: {\color{blue} change colour scheme to highlight signal component}
\item Page 151: ...The {\color{red}significance of} the measured branching fraction, $BF(B^+\rightarrow D_s^+\phi) = (1.2^{+1.6}_{-1.4} \pm 0.8  \pm 0.1)\times 10^{-7}$, is {\color{red}not large enough to constitute evidence} for the $B^+\rightarrow D_s^+\phi$ decay and {\color{red}the branching fraction is consistent with a value of zero}...

\end{itemize}
\section{Conclusions}

\begin{itemize}
\item Add conclusions

\end{itemize}
\section{Appendix A}

\begin{itemize}
\item Page 161: swapped figures A.3 and A.4
\item Page 168: {\color{blue} Check numbers in table... 5.6 out of order}

\end{itemize}
\section{Appendix C}

\begin{itemize}
\item Page 176: {\color{blue} add description about figures like in C.2}



\end{itemize}
\section{Appendix D}

\begin{itemize}
\item Page 180: ...the {\color{red}pull} is defined...  
\item Page 180: ...different $D_s^+$ decay modes are shown in {\color{red}Fig. D.3}...


\end{itemize}
\section{Appendix D}

\begin{itemize}
\item {\color{blue} Change fit colour scheme}

\end{itemize}
\section{References}


\begin{itemize}
\item Page 189: Ref [1] ...JHEP 01 (2018) {\color{red}131}...
\item Page 191: Ref [23] ...{\color{red}P. Dirac}, The quantum...
\item Page 199: Ref [199]  Add name to reference




\end{itemize}















\end{document}
